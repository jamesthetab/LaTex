I begin by recalling a result from \ref{DS} giving the Hochschild homology of a group ring in terms of Group homology of related rings. I then mirror this proof in the Iwasawa case.

We transform right-modules into left ones by introducing the opposite action. Where $kG\cong kG^{\text{op}}$ via $g\rightarrow g^{-1}$, thus $k(G \text x G)\cong kG \otimes kG^{\text{op}}$. Moreover, $kG$ as a $KG \otimes kG^{\text{op}}$-module (or equivalently a $G$-$G$-bimodule) is isomorphic to $k\uparrow_{\Delta (G)}^{G\text x G}$ where 

$$\Delta (G) = \{(g,g)\in G \text x G, \,g\in G\}$$ 

is a diagonal subgroup of $G\text x G$.

Thus, 

\begin{eqnarray}
\nonumber HH_*(kG,M)   &\cong&   Tor_*^{ kG \otimes kG^{\text{op}}}(kG,M) \text{ definition}\\
\nonumber                            &\cong&   Tor_*^{k(G\text x G)} (kG,M)\\
\nonumber 			  &\cong&   Tor_*^{k(G\text x G)} (k\uparrow_{\Delta (G)}^{G\text x G}, M)
\end{eqnarray}

Working with resolutions, $k(G\text x G)$ is a free, hence projective $\Delta(G)$-module, meaning $- \otimes_{k\Delta (G)} k(G\text x G)$ is exact:

\begin{eqnarray}
\nonumber 	HH_*(kG,M) 		&\cong&   Tor_*^{k\Delta (G)}(k, M_{\Delta(G)})
\end{eqnarray}

where $M_{\Delta (G)}$ denotes $M$ considered as a $\Delta (G)$-module - this conjugation action by elements of $G$ may be written $\overline M$:
\begin{eqnarray}
\nonumber 	HH_*(kG,M) 		&\cong&   Tor_*^{kG}(k, \overline M) \\
\nonumber			&\cong&   H_*(G,\overline M)
\end{eqnarray}
%%%%%%%%%%%%%%%%%%%%%%%%%%%%%%%%%%%%%%%%%%%%%%%%%%%%%%%%%%%%

\subsection*{Iwasawa Completion of Hochschild Homology\label{completed}}

I now recreate this construction replacing tensor products by completed tensor products (to give "$Tor$" groups over Iwasawa algebras) and introducing inverse limits where necessary. I establish two preliminary lemmas:

\subsubsection*{Lemma ($\Lambda_G$ as an induced module\label{IM}):}
\emph{Completed induction (via completed tensor product) gives,
$$ \ZP\uparrow_{\Lambda_{\Delta (G)}}^{\Lambda_G \widehat \otimes \Lambda_G^{op}}\cong \Lambda_G$$}

\subsubsection*{Proof}

In other words I need to show $[\ZP] \widehat \otimes_{\Lambda_{\Delta (G)}} [\Lambda_G \otimes_{\ZP} \Lambda_G^{op}] \cong \Lambda_G$ as a $\Lambda_G \otimes_{\ZP} \Lambda_G$-module.

Define the bihomorphism $\alpha$ as follows:

\begin{eqnarray}
\nonumber \alpha:  [\ZP]  \text{ x }[\Lambda_G \otimes_{\ZP} \Lambda_G^{op}] &\rightarrow& \Lambda_G\\
\nonumber 		[\mu ] \text{ x } [a \otimes b^{-1}] &\rightarrow& \mu (a b^{-1})
\end{eqnarray}

$\alpha$ is indeed a bihomorphism, and we see that for any other bihomomorphism, $\phi :  [\ZP]  \text{ x }[\Lambda_G \otimes_{\ZP} \Lambda_G^{op}] \rightarrow R$, the trivial action of $\Lambda_G \otimes_{\ZP} \Lambda_G^{op}$ on $\ZP$ gives

\begin{eqnarray}
\nonumber \phi([\mu] \text{ x } [a\otimes b^{-1}]) &=& \phi( [\mu] \text{ x } [ga\otimes (gb^{-1})] \,\forall g\in G\\
\nonumber							&=& \phi([\mu] \text{ x } [i\otimes (ab^{-1})] \\
\nonumber							&=& \phi([i] \text{ x } [i\otimes \mu (ab^{-1})] 
\end{eqnarray}

Referring to definition \ref{CTP}, the completed tensor product is defined by universality with respect to bihomomorphisms, and we may uniquely define $\ZP$-homomorphism, $g:\Lambda_G\rightarrow R$ by taking $g(\mu (ab^{-1})) = \phi([\mu] \text{ X } [a\otimes b^{-1}])$, then $\phi = g \circ \alpha$ and the following commutes:

$$  \begin{array}{ccc}
     [\ZP]  \text{ x }[\Lambda_G \otimes_{\ZP} \Lambda_G^{op}] & \xrightarrow{\alpha} & \Lambda_G \\ 
     &    \phi \searrow& \downarrow g \\ 
    \circlearrowright  &  & R \\ 
  \end{array}$$


\subsubsection*{Lemma (Exact Tensors\label{ET}):}
\emph{As a functor $-\widehat \otimes_{\Lambda_{\Delta (G)}} (\Lambda_G \widehat \otimes \Lambda_G^{op})$ is exact. Thus, $\Lambda_G \widehat \otimes \Lambda_G^{op}$ is a "projective" $\Lambda_{\Delta (G)}$-module.}

\subsubsection*{Proof}

$(\Lambda_G \widehat \otimes \Lambda_G^{op})$ is a free, hence projective, hence flat $\Lambda_{\Delta (G)}$-module. Alternatively, the completed tensor product functor is formed from usual ones, each of which is an exact functor.



\subsubsection*{Definition}
\emph{The n-th completed Hochschild homology of an Iwasawa Algebra $\Lambda_G$ with coefficient in the bimodule $M$ is:}
\begin{eqnarray}
\nonumber HH_n(\Lambda_G, M) &=& Tor_n^{\Lambda_G \widehat \otimes \Lambda_G^{op}}(\Lambda_G, M) \text{definition}\\
\nonumber 				      &=& Tor_n^{\Lambda_{\Delta(G)}} (\ZP,\overline M) \text{using Lemma 1 and Lemma 2}\\
\nonumber 				    &=& Tor_n^{{\Delta(G)}} (\ZP,\overline M) \text{from \ref{4.6.1}}\\
\nonumber 				    &=& H_n(G, \overline M)
\end{eqnarray}

Writing $G=\varprojlim (G/U)$ and $M=\varprojlim \overline {\ZP[G/U]} = \varprojlim A_i$ we see that the conjugation actions are compatible in the sense of \ref{COGS}:

For $U\leq V$ where $U,V <G$ are open normal subgroups, then $U$ is normal in $V$, and moreover, $G/V = (G/U)/(V/U)$, thus to show general compatibility it is sufficient to consider the case $H=G/U\leq G$ for some $U<G$ normal.

We need that conjugation is complatible with 
\begin{eqnarray}
\nonumber G &\xrightarrow{\phi}& G/U\\
\nonumber g &\rightarrow& gU
\end{eqnarray}
Hence we require $\forall m\in kG,\, g\in G$, 
$$g\circ m \xrightarrow{\phi} \phi(g) \circ \phi (m)$$
i.e.
$$gmg^{-1} U = (gU).(mU).(g^{-1} U)$$
But this is immediate since $U$ is normal in $G$.

Invoking, \ref{LHG}, we may take out a single inverse limit:
$$HH_*(\Lambda_G,\Lambda_G) = \varprojlim H_*(G/U , \ZP [G/U])$$
Hence,
\begin{eqnarray}
\nonumber HH_*(\Lambda_G) &=& HH_*(\varprojlim \ZP[G/U])\\
\nonumber				&=& \varprojlim HH_*(\ZP[G/U])\\
\nonumber    &=& \varprojlim \bigoplus_{\gamma \in C_{G/U}} HH_*(\ZP[G/U])_\gamma\\
\nonumber HH_1(\Lambda_G)				&=& \varprojlim \bigoplus_{\gamma\in C_{G/U}} (Z(g_\gamma))^{ab}
\end{eqnarray}
where for each term in inverse limit, we have a sum over conjugacy classes. In $n=1$ case this reduces to a sum of centralisers, $Z(g_\gamma)\leq G/U$


\subsection{What is the topology underlying this inverse limit?}

We may describe a topology which recovers this inverse limit using the standard concepts of product and quotient topology.

If we interpret $HH_n(\Lambda_G)$ as some quotient of the tensor product $(\Lambda_G)^{\otimes n}$, then   $HH_n(\Lambda_G)$ is the quotient of some Cartesian product $(\Lambda_G)^{ n}$, and if powers of $J = \text{Ker }(\ZP [[G]]\rightarrow \FP)$ give a topology on $\Lambda_G$ taking $\vert\vert \lambda \vert \vert = p^-{k}, \text{ where } x\in J^k - J^{k+1}$.

In the case of interest to us here, $HH_1(\Lambda_G) \cong [\Lambda_G \otimes \Lambda_G]/ \sim$, and for $A\otimes B\in [J^i\otimes J^j] - [J^{i+1} \otimes J^j] - [J^i \otimes J^{j+1}]$, then $\vert\vert A\otimes B\vert\vert = p^{-(i+j)}$ is well defined and displays $HH_1(\Lambda_G)$ as the inverse limit $\varprojlim HH_1(\ZP [G/U])$.


%%%%%%%%%%%%%%%%%%%%%%%%%%%%%%%%%%%%%%%%%%%%%%
%%%%%%%%%%%%%%%%%%%%%%%%%%%%%%%%%%%%%%%%%%%%%%
%%%%%%%%%%%%%%%%%%%%%%%%%%%%%%%%%%%%%%%%%%%%%%

\subsection{Induced Maps}



Where an argument using markers shows for a finite group $G$ acting on $kG$ by conjugation - 
$$\bigoplus_{C\in ccl(G)} HH_*(kG,kG) = \bigoplus_{C\in ccl(G)} H_*(G,kC) = \bigoplus_{g_c\in ccl(G)}  H_*(G,k(G/ Z(g_c))) = \bigoplus_{g_c\in ccl(G)} H_* (Z(g_c),k)$$


%Recall setup:
%\begin{eqnarray}
%\nonumber \rho : \ZP& \hookrightarrow & \ZP^*\\
%\nonumber   n &\rightarrow & {(1+p)}^n\\
%\nonumber   \ZP &\rightarrow& 1+p\ZP
%\end{eqnarray}

Thus it is sufficient to understand how the quotient map on finite groups,  $G \rightarrow G/U$ induces to give a map

$$\bigoplus_{g_c\in ccl(G)} (Z(g_c))^{ab} = HH_*(kG) \rightarrow HH_*(k G/U) = \bigoplus_{f_c\in ccl(G/U)} (Z(f_c))^{ab}$$



The augmentation map gives rise to a $\Z G$-free resolution of $k$ in terms of augmentation ideal:

$$0\rightarrow IG \hookrightarrow kG \rightarrow k \rightarrow 0 $$

This allows us to calculate $H_1$ explicitly, and perform the induction giving rise to centralisers explicitly.

Since $$0\rightarrow H_1(G,B)\rightarrow B\otimes IG \rightarrow B\otimes \Z G \rightarrow H_0(G,B) \rightarrow 0$$
We have 
\begin{eqnarray}
\nonumber H_1(G,B) = \text{ker}(B\otimes _G IG \rightarrow B) = B\otimes IG/(IG)^2 &\cong& B\otimes G/G'\\
\nonumber \text{where the isomorphism is given on generators by}\\
\nonumber                        b\otimes (x-1) &\rightarrow& b\otimes x
\end{eqnarray}

Thus, we may view $H_1(G,k) \cong G/ G' = G^{ab}$ on generators in $IG$, $g-1\rightarrow g$.

The key calculation is done in the following table, recalling $k(G/Z(g_c)) = k\uparrow_{k Z(g_c)}^{kG}$, on RHS I follow a representative element which generates whole group.

  $$\begin{array}{cccc}
   H_1(G, k(G/Z(g_c)))        & =  & \text{ker} (IG\otimes_{kG} k(G/Z(g_c))\rightarrow kG\otimes_{kG} k(G/Z(g_c)) & \\ 
   & = &  \text{ker} (IG\otimes_{kG} [kG\otimes_{kZ(g_c)}k]\rightarrow  kG\otimes _{kG} [kG\otimes_{kZ(g_c)}k]) &  g-1\otimes kG \otimes k \\
    \vert \vert \wr                         &      & & \downarrow \\ 
    H_1(Z(g_C), k)                & =  & \text{ker} (IG\otimes_{kZ(g_c} k\rightarrow kG\otimes_{kZ(g_c)} k) & g-1 \otimes k \\ 
    \vert \vert  \wr                         &      & � &\downarrow \\ 
    (Z(g_c))^{ab}                    &   & � &  g \\ 
  \end{array}$$

\subsubsection{Conjugacy Classes\label{conjugacy}}

Consider

\begin{eqnarray}
\nonumber G  &\rightarrow& G/U\\
\nonumber g &\rightarrow& gU
\end{eqnarray}

Then if $f$ and $h$ are conjugate, meaning there exists a $g$ such that $f = g^{-1} hg$, then certainly $fU = g^{-1}hg U$, and since $U$ is normal in G, this gives $fU = g^{-1}U. hU. gU$, and so $fU$ is conjugate to $hU$ in $G/U$.

$$f, h \text{    conj in   } G \Longrightarrow fU, hU \text{    conj in    } G/U$$

But if elements with representatives $fU$, and $hU$ are conjugate, we can only deduce there exists a $g$ such that $fU = (gU)^{-1}(hU)(gU)$, thus $U= f^{-1}g^{-1}hg U$ which gives $ f^{-1}g^{-1}hg \in U$, but is not necessarily the identity. i.e.

$$fU, hU \text{  conj in  } G/U   \nRightarrow  f, h \text{   conj in   } G$$
meaning that the \textit{ conjugacy classes in $G$ may fuse when we pass to $G/U$}.


We may now combine these ideas with the explicit maps between Hochschild homology groups and direct sums over conjugacy classes of abelianisations of conjugacy classes given above.


It is sufficient to consider what happens to each summand over conjugacy classes, so wlog consider the representative $(g_1,g)$ where $g\in (Z(g_1))$, meaning $g_1gg_1^{-1} = g$. Certainly then $g_1UgUg_1^{-1}U = gU$ and thus $gU\in (Z(g_1 U))$ giving meaning to the chain:


%$$\bigoplus (Z(g_*))^{ab} \cong \bigoplus H_1(Z(g_*),k) \cong \bigoplus H_1(G,k(G/Z(g_*))) \rightarrow \bigoplus H_1(G/U,k((G/U)/Z(g_*U))) \cong \bigoplus H_1(Z(g_*U),k) \cong \bigoplus (Z(g_*U))^{ab}$$

%$$(g_1,g) \rightarrow (g_1, g-1\otimes k) \rightarrow (g_1, g-1\otimes kG\otimes k)\rightarrow (g_1 U, gU - U \otimes k G/U \otimes k)\rightarrow (g_1U, gU)$$



  $$\begin{array}{cc}
    \bigoplus (Z(g_*))^{ab} 				& (g_1,g) \\ 
    \vert\vert\wr 						& \downarrow \\ 
    \bigoplus H_1(Z(g_*),k) 			&  (g_1, g-1\otimes k) \\ 
    \vert\vert\wr 						& \downarrow \\ 
    \bigoplus H_1(G,k(G/Z(g_*)))		& (g_1, g-1\otimes kG\otimes k) \\ 
    \downarrow	 					& \downarrow \\ 
    \bigoplus H_1(G/U,k((G/U)/Z(g_*U)))  	& (g_1 U, gU - U \otimes k G/U \otimes k) \\ 
    \vert\vert\wr 						& \downarrow \\ 
    \bigoplus H_1(Z(g_*U),k)	 		& (g_1U, gU-U\otimes k) \\ 
    \vert\vert\wr 						& \downarrow \\ 
    \bigoplus (Z(g_*U))^{ab}			&  (g_1U, gU) \\ 
  \end{array}$$


We are actually dealing with elements in the abelianization here, so we need to check $(g_1,g)\rightarrow (g_1 U, gU)$ is well defined up to multiplication by commutators. \emph{By $g\in [Z(g_1)]^{ab}$ we are referring to a class $gV$ where $$V = \{a b a^{-1} b^{-1} \vert a, b \in Z(g_1)\} = \{a b a^{-1} b^{-1} \vert ag_1 = g_1 a \,\&\, b g_1 = g_1 b\} = \{a b a^{-1} b^{-1} \vert [a,g_1]= i \,\&\, [b, g_1] = i \}  $$}

\emph{Then $gU\in [Z(g_1U)]^{ab}$ we are referring to a class $gUW$ where $$W =  \{a b a^{-1} b^{-1}U \vert aUg_1U = g_1U aU \,\&\, bU g_1U = g_1U bU\} = \{a b a^{-1} b^{-1}U \vert [a,g_1]\in U \,\&\, [b, g_1] \in U  \}  \supset V  $$ hence $gVU\subset gW$ and so the map is well defined.}





The fusion mentioned above means there might be more than one contribution to each summand in the quotient, suppose $g_1,\dots , g_n$ are representatives of distinct conjugacy classes in $G$, but are all conjugate to each other (and to $f=g_1U$ say) in $G/U$, we can think of the summands lying over each other in the following sense:


\begin{eqnarray}
\nonumber &\underbrace{[Z(g_1)]^{ab} \bigoplus  \dots \bigoplus [Z(g_n)]^{ab}} & \bigoplus \dots\\
\nonumber &[Z(f)]^{ab}& \bigoplus \dots
\end{eqnarray}


\subsubsection{Hochschild Homology under Quotient Map - Special Conjugacy Classes}

The lack of canonical representatives for conjugacy classes makes the choices in the representation arbitrary and thus only defined up to conjugacy.

Take representatives of the conjugacy classes of $G/U$:

$$\{ f_1\in C_1, \dots f_n \in C_n\}$$

Moreover choose representatives of the conjugacy classes of $G$:

$$\{g_{(1,1)}\in C_{(1,1)}, \dots g_{(1,m_1)}\in C_{(1,m_1)}, g_{(2,1)}\in C_{(2,1)}, \dots g_{(n,1)}\in C_{(n,1)}, \dots g_{(n,m_n)}\in C_{(n,m_n)} \}$$

Where the projections, $\{g_{(1,1)}U, \dots ,g_{(1,m_1)}U\}$ are not only conjugate to each other, lying in $C_1$, conjugate to $f_1$, but in fact map to $f_1U$ under projection (otherwise we might have to conjugate elements to get them lying in correct centraliser).

Similarly, for all $i = 1,2,\dots, n$,  $\{g_{(i,1)}U, \dots ,g_{(i,m_i)}U\}$ are all conjugate to each other, lying in $C_i$, projecting to $f_i$.

Then, for $h_{(i,j)}\in Z(g_{(i,j)})$ we have the map:

$$\begin{array}{cccc}
    HH_1(kG) 	& \left( \underbrace{h_{(1,1)}, \dots , h_{(1,m_1)} }\right) & , \dots , & \left( \underbrace{h_{(n,1)}, \dots , h_{(n,m_n)}}\right) \\ 
          		& � & � & �  \\ 
    \downarrow	& � & \downarrow & �  \\ 
    			& � & � & �  \\ 
    HH_1(kG/U)	& \left(\underbrace{(\prod h_{(1,1)} \dots  h_{(1,m_1)} U)} \right) & ,\dots ,& \left(\underbrace{(\prod h_{(n,1)} \dots  h_{(n,m_n)}U)}\right) \\ 
    			&\text{Conj. Class } C_1 & � & \text{Conj. Class } C_n \\ 
  \end{array}$$





\subsubsection{Hochschild Homology under Quotient Map - General Conjugacy Classes}

From \ref{conjugacy}, if $gU$ and $hU$ are conjugate in $G/U$, then $g$ is conjugate to a product of $h$ with an element of $U$, in $G$. Conversly, if $g$ and $h$ are conjugate then $gU$ and $huU = hU$ are conjugate for all $u\in U$. Thus elements of $G$ which become conjugate to $g$ when passing to the quotient is $X_g = \{g_i U \vert g_i \text{ is conjugate to } g \text{ in } G\}$.

Understanding how elements fuse requires us to consider how many elements of $X_g$ are already conjugate (to avoid repetition in counting).

In the commutative case, when each element is it's own conjugacy class, $\vert U \vert$ conjugacy classes (elements of $G$) lie over each conjugacy class (elements of $G/U$).

Since $U$ is normal it can be written as a sum of conjugacy classes of $G$: $U= \bigcup_{i=1}^n C_i$. I begin by observing that elements mapping to $\overline {id}$ in $G/U$ are infact just elements of $U\subset G$, a union of ($n$) conjugacy classes. The same is true for other conjugacy classes - $X_g = \cup \{g_i U \}$ is a sum of conjugacy classes. This is seem by considering the conjugate by $h\in G$ os a general element $g_iu \in X_g$:

\begin{eqnarray}
\nonumber h^{-1}(g_i u ) h &=& (h^{-1} g_i h ) ( h^{-1} u h)\\
\nonumber                             &=& g_j. u'\,\,\,\, (\text{since $U$ is normal})\\
\nonumber			  &\in& X_g
\end{eqnarray}

We may visualize conjugacy class fusion for projection $G$ to $G/U$, where $U= \bigcup_{i=1}^n C_i$.

$$
\begin{array}{ccccccc}
 id\in C_1 & \dots   &C_n  & g\in C_{n+1} &\dots  &C_m  &\dots \\
 \searrow & \downarrow  & \swarrow &  \searrow & \downarrow  & \swarrow   & \\
  & B_1  &  &  & B_2 &  & \\
  &  id.U &  &  & g.U &  & 
\end{array}
$$

Let us denote by $  \begin{array}{c}
    h \\ 
    �\vert\\ 
    g \\ 
  \end{array}$ the component of the direct sum lying over conjugacy class of $g$, where $h\in Z(g)$.

Now, to understand the inverse limit I ask if the connecting map,

$$\bigoplus_{g_c\in ccl(G)} (Z(g_c))^{ab} \rightarrow  \bigoplus_{f_c\in ccl(G/U)} (Z(f_c))^{ab}$$ need be onto under extension of map $ \begin{array}{c}
    h \\ 
    �\vert\\ 
    g \\ 
  \end{array}
\rightarrow
  \begin{array}{c}
    hU \\ 
    �\vert\\ 
    gU \\ 
  \end{array}$.

We observe, $  \begin{array}{c}
    hU \\ 
    �\vert\\ 
    gU \\ 
  \end{array}$ is equivalent to 
  \begin{eqnarray}
  \nonumber hU.gU.h^{-1} U &=& gU\\
  \nonumber hgh^{-1} &=& gu \text{ for some } u\in U
  \end{eqnarray}
  
  Whereas, $  \begin{array}{c}
    h \\ 
    �\vert\\ 
    g \\ 
  \end{array}$ is equivalent to 
  \begin{eqnarray}
  \nonumber hgh^{-1}  &=& g\\
    \end{eqnarray}
    giving image under projection $  \begin{array}{c}
    hU \\ 
    �\vert\\ 
    gU \\ 
  \end{array}$ with $$ hgh^{-1} = g.id $$
  
  Thus the possibility of hitting different values for commutator $[g,h^{-1}]$ could only arise as the image of a representative in a different conjugacy class which only becomes conjugate after projection. Without loss of generality we may reduce to the case of the contribution from $n$ where $g=nu$ is the representative we want:
  
  For  $\begin{array}{c}
    m\\ 
    �\vert\\ 
    n\\ 
  \end{array}$, where $[m,n] = m^{-1}n^{-1}mn = 1$ we get a contribution,  $  \begin{array}{c}
    mU \\ 
    �\vert\\ 
    nuU \\ 
  \end{array}$, where $m^{-1}(nu)^{-1}.m.nu = m^{-1}u^{-1}n^{-1}mnu = m^{-1}u^{-1}mu = 1$ if and only if  $u\in Z(m)$. This has reduced the question to how the decomposition of $G$ into left cosets of $Z(m)$ interplays with conjugacy class structure.
  
 This gives that the number of values, $u'$ attainable as a commutator under projection of $\begin{array}{c}
    m\\ 
    �\vert\\ 
    n\\ 
  \end{array}$ is less than or equal to the number of conjugacy classes lying above $gU$ which is less than or equal to $\vert U\vert$ with equality only if $G$ is abelain (and then we see that this is indeed sufficient for the connecting map to be onto).
  
  I now give a non-commutative example to show this map is not commutative in general:
  
  \subsubsection{Dihedral Example\label{dihedral}}
  Take $G = D_8 = \{i, \alpha, \alpha^2, \alpha^3, \beta, \alpha\beta, \alpha^2 \beta, \alpha^3\beta| \alpha^4 = id = \beta^2, \beta\alpha = \alpha^3\beta\}$
  
  Then this has conjugacy classes,
  \begin{eqnarray}
  \nonumber   C_1 &=& \{id \}  \\
    \nonumber  C_2 &=& \{\alpha, \alpha^3 \}   \\
      \nonumber    C_3 &=& \{\alpha^2 \} \\
        \nonumber    C_4 &=& \{\beta, \alpha^2\beta \} \\
          \nonumber     C_5 &=& \{\alpha\beta, \alpha^3\beta \}
          \end{eqnarray}
          
          moreover, $A= C_1 \cup C_3 = \{id, \alpha^2\}$  is a normal subgroup. We see that the map $G\rightarrow G/A \cong V$ (non-cyclc, commutative group of order 4).
          
          $$  \begin{array}{ccccccc}
    G & � & C_1 & C_3 & C_2 & C_4 & C_5 \\ 
    \downarrow& � & \downarrow & \swarrow & \downarrow & \downarrow & \downarrow \\ 
    G/A & � & \overline{id}\equiv\overline{\alpha^2} &  & \overline{\alpha}\equiv\overline{\alpha^3}  & \overline{\beta}\equiv \overline{\alpha^2 \beta}& \overline{\alpha\beta}\equiv \overline{\alpha^3 \beta} \\ 
  \end{array}$$

Hence, $Z(\alpha\beta) = (id, \alpha\beta, \alpha^3\beta, \alpha^2)\rightarrow (\overline{id}, \overline{\alpha\beta})\in Z(\overline{\alpha\beta})$.

But the Vier group is Abelian, thus  for all $v\in V$, $Z(v) = V$. Therefore, $Z(\overline{\alpha\beta}) = (\overline{id}, \overline{\alpha}, \overline{\beta}, \overline{\alpha\beta})$. 

So there is no pre-image of $\begin{array}{c}
    \overline{\alpha}\\ 
    �\vert\\ 
    \overline{\alpha\beta}\\ 
  \end{array}$  showing that connecting maps in the inverse limit defining the Hochschild homology of Iwasawa completed group algebras need not be surjective.

%%%%%%%%%%%%%%%%%%%%%%%%%%%%%%%%%%%%%%%%%%%%%%%%%%%%%%%%%%%%
%%%%%%%%%%%%%%%%%%%%%%%%%%%%%%%%%%%%%%%%%%%%%%%%%%%%%%%%%%%%
%%%%%%%%%%%%%%%%%%%%%%%%%%%%%%%%%%%%%%%%%%%%%%%%%%%%%%%%%%%%


%%%%%%%%%%%%%%%%%%%%%%%%%%%%%%%%%%%%%%%%%%%%%%
\subsection{Does $HH_1(\Lambda_G)$ Reduce to a Direct Product over Conjugacy Classes?\label{reduce}}

%%%%%%%%%%%%%%%%%%%%%%%%%%%%%%%%%%%%%%%%%%%%%%
%%%%%%%%%%%%%%%%%%%%%%%%%%%%%%%%%%%%%%%%%%%%%%
%%%%%%%%%%%%%%%%%%%%%%%%%%%%%%%%%%%%%%%%%%%%%%

In \ref{no compose over conj} I showed that the Hochschild homology of an Iwasawa algebra  is not contained in the direct product over conjugacy classes,
$$ \varprojlim HH_1{\ZP [G/U]} =  HH_1 (\widehat{(\ZP G)}) \subset \prod_{\gamma , \text{conj. class}} HH_1 (\ZP G)_\gamma = \prod_{\gamma , \text{conj. class}} (Z(\gamma))^{ab}$$ 

When $G$ is infinite the first homology group is no longer just the abelianisation. The universal defining property of $\varprojlim$ gives an injective map,
$$\prod_{\gamma , \text{conj. class}} (Z(\gamma))^{ab}\hookrightarrow \varprojlim (HH_1({\ZP [G/U]} ))=  HH_1 (\widehat{(\ZP G)})$$ I now ask if this is an Isomorphism:

\subsubsection*{Claim (Completion of Homology):}
\emph{$$\prod_{\gamma , \text{conj. class}} (Z(\gamma))^{ab} \cong \varprojlim (HH_1({\ZP [G/U]} ))=  HH_1 (\widehat{(\ZP G)})$$}

Consider the Classical $\ZP$-extension - $G\cong \ZP$ then everything is abelian and connecting maps are just projections modulo $p^n$ (see section \ref{fusion} for justification of this), so we are reduced to giving an element in $\varprojlim (HH_1({\ZP [\Z / p^n \Z]})) = \varprojlim \bigoplus_{\Z / p^n\Z} ({\Z / p^n\Z})$ not in $\prod_{\ZP} (\ZP)$. I will use the notation $\begin{array}{c}
(\dots, a, \dots)     \\
   | \\
   (\dots, b, \dots)  
\end{array}$ to denote the element $a$ lying in coordinate $b$. For $p$ odd, $\sum_{i=0}^{p^n-1} (-1)^i = 1$, giving consistency of following elements to build up an element $x$ in $\varprojlim \bigoplus_{\Z / p^n\Z} ({\Z / p^n\Z})$.


$$\begin{array}{cc}
  (\Z /p^0\Z):    		&     \begin{array}{c}
(1)     \\
   | \\
   (0)  
\end{array} \\
  	\uparrow		&      \\
    (\Z /p^1\Z):  				&   \begin{array}{c}
(\overbrace{1,-1,1,\dots, -1,1})     \\
   | \\
   (0, 0+1.p^0, 0+2.p^0,\dots, 0+(p-1).p^0)  
\end{array}
 \\
    \uparrow                &   \\
      (\Z /p^2\Z):          &  \begin{array}{c}
(\overbrace{1,-1,1,\dots, 1}, \overbrace{1,-1,1,\dots, 1}, \dots ,\overbrace{1,-1,1,\dots, 1})     \\
   | \\
   (0, 0+1.p^1,\dots, 0+(p-1).p^1, 1, 1+1.p^1,\dots, 1+(p-1).p^1,\dots, \\
   p-1, p-1+1.p^1,\dots, p-1+(p-1).p^1,)  
\end{array} \\
      \uparrow        &   \\
      \vdots            &  
\end{array}$$

We now study if $x\in \prod_{\ZP} (\ZP)$. If it was, then element lying over $a_0+a_1 p +\dots +a_n p^n$ is $sgn(a_0+a_1+\dots+a_n) = (-1)^{a_0+a_1+\dots+a_n}$. By Fermat's Little Theorem this is just $(-1)^{a_0+a_1 p +\dots +a_n p^n}$. Since $a^x = exp(ln\,a . x)$, whether this continues to be defined for all exponentials in $\ZP$, not necessarily finite, is equivalent to whether $ln\, (-1)$ is defined. But $ln:1+A_0\rightarrow A_0$ where $A_0 = \{ x \text{ such that }||x||\leq p\} = p\ZP$. But $-1\in \ZP - p\ZP$ and so the logarithm, and hence the exponential are not defined. Hence, $x\in \varprojlim \bigoplus_{\Z / p^n\Z} ({\Z / p^n\Z})$ does not come from an element of $x\in \prod_{\ZP} (\ZP)$, as required.









%%%%%%%%%%%%%%%%%%%%%%%%%%%%%%%%%%%%%%%%%%%%%%%%%
%%%%%%%%%%%%%%%%%%%%%%%%%%%%%%%%%%%%%%%%%%%%%%%%%


%%%%%%%%%%%%%%%%%%%%%%%%%%%%%%%%%%%%%%%%%%%%%%
%%%%%%%%%%%%%%%%%%%%%%%%%%%%%%%%%%%%%%%%%%%%%%
%%%%%%%%%%%%%%%%%%%%%%%%%%%%%%%%%%%%%%%%%%%%%%
\subsection{Products of Finite Groups}

Given a family of groups, there are many different ways of combining these to get another group.

Let $\{G_{\lambda} \vert \lambda \in \Lambda \}$ be a given set of groups over a (not necessarily finite) indexing set $\Lambda$.

\subsubsection{Catesian Product}
The \textbf{cartesian} (or unrestricted direct) product,

$$C = \text{Cr}_{\lambda\in\Lambda} G_{\lambda}$$

is the group with underlying set the product of the $G_{\lambda}$ as sets - vectors whose $\lambda$-component lies in $G_{\lambda}$, and whose group operation is defined by multiplication of components: $(g_\lambda)(h_\lambda) = (g_\lambda h_\lambda)$ for $g_\lambda, h_\lambda \in G_\lambda$.

The cartesian product may also be given a universal construction as follows:

Define the projections $\pi_\lambda : G \rightarrow G_\lambda$ by taking $\pi\lambda (x)$ to be the $\lambda$-th component of $x$. $\pi_\lambda$ is a homomorphism for each $\lambda$.

Given a family of homomorphisms $\phi_\lambda: H\rightarrow G_\lambda$ for some group $H$, there exists a unique homomorphism $\phi:H\rightarrow G$ such that $\phi \circ \pi_\lambda = \phi_\lambda$ for all $\lambda$.

The existence of the map $\phi$ gives the following commutative diagram:
$$\begin{array}{ccc}
  H&   &   \\
  \downarrow \phi&  \searrow \phi_\lambda &   \\
  G&  \xrightarrow{\pi_\lambda} & G_\lambda   
\end{array}$$


\subsubsection{Direct Product}
The subset of the $(g_\lambda)$ such that $g_\lambda = a_\lambda$ for almost all $\lambda$, so sequence is trivial with finitely many exceptions, is called the \textbf{external direct} product,

$$D = \text{Dr}_{\lambda\in\Lambda} G_{\lambda}$$

The $G_\lambda$ are referred to as the direct factors. $D$ is a normal subgroup of $C$, and equal for a finite indexing set $\Lambda = \{  \lambda_1, \lambda_2, \dots, \lambda_n \}$. The products are then written,

$$D = G_{\lambda_1}\, X \, G_{\lambda_2} \, X\, \dots \, X \, G_{\lambda_n}$$

And should the groups be abelian, and written additively,

$$D = G_{\lambda_1}\, \oplus \, G_{\lambda_2}\, \oplus \, \dots \, \oplus  \, G_{\lambda_n}$$

\subsubsection{Free Products}

A free product of the family $G_\lambda$ is a group $G$ together with a collection of homomorphisms $l_\lambda: G_\lambda \rightarrow G$ with universal property  that for another such group $H$ and set of homomorphisms  $l_\lambda: G_\lambda \rightarrow H$, there is a unique homomorphism of groups $\phi:G\rightarrow H$ such that $l_\lambda \circ \phi = \phi_\lambda$, and the following diagram commutes:

$$\begin{array}{ccc}
  G_\lambda& \xrightarrow{l_\lambda}  & G  \\
  \downarrow \phi_\lambda&  \swarrow \phi&   \\
  H&  &   
\end{array}$$

The free product is sometimes denoted $F = \text{Fr}_{\lambda\in\Lambda} G_{\lambda}$.

From the category-theoretic viewpoint this free product is a coproduct in the category of groups (the product being the cartesian product defined above).

For each $\lambda$, taking $H = G_\lambda$, and maps $\phi_\lambda = \text{id}$, other $\phi_\mu$ trivial, we see that $\phi\circ l_\lambda = \text{id}\vert _{G_\lambda}$ and so each $l_\lambda$ is injective.

Uniqueness of this construction is clear from the universal property. Existence can be shown using an explicit description on words, where letters are taken from the disjoint union of the $G_\lambda$ (we are only working up to Isomorphism of $G_\lambda$, so may assume they do not intersect), products are by juxtapostion, and the only relations are contracting/ expanding letters lying in the same group $G_\lambda$, and absorbing/inserting identity elements.

If $\Lambda$ is finte, $\Lambda = \{  \lambda_1, \lambda_2, \dots, \lambda_n \}$, it is usual to write product as

$$G_{\lambda_1}\, \star  \, G_{\lambda_2} \, \star \, \dots \, \star \, G_{\lambda_n}$$

The free product has the following properties:
\begin{enumerate}

\item For $\{G_{\lambda} \vert \lambda \in \Lambda \}$ there is a natural epimorphism of groups from $\text{Fr}_{\lambda\in\Lambda} G_\lambda$ to $\text{Dr}_{\lambda\in\Lambda} G_\lambda$, defined by *************, and with kernel ******************.

\item $(\text{Fr}_{\lambda\in\Lambda} G_\lambda)^{\text{ab}}\cong $$\text{Dr}_{\lambda\in\Lambda} (G_\lambda)^{\text{ab}}$.

\item So for abelian families $G_\lambda$, we have $(\text{Fr}_{\lambda\in\Lambda} G_\lambda)^\text{ab} \cong \text{Dr}_{\lambda\in\Lambda} G_\lambda$. For a finite index we have:

$$( \bigstar_{\lambda\in\Lambda} G_\lambda )^{\text{ab}} \cong \bigoplus_{\lambda\in\Lambda} G_\lambda$$
\end{enumerate}

%%%%%%%%%%%%%%%%%%%%%
\subsection{Products of Profinite Groups}
%%%%%%%%%%%%%%%%%%%%%


Following Melnikov I explain how the topology of profinte groups and spaces can be used to control what is happening in the inverse limits of such products, and how such products may themselves be represented as products.

I will discuss p-groups and pro-p groups in this section although the same ideas could be applied to $\mathbb c$ and pro-$\mathbb c$ groups for $\mathbb c$ a full class of finite groups (meaning it is closed under subgroups, homomorphic images and group extensions).













%%%%%%%%%%%%%%%%%%%%%
%%%%%%%%%%%%%%%%%%%%%
%%%%%%%%%%%%%%%%%%%%%
\subsubsection{When the indexing familiy is discrete}

\subsubsection{When the indexing family varies continuously over a profinite space}

\subsubsection{Inverse limits and free products}

\subsubsection{Reformulation of Completed Hochschild Homology}

We begin by explaining why Inverse Limits and Abelianization Commute. We can of course identify $H_1(G,\ZP)$ with $G^{ab}$ for any group G, and this gives,

\begin{eqnarray}
\nonumber \varprojlim_{U \ong G} (G/U)^{ab} &=& \varprojlim_{U \ong G} H_1 (G/U, \ZP)\\
\nonumber 						&=& [\lim_{U \ong G} H^1(G/U,\ZP)]^*\\
\nonumber						&=&  [\lim_{U \ong G} H^1(G/U,\ZP^{G/U})]^*\\
\nonumber 						&=& [H^1(G,\ZP)]^*\\
\nonumber						&=& H_1(G,\ZP)\\
\nonumber						&=& G^{ab}
\end{eqnarray}



\begin{eqnarray}
\nonumber \varprojlim_{U \ong G} \bigoplus_{g_U\in \ccl G/U)} Z(g_U)^{ab} &=& \varprojlim_{U \ong G} [\bigstar_{g_U\in \ccl G/U)} Z(g_U)]^{ab}\\
\nonumber                 &=& [\varprojlim_{U \ong G} \bigstar_{g_U\in \ccl G/U)} Z(g_U)]^{ab}\\
\nonumber                 &=&  [\bigstar_{g\in \ccl G)} \varprojlim_{U \ong G}  Z(gU)]^{ab}\\
\nonumber                 &=&  [\bigstar_{g\in \ccl G)}   Z(g)]^{ab}
\end{eqnarray}




%%%%%%%%%%%%%%%%%%%%%%%%%%%%%%%%%%%%%%%%%%%%%%


