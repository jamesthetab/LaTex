\section{Lie Algebra Cohomology}\label{df4}

\subsection{Definition}
Recall, a Lie algebra $\mathcal G/K$ is a K vector space, with a
bilinear map $[\, ,\, ]:\mathcal G \, X\, \mathcal G \mapsto
\mathcal G$ known as the Lie Bracket. The Lie bracket is
anti-commutative, and non-associative the Jacobi identity, see
\ref{df2.3.11} shows how far from being associative it is:
$[X,[Y,Z]]+[Y,[Z,X]]+[Z,[X,Y]] = 0$, hence
$$[X,[Y,Z]]=[[X,Y],Z]+[[Z,X],Y]$$
The Lie algebra is called abelian if $[x,y]=0 \text{ for all } x,y
\in \mathcal G$

Consider the functor $L:\mathcal G\mapsto \mathcal G_{ab} =
\mathcal G /{[\mathcal G, \mathcal G]}$ which maps to the largest
abelian quotient.


For M a K-vector space, we define the n-fold tensor product: $T_nM
= M\otimes_k M\otimes_k \dots \otimes_k M$, and the Tensor algebra
(a free K-algebra over M):$$TM = \oplus_{n=0}^\infty T_nM$$, where
the multiplication is induced from product of elements in the
a-fold tensor algebra and the b-fold tensor algebra as being the
natural gluing to an element in the $a+b$-fold tensor product:
$$(m_1\otimes\dots\otimes m_a).(m'_1\otimes\dots\otimes m'_b) =
(m_1\otimes\dots\otimes m_a\otimes m'_1\otimes\dots\otimes m'_b)$$

For our Lie algebra, we can now define the \textbf{Universal
Enveloping Algebra $U\mathcal G$ of $\mathcal G$}, where
$$U\mathcal G = {T\mathcal G}/I \text{ where } I=<x\otimes y -
y\otimes x -[x,y]>$$ By forming the quotient of the tensor algebra
by the ideal I we are effectively forcing this new algebra to be
abelian.

Recall, when defining group cohomology we used the group ring
which is characterised by being adjoint to the unit functor: the
functor : rings $\mapsto$ groups : $\Lambda \mapsto$ Group of
Units of $\Lambda$. T has an analogous property - that the
universal enveloping algebra functor $\mathcal G \mapsto U
\mathcal G$ is the left adjoint to L, and plays a similar role in
Lie algebra cohomology.

For a Lie algebra $\mathcal G$ over K, and for a $\mathcal
G$-module A, we define the \textbf{$n^{th}$ cohomology group of
$\mathcal G$ with coefficients in A},
$$\mathbf{H^n(\mathcal G, A) = Ext^n_{U\mathcal G} (K,A)}$$
where we regard K as a trivial $\mathcal G$-module.

This leads to similar results as for groups, for example $H^0$
picks out invariant elements: $H^0(\mathcal G,A) = \{ a\in A |
x\circ a = 0 \,\forall\, x\in \mathcal G \}$.

Again, $H^n(\mathcal G,A)$ may be computed via any $\mathcal
G$-projective resolution of K. We could try to mirror the
construction of the Bar complex for groups, see \ref{df3.3}, but
there exists one which is simpler for doing calculations:

\subsection{The Koszul Complex}\label{df4.2} As a reference,
see for example the paper of Lazard, chapter 5, section 1.3.3.

The $n^{th}$ exterior power of a module M, a universal object on
the set of alternating maps is defined as a quotient of the n-fold
tensor product:
$$E_nM = {T_nM}/{<x_1\otimes x_2\otimes \dots \otimes x_n -
(sgn\,\sigma )x_{\sigma 1} \otimes \dots\otimes x_{\sigma n}>}$$
where $sgn\,\sigma$ is the parity of $\sigma$. As examples:
$E_0V\cong K$, and $E_1V\cong V$.

\subsection{Alternating Property}\label{df4.2.1}
Notice $<x_1 , \dots , x_i ,\dots , x_j , \dots ,x_n > \cong <x_1
, \dots , x_j ,\dots , x_i , \dots ,x_n >$, since a transposition
has parity $-1$. Hence, when 2 elements in a bracket are equal,
the bracket is its own negation, hence zero. The linearity of
$<-,\dots ,->$ allows us to expand to leave a sum of brackets
consisting of basis elements. When $n>dim\,V$ it is inevitable
that in each bracket of the expansion basis elements are repeated,
hence, $E_nV=0$.

As when we were forming the tensor algebra, we define the Exterior
Algebra, with induced composition by
$$EM = \oplus_{n=0}^\infty E_n M$$

\subsection{Construction of Complex}\label{df4.2.2}
Set $V=\text{ underlying vector space of }\mathcal G$, where
$\mathcal G$ is considered as a vector space over K but forgetting
the composition given by Lie bracket.

Define $C_n$ as the tensor product of the universal enveloping
algebra with the $n^{th}$ exterior power of V:
$$C_n = U\mathcal G \otimes_K E_n V = \{ u<x_1, \dots , x_n>\}$$

\subsection{Lemma}\label{4.2.3}
\begin{eqnarray}
\nonumber \text{Let }d_n &:& C_n\mapsto C_{n-1}\\
\nonumber                &:& <x_1,\dots ,x_n>\mapsto
\sum_{i=1}^n{(-1)}^{i+1} x_i<x_1,\dots ,\widehat x_i, \dots , x_n>
+\\
\nonumber               &&\,\,\,\, \sum_{1\leq i<j\leq n}
{(-1)}^{i+j} <[x_i,x_j],x_1,\dots , \widehat {x_i},\dots
,\widehat{x_j}, \dots x_n >
\end{eqnarray}
and we keep $U\mathcal G$ fixed. Then, by mechanical checking we
get that the chain $\dots \rightarrow C_n \rightarrow ^{d_n}
C_{n-1}\rightarrow \dots \rightarrow C_1\rightarrow C_0 =
U\mathcal G$ is a $\mathcal G$-projective resolution of K, where
$C_0\rightarrow K$ is the augmentation map, similar to the
construction for groups.

Since, from \ref{df4.2.1}, $E_kV=o$ for $k>dim\,V$, taking homology
of above chain we have:
$$H^k(\mathcal G,A) = 0 \text{ for } k\geq dim\,V +1$$

\subsection{Example}\label{df4.2.4}
To illustrate the methods discussed above I will now give an
example from Totaro's paper. Working with $\mathcal G$ a finite
dimensional Lie algebra over K. In this case i work with homology
groups to ease notation, but of course a dual theory applies for
cohomology.

Firstly, recall the definition of the Euler character:

\emph{Claim:
\begin{eqnarray}
\nonumber\text{Euler Character }&=&\text{ Alt Sum Homologies of Complexes}\\
\nonumber                       &=&\text{ Alt Sum of dimensions of vector spaces in a complex used to calculate Homology}
\end{eqnarray}}
Consider calculating complex:
$$\dots \rightarrow^\epsilon D\rightarrow^\delta
C\rightarrow^\gamma B\rightarrow ^\beta A\rightarrow ^\alpha 0$$
Then finiteness allows us to use the rank-nullity identity:
\begin{eqnarray}
\nonumber \text{Alt Sum dimensions} &=&
dim\,A-dim\,B+dim\,C-dim\,D+\dots\\
\nonumber &=& (Null\,\alpha)-(Im\,\beta
+Null\,\beta)+(Im\,\gamma+Null\,\gamma)-(Im\,\delta
+Null\,\delta)+\dots\\
\nonumber &=& (Null\,\alpha -Im\,\beta)-(Null\,\beta -Im\,\gamma)+\dots\text{(regrouping)}\\
\nonumber &=& \sum {(-1)}^i dim|H_*|
\end{eqnarray}
We are now ready to prove the following proposition:
\subsection*{Proposition}
\emph{For $\mathcal G$ a finite dimensional Lie algebra (of
dimension N), and M a finite dimensional representation of
$\mathcal G$. The Euler Characteristic,
$$\chi(\mathcal G,M) = \sum_{i\geq o} {dim}_k H_i(\mathcal G,M) =
\begin{cases}
  0 & \mathcal G\neq 0 \\
  dim\,M & \mathcal G=0 \\
\end{cases}$$}

\subsection*{Proof}


\underline{$\mathcal G = 0$}, $H_0 (\mathcal G, M) = M$

\underline{$\mathcal G \neq 0$}, To calculate homology take $Tor$
groups of the Koszul complex:
$$M\otimes_{U\mathcal G} C_n = M\otimes_{U\mathcal G} (U\mathcal
G\otimes_k E_nV) = M\otimes_k E_nV$$ Hence, we are reduced to
calculating the homology of the complex of finite dimensional
vector spaces,
$$\dots \rightarrow E_2\mathcal G\otimes_kM\rightarrow \mathcal
G\otimes_k M\rightarrow M\rightarrow 0$$

The theory of Exterior Algebras gives $dim E_n\mathcal G\otimes_k
M = dim\, M.dim\,E_n\mathcal G = dim\,M.\left(%
\begin{array}{c}
  dim \mathcal G \\
  n \\
\end{array}%
\right)$, hence invoking the claim gives
$$\chi(\mathcal G, M) = \sum_{i=0}^N {(-1)}^i dim\, M.\left(%
\begin{array}{c}
  N \\
  i \\
\end{array}%
\right) = dim\, M.{(1-1)}^N = 0 \text{ for } N>0$$



\subsection{Structure Theorem}\label{df4.3}
For completeness I include the statement of the
Poincare-Birkhoff-Witt Theorem which gives the structure of
$U\mathcal G$. I will use this theorem to check my calculation of
the Universal Enveloping Algebra of my induced Lie Algebra in my
calculations section, chapter 7.

Let$\{ e_\alpha \}$ be a k-basis of $\mathcal G$.

For each sequence $I=(\alpha_1, \dots , \alpha_p )$ denote
$e_{\alpha_1}\dots e_{\alpha_p} \in U\mathcal G$ by $e_I$. We call
a sequence $I$ increasing if $\alpha_1\leq \dots \leq \alpha_p$
(and we have a convention that $\phi$ is increasing, where $e_\phi
= 1$).

\subsection*{Theorem: Poincare-Birkhoff-Witt}

\emph{If $\mathcal G$ is a free k-module, then $U\mathcal G$ is
also a free k-module. Moreover, if $\{ e_\alpha \}$ is an ordered
basis of $\mathcal G$, then the elements $e_I$ with $I$ an an
increasing sequence form a basis of $u\mathcal G$.}

\subsection{Calculating Ext Groups using Cohomology}\label{df4.4}
Let $M$ and $N$ be left $\mathcal G$-modules: k-modules with a
k-bilinear product \\ $\mathcal G \otimes_k M\mapsto M$, written
$x\otimes m\mapsto xm$, such that:
$$[x,y]m = x(ym)-y(xm) \,\,\forall\, x,y \in \mathcal G \text{
and }m\in M$$ Then $Hom_k(M,N)$ is a $\mathcal G$-module by
$$(xf)(m) = xf(m)-f(xm)\,\forall\,x\in\mathcal G,\, m\in M$$
since
\begin{eqnarray}
\nonumber ([x,y]f)(m) &=& [x,y]f(m) - f([x,y]m)\\
\nonumber             &=& x(yf(m))-y(xf(m))-f(x(ym)-y(xm))\\
\nonumber             &=& xyf(m)-xf(ym)-yf(xm)+f(xym)\\
\nonumber             & & -yxf(m) +yf(xm)+xf(ym)-f(yxm)\\
\nonumber             &=& \{x(yf)\}(m)-\{y(xf)\}(m)
\end{eqnarray}

Hence it is a $\mathcal G$-module. We now deduce a crucial
isomorphism of $\mathcal G$-modules:

\subsection{Lemma}\label{df4.4.1}
\emph{$$Hom_\mathcal G (M,N) \cong {Hom_k(M,N)}^\mathcal G$$}

\subsection*{Proof}
Given $f\in Hom_\mathcal G (M,N)$ we have \begin{enumerate}
    \item $f(gm)=gf(m)$ since f is a $\mathcal G$ homomorphism.
    \item $f(km) = k f(m)$ since $\mathcal G$ is a k-module.
\end{enumerate}
We now define $\Theta \in Hom_\mathcal G (M,N) \cong {Hom_k(M,N)}$
via
$$[\Theta f ](m)= f(m)$$
Then condition 2 gives $\Theta f \in Hom_k(M,N)$, and moreover
condition 1 gives invariance under $\mathcal G$, hence $\Theta f
\in {Hom_k(M,N)}^\mathcal G$. Clearly, any such map
${Hom_k(M,N)}^\mathcal G$ arises in this way and we may define an
inverse morphism by inclusion, giving the required isomorphism.

\subsection{Corollary}\label{df4.4.2}
\emph{The natural isomorphism above may be extended to a natural
isomorphism of functors:
$$ Ext_{U\mathcal G}^* (M,N) \cong H^*_{Lie}(\mathcal G,
Hom_k(M,N))$$ Note, by the Global Dimension Theorem, see [W], this
implies that the global dimension of $U\mathcal G$ equals the Lie
algebra cohomological dimension of $\mathcal G$.}

\subsection*{Proof}
Clearly, from definition of cohomology and of $Ext$ groups as a
measure of how far from being exact $Hom$ is,
\begin{eqnarray}
\nonumber Ext_{U\mathcal G}^0 (M,N) &\cong& Hom_\mathcal G (M,N)\\
\nonumber                           &\cong& Hom_k(M,N)^\mathcal G \text{ from Lemma above,}\\
\nonumber                           &\cong& \text{Invariants of
$Hom_k(M,N)$ under action of $\mathcal G$}\\
\nonumber                           &\cong& H^0_{Lie}(\mathcal G,
Hom_k(M,N))\end{eqnarray}

I now quote [HS], chapter IV, Proposition 5.7:

\emph{If T is left-exact, then $R^0T$ is naturally equivalent to
T.}

Hence, since we know $H^*$ and $Ext$ are both derived functors,
the isomorphism may be induced from isomorphism of their values
for $n=0$ which recovers the original additive functors from which
the derived functors are constructed.



\section{Spectral Sequences}\label{df4.5} Following the treatment
in [W] I will introduce the theory of spectral sequences up to the
theory behind the Hoschild-Serre spectral sequence as a special
case of the Grothendieck spectral sequence. I will avoid reference
to the exact couple.

The aim of this chapter is to interpret and apply the formula:

$$E_2^{pq} = Ext_R^p (A,Ext_R^q(S,B)) \Longrightarrow Ext^{p+q}_R
(A,B)$$

\subsection{Spectral Sequences as Approximations of the Total
Homology}\label{df4.5.1}

Recall, a \textbf{Double Complex} (bicomplex) in a category
$\mathcal A$ is a family $\{c_{p,q}\}$ of objects of $\mathcal A$,
together with maps:
$$d^h:c_{p,q}\rightarrow c_{p-1,q} \text{ and }
d^v:c_{p,q}\rightarrow c_{p,q-1}$$

such that $d^h\circ d^h = d^v\circ d^v = 0 = d^v \circ d^h +
d^h\circ d^v$ - an \emph{anticommutative lattice} with chains for rows
and columns.

We can now define a total complex $Tot^\Pi (C)$ by
$$Tot^\Pi (C)_n = \prod_{p+q = n} c_{p,q}$$ and the formula
$d=d^h +d^v$ defines maps:
$$d: Tot^\Pi (C)_n\rightarrow Tot^\Pi (C)_{n-1}$$
moreover,
$$d\circ d = (d^h+d^v)\circ (d^h+d^v) = d^h\circ d^h + d^v\circ
d^v +(d^h\circ d^v +d^v\circ d^h ) = 0$$ ie. anticommutivity
property gives that d is a differential.

Suppose double complex E consists of just two columns at p and
p-1:
\begin{equation}
\nonumber \begin{array}{ccc}
  \vdots      \,     &                         & \vdots \,\\
  \downarrow d^v  &                         & \downarrow d^v \\
  E_{p-1,2}        &\longleftarrow^{d^h}       & E_{p,2} \\
  \downarrow d^v  &                         &  \downarrow d^v \\
  E_{p-1,1}       &\longleftarrow^{d^h}         & E_{p,1} \\
  \downarrow d^v &                          & \downarrow d^v \\
  E_{p-1,0}        & \longleftarrow^{d^h}       & E_{p,0} \\
  \downarrow d^v  &                         & \downarrow d^v \\
  \vdots        \,   &                         & \vdots \,\\
\end{array}
\end{equation}

For fixed $n$, set $q=n-p$. Hence, an element of $Tot(E) =T$ is
represented by an element $(a,b)\in E_{p-1,q+1}\, X \, E_{pq}$.

Viewing columns as vertical chains we take homologies, with
notation:

\begin{eqnarray}
\nonumber E^1_{pq} &=& H( E^o_{pq} ) \text{ (taken vertically)}\\
\nonumber          &=& {ker\, d^v_{pq}} / {Im\, d^v_{pq+1}}
\end{eqnarray}
The horizontal maps $d^h$ induce maps between these homology
groups, since if $x\in ker\, d^v$:

$$d^v(d^h x) = -d^h (d^v x) = -d^h (0) = 0$$
Hence $d^h x \in ker\, d^v$, which induces $d^h : E^1_{p,q}
\rightarrow E^1_{p-1, q}$

We now take the horizontal homology to form $E^2$:
$$ E_{p-1,q+2}\, \text x \, E_{p,q+1} \longrightarrow^{d^v\, \text x \, d^h +d^v} E_{p-1,q+}\, \text x \, E_{p,q}
\longrightarrow^{d^v\, \text x \, d^h +d^v} E_{p-1,q}\, \text x \,
E_{p,q-1}$$

Then,
$$H_{p+q}(T) = \{(x,y)\,  | \, d^v x +d^h y = 0 \text{ and } d^v y = 0
\}$$ This gives,

\begin{eqnarray}
\nonumber E_{pq}^2 &=& \{ x\in E^1_{pq} \, |\, d^h x = 0\}\\
\nonumber          &=& \{ d^v x = 0 ,\, d^h x = 0 \}
                   \end{eqnarray}
This gives rise to the canonical injection:

\begin{eqnarray}
\nonumber E^2_{p-1,q+1} = \{x\in E^0_{p-1,q+1}\, |\, d^v x = 0,\,
d^h x = 0 \} &\hookrightarrow& H_{p+q} (T) = \{(x,y)\, | \, d^v x
+ d^h y =
0,\, d^v y = 0 \}\\
\nonumber x &\hookrightarrow& (x,0)
\end{eqnarray}

This is an injection with cokernel $\{ y\in E_{pq}\, |\, d^v y = 0
\text{ and } 0+d^h y = 0\} = E^2_{pq}$. Hence,
$$0\rightarrow E^2_{p-1,q+1} \rightarrow H_{p+q}(T) \rightarrow
E^2_{pq}\rightarrow 0$$

Hence, up to extension, $E^2$ gives homology of $T=Tot(E)$.\\
Continuing this analysis gives rise to the spectral sequence as an
approximation to homology of T.


\subsection{Definition of Spectral Sequence}\label{df4.5.2}

\subsection*{Definition}
A \textbf{homology spectral sequence} (starting with $E^a$) in an
abelian category $\mathcal A$ has the following structure:
\begin{enumerate}
    \item A family $\{ E^r_{pq}\}$ of objects of $\mathcal A$
    defined for all integers $p,q,$ and $r\geq a$.
    \item Maps $d^r_{pq} : E_{pq}^r \rightarrow E^r_{p-r, q+r-1}$
    that are differentials. ie. $d^r\circ d^r = 0$. In other
    words, lines of slope $-(r+1)/r$ of the lattice $E^r_{**}$
    form chain complexes
    \item There are isomorphisms between $E^{r+1}_{pq}$ and the
    homology of $E^r_{**}$ at the $pq$ spot. Hence,
    $$ E_{pq}^{r+1} \cong ker(d^r_{pq})/image(d^r_{p+r,q-r+1})$$
\end{enumerate}
Immediately we see that $E^{r+1}_{pq}$ is a subquotient of
$E_{pq}^r$. We define the total degree of a term $E_{pq}^r$ as
$n=p+q$ (as when defining the total complex), then, viewing the
double complex as a lattice, we see the terms of degree n lie on a
line of slope $-1$. Moreover, each differential $D_{pq}^r$
decreases the total degree by 1.

\subsection*{Definition}
A homology spectral sequence is said to be \textbf{bounded}, if
for each n there are only finitely many nonzero terms of total
degree n in $E^a_{**}$ (if true for one such $a$ true for all
greater ones). Then, for each $p \,\& \, q\,\exists \, r_0$ such
that $E_{pq}^r = E_{pq}^{r+1}\,\forall \, r\geq r_0$, and we write
$E_{pq}^\infty$ for this \emph{STABLE VALUE} of $E_{pq}^r$.

\subsection*{Definition}
A bounded spectral sequence \textbf{converges} to $H_*$ if we are
given a family of objects $H_n$, each having a \textit{finite}
filtration
$$0=F_sH_n\subseteq\dots\subseteq F_{p-1}H_n\subseteq F_p
H_n\subseteq F_{p+1}H_n\subseteq\dots \subseteq F_t H_n = H_n$$
and there exist isomorphisms $E_{pq}^\infty \cong
F_pH_{p+q}/F_{p-1}H_{p+q}$, this situation is \\ denoted
$$E_{pq}^a \Longrightarrow H_{p+q}$$

\subsection*{Definition}
A homology spectral sequence \textbf{collapses} at $E^r\,(r\geq
2)$ if there is exactly one nonzero row or column in the lattice
$\{E^r_{pq}\}$. If we know a collapsed sequence converges to $H_n$
it is easy to read these values off - $H_n$ is the unique nonzero
$E^r_{pq}$ with $p+q = n$. Compare this with the 2 column collapse
of \ref{df4.5.1}.

If the spectral sequence does not collapse at some finite level
then it becomes very difficult to manipulate.

\subsection*{Definition}
A spectral sequence is \textbf{regular} if for each pair $p,q$ the
differentials $d^r_{pq}$ are zero for all large r. ie if
$$Z_{pq}^\infty = \bigcap_{r=a}^\infty Z^r_{pq}=Z_{pq}^r$$ for all
large r. As we will see regularity is a very convenient condition
ensuring convergence.



\subsection{Spectral Sequences Arising from a
Filtration}\label{df4.5.3}

To every exhaustive filtration F of a chain complex \\ $\mathbf
C$: $\dots\subseteq F_{p-1}\mathbf C\subseteq F_p\mathbf
C\subseteq \dots$ (with $\mathbf C=\bigcup F_p \mathbf C$) we
construct an associated spectral sequence (without worrying about
its convergence properties).

\subsection*{Theorem}
\emph{A filtration F of a chain complex $\mathbf C$ naturally
determines a spectral sequence starting with \\ $E^0_{pq}=
F_pC_{p+q}/F_{p-1}C_{p+q}$ and $E^1_{pq} = H_{p+q}(E^0_{p*})$.}

\subsection{Construction}\label{df4.5.4}
Write $\eta_p : F_p\mathbf C \twoheadrightarrow F_p\mathbf C /
F_{p-1}\mathbf C = E^0_p$ - a surjection.

We now define \textit{cycles modulo $F_{p-r}\mathbf C$ }-
approximate cycles vanishing to a smaller set under the boundary
map.

$$A_p^r = \{ c\in F_p \mathbf C : d(\mathbf C)\in F_{p-r} \mathbf
C \}$$ and their images:
\begin{eqnarray}
\nonumber Z_p^r &=& \eta_p(A_p^r) \text{ in } E_r^0\\
\nonumber B_{p-r}^{r+1} &=& \eta_{p-r} (d(A_p^r)) \text{ in }
E^0_{p-r}
\end{eqnarray}

Taking $Z_p^\infty = \bigcap_{r=1}^\infty$ and $B_p^\infty =
\bigcup_{r=1}^\infty B_p^r$ we define a tower of subobjects of
$E_p^0$ -
$$0=B^0_p\subseteq B^1_p\subseteq\dots\subseteq
B^r_p\subseteq\dots\subseteq B_p^\infty\subseteq
Z_p^\infty\subseteq\dots\subseteq Z_p^r\subseteq\dots\subseteq
Z^1_p\subseteq Z^0_p = E^0_p$$ Since $A_p^r \cap F_{p-1}\mathbf C
= A_{p-1}^{r-1}$, hence $Z_p^r \cong A_p^r / A_{p-1}^{r-1}$ we
have
$$ E_p^r = \frac{Z_p^r}{B_p^r}\cong \frac{A^r_p + F_{p-1}(\mathbf C)}{d(A_{p+r-1}^{r-1})+F_{p-1}(\mathbf
C)}\cong \frac{A^r_p}{d(A^{r-1}_{p+r-1}) + A^{r-1}_{p-1}}$$and the
differential of $\mathbf C$ induces $d^r_p : E^r_p \rightarrow
E^r_{p-r}$ and the map $d$ determines isomorphisms $Z_p^r /
Z^{r+1}_p \cong B_{p-r}^{r+1} / B^r_{p-r}$ ($\star$).

The kernel of $d_p^r$ is
$$\frac{\{z\in A^r_p : d(z) \in d(A^{r-1}_{p-1}) + A^{r-1}_{p-r-1} \}}{d(A^{r-1}_{p+r-1}) +
A^{r-1}_{p-1}}=
\frac{A^{r-1}_{p-1}+A_p^{r+1}}{d(A^{r-1}_{p+r-1})+A^{r-1}_{p-1}}\cong
\frac{Z_p^{r+1}}{B^r_p}$$ and by ($\star$) this factors as

$$E^r_p = Z^r_p / B^r_p \rightarrow Z^r_p / Z^{r+1}_p\cong
B^{r+1}_{p-r} / B^r_{p-r} \hookrightarrow Z^r_{p-r} / B^r_{p-r} =
E^r_{p-r}$$

Hence image of $d^r_p$ is $B^{r+1}_{p-r} / B^r_{p-r}$, and
relabelling ($p+r$ instead of $p$) we have isomorphisms:
$$E_p^{r+1} = Z^{r+1}_p / B^{r+1}_p \cong
ker(d^r_p)/im(d^r_{p+r})$$

and hence get isomorphisms between $E^(r+1)$ and $H_* (E^r)$ which
completes the construction.



\subsection{Spectral Sequences of a Double Complex}\label{df4.5.5}
Given a Double Complex $\mathbf C$ we may collapse either rows or
columns to give two different filtrations of $\mathbf C$ each
giving rise to spectral sequences related to the homology of
$Tot(\mathbf C)$. The tactic here is to compare the two as a way
to calculate homology. Let $\mathbf C = C_{**}$ be a double
complex.

\subsection*{Definition}
We first consider \textbf{Filtration by Columns}.

Filter the total complex $Tot(C)$ by columns of C:

Set $F^I_n Tot(C)$ as the total complex of the double subcomplex
(copying $\mathbf C$ for the first n columns and 0 elsewhere)

$$\begin{array}{ccccccc}
  \dots & * & * & | & 0 & 0 & \dots \\
  \dots & * & * & | & 0 & 0 & \dots \\
  \dots & * & * & | & 0 & 0 & \dots \\
\end{array}$$

This filtration gives a spectral sequence $\{ {E^r_{pq}}^I \}$
with $\{ {E^0_{pq}}^I \} = C_{pq}$, where $d^0$ are just the
vertical differentials $d^v$ of $C$ -
$$ {E^1_{pq}}^I  = H^v_q ( C_{p*} )$$
and the horizontal differentials induce $d^1 : H^v_q (C_{p,*} )
\rightarrow H_q^v (C_{p-1,*})$ giving
$$ {E^2_{pq}}^I  = H^h_p H_q^v (C)$$

If $C$ is a first quadrat double complex. the filtration is
bounded, giving convergence of the spectral sequence:
$$ {E^2_{pq}}^I  = H^h_pH^v_q (C) \Longrightarrow H_{p+q}
(Tot(C))$$

Secondly consider \textbf{Filtration by Rows}.

Let ${F_n}^{II}(Tot(C))$ be he total complex of the subcomplex
formed by first n rows, and zero elsewhere:

$$\begin{array}{ccc}
  \vdots & \vdots & \vdots \\
  0 & 0 & 0 \\
  0 & 0 & 0 \\
  - & - & - \\
  * & * & * \\
  * & * & * \\
  \vdots & \vdots & \vdots \\
\end{array}$$

$E^0$ is calculated as $F_p Tot (C) / F_{p-1} Tot(C)$ in the row
$C_{*p}$, ${E_p^0}^{II} = C_{qp}$ and ${E_{pq}^1}^{II} =
H^h_q(C_{*p})$, and the vertical differentials $d^v$ induce $d^1$:
$${E_{pq}^2}^{II} = H_p^v H_q^h (C)$$
As above, the spectral sequence converges to $H_* Tot (C)$.




\subsection{Grothendieck Spectral Sequence}\label{df4.6}

\subsection{Theorem: Grothendieck}\label{df4.6.1}
\emph{Given $F : \mathcal U \mapsto \mathcal B,\, G:\mathcal B
\mapsto \mathcal C$, assume that if $I$ is an injective object of
$\mathcal U$, then $F(I)$ is G-acyclic. Then there is a spectral
sequence $\{ E_n (A) \}$ corresponding to each object $A$ of
$\mathcal U$, such that
$$E_1^{p,q} = (R^pG)(R^{q-p}F)(A) \Longrightarrow R^q(GF)(A)$$
which converges finitely to the graded object associated with
$\{R^q (GF)(A)\}$, suitably filtered.}

 To prove this we study first Filtration by Columns as in
 \ref{df4.5.4} to obtain \\$H^q(Tot\mathbf B)= R^q (GF)(A)$ where
 $\mathbf B$ is a double chain complex constructed from a base row
 of a resolution of A, and each column a resolution of elements in
 the initial resolution. We now explicitly calculate terms using Filtration by
 Rows. The technical hypotheses give convergence of the 2 spectral
 sequences to the same value. See [HS] for the details.

 \subsection{Application}\label{df4.6.2}
 Let N be a normal subgroup of K with quotient group Q:
 $$N\rightarrowtail^i K \twoheadrightarrow ^p Q$$
 Let $\mathcal U$ be the category of K-modules.

 Let $\mathcal B$ be the category of Q-modules.

 Let $\mathcal C$ be the category of Abelian Groups.

Define $F:\mathcal U\rightarrow B$ where $F(A) = Hom_N(\Z ,A) =
A^N$ - subgroup of A of elements fixed by N.

Similarly, $G:\mathcal B\rightarrow C$ where $G(B) = Hom_Q(\Z ,B)
= B^Q$

Then $A^N$ is a Q-module via $(\star ) (px)\circ a = xa,\,x\in K
,\, a\in A^N$ making F (and G) additive functors. Also,
$$GF(A) = Hom_K (\Z , A) = A^K$$

Given this structure, as an application of \ref{df4.6.1} we now
prove,

\subsection{Hochschild-Serre}\label{df4.6.3}
\emph{Let A be a K-module, then there is a natural action of Q on
the cohomology groups $H^m(N,A)$. Moreover, there is a spectral
sequence $\{E_n(A)\}$ such that
$$E_1^{p,q} = H^p(Q,H^{q-p}(N,A)) \Longrightarrow H^q (K,A)$$
which converges finitely to the graded group associated with $\{
H^q (K,A) \}$ suitably filtered.}

By $\star$, F preserves monomorphisms, hence injectives. In
particular, injectives are mapped to acyclics, and thus the
hypotheses of \ref{df4.6.2} are satisfied.

Since $\Z K$ is a free $\Z N$-module, a K-injective resolution of
A is also an N-injective resolution. Given any such K-injective
resolution of A, $I_0\rightarrow I_1\rightarrow I_2\rightarrow
\dots$ the complex $Hom_N(\Z ,A)\rightarrow Hom_N(\Z
,I_0)\rightarrow Hom_N(\Z ,I_1)\rightarrow \dots$ is a Q-complex
and so the groups $H^m(N,A)$ become Q-modules with
$$R^mF(A) = H^m(N,A)$$
Since, $R^mG(B) = H^m(Q,B)$ we have $R^m(GF)(A)=H^m(K,A)$.
Recalling \ref{df4.6.2} completes the proof.

Alternatively, since $N\rightarrow G\rightarrow G/N$ we can take
cohomolgy once more to state the theorem as:

\emph{For G a profinite group, N a closed normal subgroup of G,
$$ H^p(G/N , H^{q}(N,A)) \Longrightarrow_p H^q(G,A)$$}




\subsection{Sequences Formed by Terms of Low
Degree}\label{df4.6.4} Once we have the existence of a spectral
sequence $E_2^{pq} \Longrightarrow H^*$ we can look at the first
few terms to form short exact sequences.

\subsection*{Notation:}
A filtration on A is a family of subgroups ${\{A^n\}
}^\infty_{-\infty}$ of A with $A^{n+1} \subseteq A^n$,
$"A^{-\infty}" = \cup A^n = A$, $"A^\infty" = \cap A^n = \{ 0\} $.
These two objects together form a filtered abelian group.

The definition of $E_\infty^{p,q}$ gives
$$E_\infty^{p,q} = \frac{H^{p+q}(A)\cap H^*(A)^p}{H^{p+q}(A)\cap
H^*(A)^{p+1}}$$ Thus, for $p+q = n$ the $E_\infty^{p,q}$ are just
the composition factors in the filtration
$$H^n(A)\supseteq H^n(A)^1\supseteq H^n(A)^2\supseteq \dots
\supseteq H^n(A)^v\supseteq \dots$$ ie. $E_\infty^{p,n-p}$ is the
p-th composition factor in $H^n(A)$. Representing $E_r^{p,q}$ as a
lattice, $d_r^{p,q}$ is an arrow "going over $r$ and down $r-1$".

This interpretation immediately gives, dually to \ref{df4.5.1},
that when a \emph{spectral sequence degenerates except for columns
$j,j+1$ we have only 2 terms $E_2^{j+1,i-1}$ and $E_2^{0,i}$ in
the filtration of $H^{i+j}$}, hence:

\subsection*{Lemma}
\emph{If the spectral sequence degenerates as above, then:
$$0\rightarrow E_2^{j+1,i-1}\rightarrow H^{i+j}\rightarrow
E_2^{j,i}\rightarrow 0$$}
\subsection*{Example}
Invoking the Hochschild-Serre spectral sequence above, we have
if\\
$E_2^pq = H^p(G/N,H^{q-p}(N,W))$ vanishes for $p\neq 0,1$ (for
example if $G/N$ has cohomological dimension 1) then the Lemma
gives:
$$0\rightarrow H^1(G/N,H^{i-1}(N,W))\rightarrow
H^i(G,W)\rightarrow H^0(G/N,H^i(N,W))\rightarrow 0$$










%%%%%%%%%%%%%%%%%%%%%%%%%%%%%%%%%%%%%%%%%%%%%%%%%%%%%%%%%%%%%%%%%%%%%%%%%%%%%%%%%%%%%%%%%%%%%%%%%%%%%%%%%%%%%%%%%%%%
%%%%%%%%%%%%%%%%%%%%%%%%%%%%%%%%%%%%%%%%%%%%%%%%%%%%%%%%%%%%%%%%%%%%%%%%%%%%%%%%%%%%%%%%%%%%%%%%%%%%%%%%%%%%%%%%%%%%
%%%%%%%%%%%%%%%%%%%%%%%%%%%%%%%%%%%%%%%%%%%%%%%%%%%%%%%%%%%%%%%%%%%%%%%%%%%%%%%%%%%%%%%%%%%%%%%%%%%%%%%%%%%%%%%%%%%%
%%%%%%%%%%%%%%%%%%%%%%%%%%%%%%%%%%%%%%%%%%%%%%%%%%%%%%%%%%%%%%%%%%%%%%%%%%%%%%%%%%%%%%%%%%%%%%%%%%%%%%%%%%%%%%%%%%%%
%%%%%%%%%%%%%%%%%%%%%%%%%%%%%%%%%%%%%%%%%%%%%%%%%%%%%%%%%%%%%%%%%%%%%%%%%%%%%%%%%%%%%%%%%%%%%%%%%%%%%%%%%%%%%%%%%%%%
%%%%%%%%%%%%%%%%%%%%%%%%%%%%%%%%%%%%%%%%%%%%%%%%%%%%%%%%%%%%%%%%%%%%%%%%%%%%%%%%%%%%%%%%%%%%%%%%%%%%%%%%%%%%%%%%%%%%
%%%%%%%%%%%%%%%%%%%%%%%%%%%%%%%%%%%%%%%%%%%%%%%%%%%%%%%%%%%%%%%%%%%%%%%%%%%%%%%%%%%%%%%%%%%%%%%%%%%%%%%%%%%%%%%%%%%%
%%%%%%%%%%%%%%%%%%%%%%%%%%%%%%%%%%%%%%%%%%%%%%%%%%%%%%%%%%%%%%%%%%%%%%%%%%%%%%%%%%%%%%%%%%%%%%%%%%%%%%%%%%%%%%%%%%%%
