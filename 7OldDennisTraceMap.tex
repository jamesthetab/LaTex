\subsection{Homotopy Definition of Dennis Trace Map\label{DTM}}



Writing $G=GL(R)$ I now define the D.T.M., $\delta:K_n(R)\rightarrow HH_n(R)$ as the composition of maps defined in \ref{Quillen}, and $\chi:H_n(G,k)\hookrightarrow HH_n(kG)$ which  is inclusion as a direct summand by \ref {DSD}.

$$\begin{array}{ccc}
     	K_n(R)   		&\xrightarrow{\delta}	&HH_n(R)\\
          \|\text{ def.} 	&				&\uparrow \phi\\
      	\Pi_n(BG^+)	& 				&HH_n(kG)\\
	\downarrow "*"  &				&\uparrow\chi\\
	H_n(BG^+,k)     &\xrightarrow{\cong}& H_n(G,k) \\\end{array}$$

In the case $n=1$, we observe $K_1(R) = H_1(G,k)$, and indeed the composition is trivial, and so the D.T.M., $\delta$ is reduced to understanding inclusion in the direct sum, $\chi$, and the fusion map, $\phi$:

$$\delta = \phi\circ \chi: K_1(R) = H_1(G,k)\rightarrow HH_1(R)$$

This is all rather abstract so I now give an example as illustratation.

\subsubsection{Theorem (Example of Laurent Polynomials):}
\emph{Let k be a commutative ring, and $R = k[v, v^{-1}]$ the Laurent polynomial ring in one variable. From \ref{laurent}, 
\begin{itemize}
\item $HH_n(R) = 0\,\,\,\forall n>1$
\item $HH_1(R) = $ free $R$-module on generator $\mathfrak d v$.
\end{itemize} 
Then the Dennis Trace map sends the class of $v\in R^\text X\subset K_1(R)$ (embedded as a $1\text{ X }1$-matrix to the Kahler differential $v^{-1}\mathfrak d v\in HH_1(R)$, the logarithmic derivative.
\begin{eqnarray}
\nonumber \delta:R^\text X\subset K_1(R) &\rightarrow& HH_1(R)\\
\nonumber v                                            &\rightarrow& v^{-1}\mathfrak d v
\end{eqnarray}}

\subsubsection*{Proof}
Let $G$ be the infinite cyclic group with generator $v$, $G=<v>$. Then $G\subset R^\text X$, and the group algebra $kG = R$, thus the fusion map, $\phi = i$, and to compute the D.T.M. on the class of $[v]$ we need only look at the assembly map $\chi$.

In Hochschild Homology, $[v]$ is sent to the class of the cycle corresponding to $v$ in the chain complex $C_1(G)$, the first term in the Hochschild Chain Complex, which is just $v^{-1}\otimes v$ where we are thinking of $HH_*(kG)$ as $H_*(G,kG)$ where G acts by conjugation - see claim in the proof of \ref{DSD}. Identification of $HH_1(R)$ with Kahler differentials gives: $[v]\rightarrow v^{-1}\mathfrak d v$.

The case of Laurent polynomials is infact a universal example for the whole of the commutative case, giving a simple result on the direct summands $R^\text X\subset GL_1(R)\subset K_1(R)$, and $SK_1$, the kernel of the induced determinant map splitting $K_1(R) = R^\text X \oplus SK_1(R)$. The proof uses functoriality of K-groups and Homology, and Morita Equivalence, together with the method of calculating the inverse of a matrix using adjoints, see \cite{R} section 10.3.

\subsubsection{Corollary (Commutative Case):}
\emph{For $R$ a commutative $k$-algebra, the D.T.M. vanishes on $SK_1(R)$, and sends any $g\in R^\text X \subset K_1(R)$ to it's logarithmic derivative, $g^{-1}\mathfrak d g\in HH_1(R)$}

Putting these results together, and scaling a given matrix, $g$, to one of determinant 1, $\overline g$:

\subsubsection{Corollary (D.T.M. as a Determinant):\label{dtmdet}}
\emph{For R commutative, splitting an element of $K_1(R)$, considered as $g\in GL(R)$ into the pair $(\text{det }g, \overline g)$, the D.T.M., $\delta: g \rightarrow (\text{det }g)^{-1}\mathfrak d (\text{det }g)$.}

\bigskip

In the non-commutative case, the same arguments give $\delta:K_1(kG)\rightarrow HH_1(kG):M\rightarrow Tr(M^{-1}\otimes M)$. This tensor image still has the same derivative property, but the pair must now commute:

$$HH_1(R) = \{a\otimes b| ab=ba\}/(\sim)$$

This is indeed well defined since for N a commutator, $Tr(N^{-1}\otimes N) \equiv 0$.

The direct sum decomposition, \ref{DSD} still holds, but we must be careful, even in the commutative case to separate out the additive structure of $\mathfrak d$ from the multiplicative structure of $\bigoplus_{g\in G} G$: $\Theta: (\text{det }M)^{-1}\mathfrak d (\text{det } M) \nrightarrow ((\text{det }M)^{-1}, 1, \dots)$ as we must multiply out all pairs before applying $\Theta$. This is best illustrated by an example.

Take $G=C_5 = <h>$, the cyclic group of order 5 generated by h. Then we may calculate image of element $g= \left (\begin{array}{cc}
i& h+h^2\\
h & i+h^2+h^4\\
\end{array}\right )$, using that for $\lambda\in k$ such that $2.\lambda = 1$, we have $\lambda.(i+h).(i-h+h^2-h^3+h^4) = \lambda.2.i = \lambda$
\begin{eqnarray}
\nonumber \delta: K_1(kG) &\rightarrow& HH_1(kG) \cong \bigoplus_{g\in G}G\\
\nonumber             g              &\rightarrow& (\text{det }g)^{-1}\mathfrak d (\text{det }g) = \lambda(i+h)\mathfrak d (i-h+h^2-h^3+h^4)\\
\nonumber                              &\cong& \lambda(i,h^{-1},h^2,h^{-3},h^4).(h^4,i,h^{-1},h^2,h^{-3})\\
\nonumber			   &\cong& \lambda (h^4,h^4,h,h^4,h)\\
\nonumber 			   &\cong& (h^2,h^2,h^3,h^2,h^3) \in \bigoplus_{g\in G} G
\end{eqnarray}

In the way the determinant function appears in the commutative case, the Dieudonne Determinant, with image consisting of cosets arises from the D.T.M. in the non-commutative case, so we are interested in elements occuring as determinants which are units in the group ring.

In general, identifying the units in a group ring is difficult, and only isolated theories exist. The simplest being:
\subsubsection{Proposition (Splitting of Group Rings):}

\emph{The units in the group ring, $(\Z G )^*$ splits up as a direct sum with the group $G$ itself as one of the summands, using the following maps:
\begin{eqnarray}
\nonumber  {( \Z G)}^{*}&\cong&G\oplus\dots\\
\nonumber        1.g  &\leftarrow& g\\
\nonumber   \sum n_g g &\rightarrow& \prod g^{n_g}
\end{eqnarray}}

