Abelianisation and taking of inverse limits does not commute, hence considering ring $\Lambda_G = \varprojlim_{U\leq G} \ZP [ G/U]$, $K_1(\Lambda_G) = K_1{(\varprojlim \ZP [ G/U])}\neq \varprojlim K_1(\ZP [ G/U])$, but the universal defining property of $\varprojlim$ gives a map,

$$K_1(\Lambda_G) = K_1{(\varprojlim \ZP [ G/U])}\rightarrow \varprojlim K_1(\ZP [ G/U])$$

It was shown in \ref{ET} that 

$$HH_1(\Lambda_G) = HH_1{(\varprojlim \ZP [ G/U])} = \varprojlim HH_1(\ZP [ G/U])$$

Moreover from \ref{reduce}, there exists a map $\prod_{\gamma , \text{conj. class}} (Z(\gamma))^{ab}\rightarrow  \varprojlim HH_1(\ZP [ G/U]) = HH_1(\Lambda_G)$.

I now use these maps to build up a generalised trace map on $K_1$ of  completed group algebras.

From \ref{DTM} there exists a map $\delta_U : K_1(\ZP [ G/U])\rightarrow  HH_1(\ZP [ G/U])$ for these to induce a map

$$ \delta : \varprojlim  K_1(\ZP [ G/U])\rightarrow  \varprojlim  HH_1(\ZP [ G/U])$$

We need to show the following square commutes for $U\leq V$ giving projections $G/ U \xrightarrow{\pi} (G/U) / (V/U) = G/V$:

$$\begin{array}{ccc}
   K_1(\ZP [ G/U])&  \xrightarrow{\delta_U}& HH_1(\ZP [ G/U])  \\
  \downarrow \pi&   &   \downarrow \pi\\
   K_1(\ZP [ G/V])&  \xrightarrow{\delta_V} &HH_1(\ZP [ G/V])   
\end{array}
$$

Equivalently,  without loss of generality it is enough to show this holds when passing to a quotient group for finite $G$, $G\xrightarrow{\pi}G/U:g\rightarrow \overline g$:

$$\begin{array}{ccc}
   K_1(\ZP [ G])&  \xrightarrow{\delta}& HH_1(\ZP [ G])  \\
  \downarrow \pi&   &   \downarrow \pi\\
   K_1(\ZP [ G/U])&  \xrightarrow{\delta_U} &HH_1(\ZP [ G/U])   
\end{array}
$$

For $M\in M_n(\ZP[G])$, 
\begin{eqnarray}
\nonumber \pi \circ \delta (M) &=& \pi (Tr(M^{-1}\mathfrak d  M))\\
\nonumber                                &=& Tr({\overline M}^{-1}\mathfrak d  \overline M) \text{ since Projection commutes with Trace and Inverses}\\
\nonumber                                &=& \delta_U\circ \pi (M)
\end{eqnarray} 

Thus the maps $\delta_U$ are consistent and we may define their inverse limit:

$$ \delta = \varprojlim \delta_U : \varprojlim  K_1(\ZP [ G/U])\rightarrow  \varprojlim  HH_1(\ZP [ G/U])$$

This gives the definition of a Dennis Trace Map, $ \Delta: K_1(\Lambda_G)\rightarrow  \varprojlim  HH_1(\Lambda_G)$ given by the composition:
 
$$\Delta: K_1(\Lambda_G)\rightarrow \varprojlim  K_1(\ZP [ G/U])\rightarrow  \varprojlim  HH_1(\ZP [ G/U]) = HH_1(\Lambda_G)$$

Thus, the amount of useful information this Dennis Trace Map can carry is limited by how much is lost in passing from $K_1(\Lambda_G)$ to $\varprojlim  K_1(\ZP [ G/U])$.

We hope to define a characteristic element in this image - $ \varprojlim  HH_1(\ZP [ G/U])$, the inverse limits of direct sums. However, the following commutative diagram shows we are infact just dealing with a direct product:

\subsubsection*{Claim (DTM Commutes with Direct Products):}
\emph{For $\Lambda_G$ semi-local the following diagram commutes, considering appropriate sums formally. Let element of $K_1(\Lambda_G)$ be represented by $x\in (\Lambda_G)^*$.
$$\begin{array}{ccc}
x\in \Lambda_G^*	& \rightarrow 	& x^{-1}\mathfrak d x\\
\downarrow      		&			& \downarrow \Theta \\
\varprojlim \overline {x_U}\in \varprojlim K_1([G/U]) & & \prod y \in \prod_{\gamma , \text{conj. class in }G} (Z_{(G)}(\gamma))^{ab}  \\
\downarrow \delta_U	&			& \downarrow\\
\overline{x_U}^{-1}\mathfrak d \overline{x_U} & \xrightarrow {\Theta} & \varprojlim \bigoplus \overline y \in \varprojlim \bigoplus_{\gamma , \text{conj. class in }G/U} (Z_{(G/U)}(\gamma))^{ab} = HH_1(\Lambda_G)
\end{array}$$}

Thus to calculate image under Dennis Trace Map in the Semi-Local case it is enough to understand the image in the formal object $\prod_{\gamma , \text{conj. class in }G} (Z_{(G)}(\gamma))^{ab}$. 

For this Trace Map to be useful we need it to be well defined on characteristic elements - elements of $K_1(\Lambda_G)\hookrightarrow K_1((\Lambda_G)_T)$ must vanish. I investigate this in \ref{12.4}, and show that infact the Hochschild homology id too corse, for the number theoretical situations we are interested in: localisation kills the whole group and not just elements induced from un-localised $HH_1(R)\rightarrow HH_1(R_S)$.


%$$\begin{array}{ccccc}
%   K_1(\Lambda_G) = K_1( \varprojlim   \ZP [ G/U]) &\rightarrow & \varprojlim  K_1(\ZP [ G/U])   &\rightarrow& HH_1(\Lambda_G) = \varprojlim  HH_1(\ZP [ G/U])  \\
%  \uparrow&  && & \uparrow  \\
%K_1( \ZP [ G])   & \xrightarrow{M\rightarrow Tr(M^{-1}\otimes M)}& HH_1(\ZP [G]) & \xrightarrow{\Theta} & \prod_{\gamma , \text{conj. class}} (Z(\gamma))^{ab}  
%\end{array}$$



%%%%%%%%%%%%%%%%%%%%%%%%%%%%%%%%%%%%%%%%%%%%%%%%%%%%%%%%%%%%

