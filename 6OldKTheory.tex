%%%%%%%%%%%%%%%%%%%%%%%%%%%%%%%%%%%
$R$ is semilocal if it has a finite number of maximal ideals, and such that for $J$ the Jacobson radical, $R/J$ is a simple Artinian Ring (so certainly true when $R$ is local - $R/J$ field). By Wedderburn's Theorem, a semi-simple Artinian ring id a direct sum of matrix rings, so $R/J$ is a full matrix ring. We will see that in this case, $K_1(R)$ is "small". 

%%%%%%%%%%%%%%%%%%%%%%%%%%%%%%%%%%%


\subsection{Idempotents}
$e$ and $1-e$ always orthog for idempotent $e$.

Idempotent is primative if it does not split as sum of 2 idempotents.
'
\subsection{Blocks}
For a ring $A$,

$$A = B_1 \oplus \dots \oplus B_r$$
 decomposition into into 2 sided ideals - whenever A is NOETHERIAN.
 
 $B_i$s are the blocks of $A$, generated by central idempotent, decomposition gives:
 
 $$1 = e_1 + \dots e_r$$
 
 $B_i = e_1 A$ is itself a ring, with identity $e_i$
 
 \subsection{Grothendieck Groups}
 
 $$K_0 (A) = K_0 (\mathbb P (A))$$
 $$K_0 (A) = K_0 (\mathbb M (A)) $$

\subsection{Semisimple Rings}
\subsubsection{Lemma}
\emph{Let $A$ be a semisimple ring and let $V_1, \dots ,V_s$ be a complete list of representatives for the isomorphism classes of simple $A$-modules.}
\begin{itemize}
\item If $P,Q$ are finitely generated A-modules, then
$$P\cong Q \text{ if and only if } [P] = [Q] \text{ in } K_0(A)$$
\item $$K_0(A) = \oplus_{i=1}^s \Z [V_i]$$
\item $$b(A) = rk\, K_0(A)$$
\end{itemize}

\subsection{Semilocal Rings}
Notation $J(A)$ for jacobson radical, and $\overline A = A / J(A)$. A is
\begin{itemize}
\item semilocal if $\overline A$ is Artinian
\item local if $\overline A$ is simple Artinian
\item scalr local if $\overline A$ is a division ring
\end{itemize}

\emph{A is a complete semilocal ring if A is semilocal and complete wrt $J(A)$-adic filtration.}
\subsection{Idempotent Lifting}
For A a complete semilocal ring... with simple modules as above.

Since $\overline A$ is semisimple we can find primative orthogonal idempotents, $a_1, \dots a_s$ in $\overline A$ such that $V_i\cong a_i \overline A$ as an $\overline A$-module. 

***A is $J(A)$-adically complete so can lift $a_i$ to primative orthogonal idempotents of A***

$$a_i = \overline e_i$$



\subsection{Quillen's Construction of the Higher $K$-groups\label{Quillen}}

First, I give a general construction of $K$-groups which will allow me, in \ref{DTM} to simultaneously define a whole set of Dennis Trace Maps, $\delta:K_n(R)\rightarrow HH_n(R),\,\,n\geq 0$, and then I focus on the $n=1$ case, when $HH_n(kG)$ has an especially simple form as a direct sum of commutative groups, see \ref{FHH}. It is first necessary to recall the maps which form the background to Quillen's construction of the higher K-groups.

\begin{enumerate}
\item \textit{Hurewicz Map, $*$}

"$*$" is a map from the n-th Homotopy Group of a based space $(X,x_0)$ to it's n-th Homology Group. A loop $f\in \Pi(X,x_0)$ is a map from the $n$-sphere based at $s_0$ to the based space $f: (S^n,s_0)\rightarrow (X,x_0)$. Taking homologies of the spaces, $f$ induces a homomorphism $f_*:H_n(S,s_0)\rightarrow H_n(X,x_0)$. But $H_n(S,s_0) = \Z$, and so the homomorphism is completely determined by specifying the image of 1, where $f_*:1\rightarrow f_*(1) \in H_n(X,x_0)$ is associated to the loop $f$ giving the Hurewicz map:
\begin{eqnarray}
\nonumber "*": \Pi_n(X,x_0) &\rightarrow& H_n(X,x_0)\\
\nonumber          f &\rightarrow& f_*(1)
\end{eqnarray}

\item  \textit{Fusion Map, $\phi$}

For a ring, $R$, let $\phi':kGL_r(R)\rightarrow M_r(R)$ be componentwise evaluation of the formal sum. This induces $\phi'':HH_n(kGL_r(R)\rightarrow HH_n(M_r(R))$, and composing with the Generalised Trace Map (defined in \ref{GTM}) construct,
$$\phi = Tr\circ \phi'' :HH_n(GL_r(R))\rightarrow HH_n(R)$$

\item \textit{Plus Construction and Milnor's Base Space, $+$ and $B$}

The Base map $"B"$ is constructed to utilise geometry to solve algebraic question by associating to a given group a space,  and $"+"$ is a map between spaces, turning homology equivalences into homotopy equivalences, and preserving homology - $H_n(BG^+,k)\cong H_n(G,k)$. They are defined in such a way that for n=0,1,2 the groups $\Pi_n(B(GL(R))^+) = K_n(R)$, the classically defined K-groups. Defining the higher K-groups as $K_n(R) = \Pi_n(B(GL(R))^+),\, n\geq 0$, the long exact sequence of such "K-groups" which immediately arises from the homotopy suggests this is the correct definition. A full account is given in \cite{B}, section 2.10.
\end{enumerate}


%%%%%%%%%%%%%%%%%%%%%%%%%%%%%%%%%%%%%%%%%%%%%%%
%%%%%%%%%%%%%%%%%%%%%%%%%%%%%%%%%%%%%%%%%%%%%%%
%%%%%%%%%%%%%%%%%%%%%%%%%%%%%%%%%%%%%%%%%%%%%%%
%%%%%%%%%%%%%%%%%%%%%%%%%%%%%%%%%%%%%%%%%%%%%%%


\subsection{Dieudonne Determinant\label{DD}}

Let $A$ be a K-algebra, $a\in A$ let $\text{char.pol.}_{A/K}(a)$ be the characteristic polynomial of the $K$-linear map $x\rightarrow ax,\, x\in A$. Then trace, $T_{A/K}$ and norm maps, $N_{A/K}$ are defined as follows: 

$$\text{char.pol.}_{A/K} a = X^m - (T_{A/K} a)X^{m-1} + \dots + (-1)^m N_{A/K}a$$

However, if we take $A=M_n(K)$, so each $\mathbf a\in A$ is a matrix over $K$.

$$\text{char.pol.}_{A/K} \mathbf a = \{\text{char.pol.matrix}_{A/K} \mathbf a\}^n$$
giving $T_{A/K} \mathbf a = n. (\text{ trace of matrix } \mathbf a)$ and $N_{A/K} \mathbf a = (\text{det }\, \mathbf a)^n$, suggesting the trace and determinant are not fine enough for calculations. We introduced a "reduced" theory to overcome this.

Let $A$ be an arbitrary separable $K$-algebra and $E$ a splitting field for $A$ over $K$, so there is an isomorphism of $E$-algebras

$$E\otimes_K A\cong  \bigsqcup_{i=1}^s M_{n_i}(E)$$

For each $a\in A$, let
$$h(1\otimes a) = \bigsqcup \phi_i(a),\text{ where } \phi_i(a)\in M_{n_1}(E), \, 1\leq i\leq s$$

Take reduced characterisitc polynomial to be,
$$\text{red.char.pol.}_{A/K} a = \prod_{i=1}^s \text{char.pol.} \phi_i(a), \, \forall a\in A$$
This polynomial turns out to be independent of the choice of $E$, and moreover its coefficients lie in the ground field $K$. Taking reduced norm and trace, $nr_{A/K}$ and $tr_{A/K}$ as coefficients we overcome the problem discused above and still have nice multiplicative/ additive properties.

Suppose $A=M_n(D)$ with $D$ a skewfield with centre $K$. Write $K^\bullet = K - \{ 0 \}$. Then 
$$nr_{A/K}:GL_n(D) \rightarrow K^\bullet$$
is a multiplicative homomorphism which may be thought of as a "determinant map" in some sense.

Let $D^. = D-\{ 0\},\, D^{\#} = D^\bullet / [D^\bullet , D^\bullet ]$, then $nr_{D/K}:D^\bullet \rightarrow K^\bullet$ is a homo and induces
$$nr_{D/K} : D^{\#} \rightarrow K^\bullet$$

There is also a homomorphism, $det: GL_n(D)\rightarrow D^{\#}$, the \textbf{Dieudonne Determinant} such that,
$$nr_{D/K} \text{det } a = nr_{A/K} a \text{ for all } a \in GL_n(D)$$

Let $\mathbf E\in GL_n(D)$ be an elementary matrix. Then for $\mathbf X\in GL_n(D)$, $\mathbf{EX}$ is obtained from $\mathbf X$ by increasing some row of $\mathbf X$ by a left multiple of another row, and similarly for $\mathbf{XE}$ and columns.

Given any such $\mathbf X$ we can find products $\mathbf P$ and $\mathbf Q$ of elementary matrices such that,

$$\mathbf{PXQ} = \text{diag}(a_1, \dots, a_n),\, a_i\in D$$

By, \ref{elementary} $\text{diag}(a,a^{-1})$ is itself a product of elem. matrices we may improve matrices to give:
$$PXQ = \text{diag} (1, \dots, 1,a) \text{ where } a = \prod_1^n a_i \in D^\bullet$$

The image of $a$ in $D^{\#}$ is the Dieudonne Determinant of $\mathbf X$, det $\mathbf X$.

 If $D$ is a field then $D^{\#} = D^\bullet$, and det$\mathbf X$ is the usual determinant of $\mathbf X$. If $D$ is not a field then the element $a\in D^\bullet$ is not uniquely determined by $\mathbf X$, but it's image in $D^{\#}$ is. 
 
 \begin{itemize}
 \item Every elementary matrix has determinant $1$. 
\item $\left( \begin{array}{cc}
\mathbf X& *\\
0 & \mathbf Y\\ \end{array}\right) = (\text{det}\mathbf X)(\text{det}\mathbf Y) \text{ for } \mathbf X\in GL_n(D), \, \mathbf Y\in GL_m(D)$
\item $\text{det}(\mathbf{XY})) = (\text{det}\mathbf X)(\text{det}\mathbf Y)\text{ for } \mathbf{ X, Y} \in GL_n(D)$
\end{itemize}

I now work towards the key theorem connecting this determinant with the Whitehead Goups. Recall,

\subsection{Lemma (Grothendieck Groups of Semisimple Rings (see \cite{AW} 2.4))}
\emph{Let $A$ be a semisimple ring and let $V_1, \dots, V_s$ be a complete list of representatives for the isomorphism classes of simple $A$-modules.
\begin{enumerate}
\item If $P,Q$ are finitely generated $A$-modules, then
$$P\cong Q \text{ if and only if }|P| = |Q|\text{ in } K_0(A)$$
\item $K_0(A) = \bigoplus_{i=1}^s \Z [V_i]$ is free of rank $s$.
\item The number of blocks of $A$, giving a direct sum decomposition into $2$-sided ideals, $b(A) =$rk$K_0(A)$ 
\end{enumerate}}

For a left semisimple ring $A$, the number of homogeneous components $\{A_i\}$ of the left regular module $_AA$ is finite, and $A$ is their direct sum:
$A=A_1\bigoplus \dots \bigoplus A_i$
These $A_i$ are called the Wedderburn components of $A$, each is a 2-sided ideal in $A$, with $A_iA_{i'}=0\text{ if } i\neq i'$. Moreover, each $A_i$ is a simple Artinian ring. Each projective $A$-module $M\in \mathfrak P(A)$ decomposes into a direct sum $M=\bigoplus M_i$, where each $M_i\in \mathfrak P(A_i)$ is a projective $A_i$-module, and each $\mu\in \text{Aut }M$ has a corresponding decomposition. Then,

$$K_1(A) \cong \bigsqcup_i K_1(A_i)$$
and the problem is reduced to the case of simple Artinian rings.

Let $A$ be a simple Artinian Ring, and thus is Morita Equivalent to a division ring $D$, and as $K_1$ is unchanged under this equivalence,
$$K_1(D) \cong K_1(A)$$
Then $A= \text{End}_D V$ where V is a finite dimension vector space over $D$. Then the isomorphism between projective $D$-modules  and projective $A$-modules is given by mapping $M\in\mathfrak P (D)$ onto $V\otimes_D M \in \mathfrak P(A)$. The isomorphism follows from $\text{Aut}_D M \cong \text{Aut}_A(V\otimes_D M)$ by $[M,\mu ] \rightarrow [V \otimes_D M, 1\otimes \mu ] $ for $M\in \mathfrak P(D), \, \mu\in \text{Aut}_D M$ and interpreting $K_1$ as maps from a free space to itself.

We now calculate $K_1(D)$, where $D$ is a skewfield, making essential use of the Dieudonne Determinant. Let $D^\bullet = D - \{0\}$, the $D^{\#} = D^\bullet / [D^\bullet, D^\bullet]$ is an abelian multiplicative group.

\subsection{Theorem: ($K_1$ and Dieudonne Determinant (\cite{CR} 38.32))}
\emph{The Dieudonne determinant gives an isomorphism of groups 
$$K_1(D) \cong D^{\#}$$
for every skewfield $D$.}

Moreover, the proof will show that infact every element of $K_1(D)$ has the form $[ D , d_r ]$ for some $d\in D^\bullet$, where $d_r$ represents right multiplication by $d$. Also, when $R$ is a local ring (not necessarily commutative) Klingenberg has shown there is a well defined isomorphism $\text{det}:K_1(R)\cong (R^*)^{ab}$.

\subsubsection*{Proof:}
For each ordered pair $(M, \mu)$, where $M\mathfrak \in P (D)$ and $\mu \in \text{Aut}_D M$. Each $M\in \mathfrak P(D)$ has a finite $D$-basis, relative to which $\mu$ may be represented by a matrix $\mathbf X \in GL_n(D)$, where $n=\text{dim}_D M $. Of course change of basis just replaces $\mathbf X$ by 
$\mathbf{TXT^{-1}}$ for some $\mathbf T \in GL_n (D)$. The Dieudonne determinant 
$$\text{det} \mathbf X = \text{det} \mathbf{TXT^{-1}}\in D^{\#}$$

Setting $\text{det }\mu = \text{det} \mathbf X \in D^{\#}$ this is well defined.

Consider an exact sequence of pairs (with morphisms coming from commuting square), with $L,M,N$ being $D$-spaces.
$$0\rightarrow (L, \lambda) \rightarrow (M, \mu) \rightarrow (N, \nu)\rightarrow 0$$
We may choose a $D$-basis of $M$ so that the matrix $\mathbf X$ representing $\mu$ has the form

$$\mathbf X = \left( \begin{array}{cc}
\mathbf X_1& *\\
0 & \mathbf X_2\\ \end{array}\right)$$
with $\mathbf X_1$ the matrix of $\lambda$, and $\mathbf X_2$ that of $\nu$. Clearly, $\text{det} \mathbf X = (\text{det} \mathbf X_1)(\text{det} \mathbf X_2)$, giving
$$\text{det} \mu = (\text{det} \lambda).(\text{det} \mu)$$
Moreover, $\text{det} \mu\mu' = (\text{det} \mu).(\text{det} \mu')$ for $\mu, \mu' \in \text{Aut }M$. This gives a well-defined homomorphism
$$\text{det}:K_1(D)\rightarrow D^{\#}$$
which maps $[M,\mu ]$ onto det$\mu$. 

Surjectivity is clear. For injectivity suppose det$\mu=1$, with $\mu$ represented by some matrix $\mathbf X$. The we can write down products $\mathbf{P,Q}$ of elementary matrices over $D$ such that
$$\mathbf{PXQ} = \text{diag} (1, \dots , 1 , d)$$ where $d\in [D^\bullet, D^\bullet]$. We now observe that every elementary matrx $\mathbf E$ represents the zero element of $K_1(D)$ as it may always be written in  the form
$$ \left( \begin{array}{cc}
\mathbf I & *\\
0 & \mathbf I\\ \end{array}\right) 
\text{ or }
\left( \begin{array}{cc}
\mathbf I& 0\\
* & \mathbf I\\ \end{array}\right)
$$
And we may think of it lying in the s.e.s.:
$$0\rightarrow (X_1, 1)\rightarrow (X_2, *)\rightarrow (X_3, *)\rightarrow 0$$
with $[X_1, 1] = 0 = [X_2, 1]$ in $K_1(D)$. Finally, $\text{diag}(1,\dots, 1,d)$ represents the zero element in $K_1(D)$ since $d\in [D^\bullet, D^\bullet$ and $K_1(D)$ is abelian. Hence, det$\mu=1$ ensures $[M,\mu] = 0$ in $K_1(D)$ hence map is injective.


%%%%%%%%%%%%%%%%%%%%%%%%%%%%%%%%%%%%%%%%%%%%%%%
%%%%%%%%%%%%%%%%%%%%%%%%%%%%%%%%%%%%%%%%%%%%%%%
%%%%%%%%%%%%%%%%%%%%%%%%%%%%%%%%%%%%%%%%%%%%%%%
%%%%%%%%%%%%%%%%%%%%%%%%%%%%%%%%%%%%%%%%%%%%%%%


\subsection{ $K_1 (R)$ for Non-Commutative Semilocal Rings\label{non-comm}}


When $R$ is a commutative ring, the existence of a simple determinant function and good control of the commutator gives a nice structure for $K_1(R)$, where the determinant map gives an obvious splitting

$$K_1(R) \cong SK_1(R) \text{ x } R^*$$
And since the commutators all occur as elementary matrices (\cite{CR}, 40.25) 
$$SK_1(R) \cong SL(R) / E(R)$$

However, there is a strong result due to Vaserstein which gives a good understanding of $K_1$ for $R$ semilocal: when $R/ J(R)$ is semisimple Artinian, where $J(R)$denotes the Jacobson Radical of R. We write $\overline R$ for the image, $R/J(R)$. Moreover, we say that $R$ is \emph{complete semilocal ring} if $R$ is semilocal and complete with respect to the $J(R)$-adic filtration.

We first recall some background Lemmas:

\subsection{Lemma: (Units in Semisimple Rings - \cite{CR} ex 40.1)\label{units in semisimple}}
\emph{Let $A$ be a \textbf semisimple ring, and suppose that $A = Ax +m$ for some left ideal $M$ of $A$ and some $x\in A$. Show that $x+m\in A^*$ for some $m\in M$.}



\subsection{Lemma: (Units in Ring  - \cite{CR} ex 5.2)\label{units in rings}}
\emph{An element $x$ of a ring $A$ is a \textbf unit if and only if the image of $x$ in $A/ rad \, A$ is a \textbf unit}

\subsection{Lemma: (\cite{CR} 40.8)}
\emph{For $\mathbf{X,Y}\in GL(\Lambda)$, we have
$$ \left( \begin{array}{cc}
\mathbf X& *\\
\mathbf 0 & \mathbf Y\\ \end{array}\right)=
 \left( \begin{array}{cc}
\mathbf X& \mathbf  0\\
* & \mathbf Y\\ \end{array}\right)= \mathbf{XY}\text{ in } K_1(\Lambda)$$}

\subsection{Lemma: (Elementary Matrices - \cite{CR} 40.22) \label{elementary}}
\emph{Let $E_n(R)$ be an elementary matrix in $GL_n(R)$ is it is obtained from the identity by changeing one off-diagonal entry. Let  $\mathbf E_{ij}(A) = \mathbf I + a \mathbf e_{ij}\in M_n(R) \text{  for } 1\leq i,j\leq n, i\neq j, a\in R$, where $\mathbf e_{ij}$ is the matrix unit with $1$ at position $(i,j)$ and zeroes elsewhere. Calculation gives $\{ \mathbf E_{ij}(a) \}$ satisfy the \textbf{Steinberg Relations} for $1\leq i,j,k,l \leq n$:}
$$\mathbf E_{ij}(a) \mathbf E_{ij}(b) = \mathbf E_{ij}(a+b)     \text{ if } i\neq j $$
$$ [\mathbf E_{ij}(a), \mathbf E_{kl}(b) ] = 1                                 \text{ if } i\neq j, k\neq l , j\neq k, i\neq l$$
$$ [\mathbf E_{ij}(a), \mathbf E_{jk}(b) ] = \mathbf E_{ik}(ab)  \text{ if } i,j,k \text{ distinct}$$

The third relation gives $[E_n(R), E_n(R)] = E_n(R)$  if $n\geq 3$, andso the direct limit, $E(R)$ is a perfect group (coincides with it's commutator subgroup). Whitehead's Lemma now states $E(R) = [GL(R), GL(R)]$ and is deduced from the following identity:

$$(*) \left( \begin{array}{cc}
\phi & 0\\
0 & \phi ^{-1}\\ \end{array}\right)
= 
\left( \begin{array}{cc}
1 & \phi^{-1}\\
0 & 1\\ \end{array}\right)
\left( \begin{array}{cc}
1 & 0\\
1 & 1\\ \end{array}\right)
\left( \begin{array}{cc}
1 & \phi^{-1} - 1\\
0 & 1\\ \end{array}\right)
\left( \begin{array}{cc}
1 & 0\\
-\phi & 1\\ \end{array}\right)$$
which holds for each $\phi\in GL_n(R)$, hence $[GL_n(R), GL_n(R)]\subset E_{2n}(R) \text{ for all } n\geq 1$.



\subsection{Theorem (Whitehead Group of Semilocal Rings (\cite{CR}, 40.31)\label{whitehead semilocal}}
\emph{Let $R$ be a semilocal ring. Then the natural map $R^* \rightarrow K_1(R)$ is surjective, that is, each element of $K_1(R)$ is of the form $(u)$ for some $u\in R^*$. Moreover, the kernel of this surjection is the subgroup of $R^*$ generated by all expressions $(1+xy)(1+yx)^{-1}$ in $R^*$ (for $x,y \in R$).}

\subsection*{Proof:}

Suppose that $R=L+Rb$ for some left ideal $L$ of $R$ and some $b\in R$. If bars denote reduction $\text{mod } J(R)$, then $\overline R$ is a semisimple ring, and $\overline R = \overline L + \overline{Rb}$. 

By \ref{units in semisimple} there exists an element $y\in \overline L$ with $y+\overline b \in \overline R^*$. Therefore $x+b\in R^*$ for some $x\in L$, by \ref{units in rings}. We shall use these facts below.

An $n$-tuple $\mathbf a =  (a_1, \dots , a_n)^t$ of elements of $R$ is called \textit{unimodular} if $\sum_{i=1}^n R a_i = R$. If $\textbf X \in GL_n (R)$, then from $\textbf X^{-1} \textbf X = \textbf I$ it follows that every column of $\textbf X$ is unimodular. If $\textbf x = (x_1, \dots , x_n)^t$ is the first column of $\textbf X$, and $\textbf E_{ij} (a)$ is as in \ref{elementary}, the first column of $\textbf E_{ij} (a) \textbf X$ is 

$$(x_1, \dots, x_{i-1}, x_i +ax_j, x_{i+1}, \dots , x_n)^t$$

We write $\textbf x \sim \textbf x'$ if the vector $\textbf x'$ can be obtained from $\textbf x$ by a finite number of such elementary (row) operations. If $n\geq 2$ and $x_1\in R^*$, it is clear from identity in \ref{elementary} that

$$(x_1, \dots , x_n)^t \sim  (1,0, \dots , 0)^t$$

We use these ideas to show that when $R$ is semilocal, every unimodular $\textbf x = ^t(x_1, \dots , x_n)$ is equivalent to $^t(1,0, \dots, 0)$ if $n\geq2$. By hypotheses, $Rx_1+\dots+Rx_n = R$; by the frst step in the proof we obtain 

$$a_1x_1 + \dots + a_{n-1} x_{n-1} +x_n \in R^*$$
for some elements $a_i\in R$. Then,

$$\textbf x \sim (x_1, \dots, x_{n-1}, a_1x_1+\dots+a_{n-1}x_{n-1}+x_n)^t \sim (1,0,\dots, 0)^t$$
as desired.

In matrix form, let $\textbf X\in GL_n(R)$ with $n\geq 2$. By the above there exists a product $\mathbf E$ of elementary matrices such that

$$\mathbf{EX} = \left( \begin{array}{cc}
1 & *\\
0 & \mathbf X_1 \\ \end{array}\right) \text{ for some }\mathbf X_1\in GL_{n-1}(R).$$

But then $\mathbf X$ and $\mathbf X_1$ represent the same element of $K_1(R)$, since from $(*)$, for $\mathbf {X,Y}\in GL(R)$, we have

$$\left( \begin{array}{cc}
\mathbf X & *\\
0 & \mathbf Y\\ \end{array}\right)
=
\left( \begin{array}{cc}
\mathbf Y& 0\\
* & \mathbf X\\ \end{array}\right)
=
\mathbf{XY} \text{ in } K_1(R)$$



Continuing this procedure with $\mathbf X_1$ (if $n\geq 3$), we eventually obtain an element $\mathbf X = (u)$ in $K_1(R)$ for some $u\in R^*$. This completes the proof.




If $R$ is semilocal, the surjection $R^* \rightarrow K_1(R)$ induces a surjection
$$R^*/ [R^* , R^*] \rightarrow K_1(R).$$

If $R$ is a skewfield, or more generally a local ring, this surjection is an isomorphism with inverse the Dieudonne determinant (rmks end \ref{DD}).
