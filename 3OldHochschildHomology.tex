%%%%%%%%%%%%%%%%%%%%%%%%%%%%%%%%%%%%
%%%%%%%%%%%%%%%%%%%%%%%%%%%%%%%%%%%%
%%%%%%%%%%%%%%%%%%%%%%%%%%%%%%%%%%%%
%%%%%%%%%%%%%%%%%%%%%%%%%%%%%%%%%%%%
%%%%%%%%%
$\,\,\,\,\,\,\,\,\,\,$I begin this section by constructing a map from the Grothendieck Group of a ring, $R$ to the abelianisation of R, $Tr: (K_0(R)\rightarrow R/[R,R])$ which gives the motivation, and language for the Dennis Trace map for higher order K-groups.

I give the definition of Hochschild Homology in \ref{HH}, this is the homology of Bimodules written $HH_n(R,M)$, and specialise to the case $n=1$ before giving some explicit examples.

I then introduce the Generalised Trace Map, $tr$, and explain how this gives the Morita Equivalence of the Hochschild Homology.

In the next section, \ref {DS}, I give a decomposition of Hochschild Homology of a group algebra as a direct sum of more familiar objects - Group Homology of centralisers in the group, where the sum is taken over conjugacy class  representatives.

Section \ref{EQ} considers how the interpretation of $HH_1(kG)$ as both a direct sum, and as Kahler Differentials are equivalent.

After giving the Homotopy definition for K-Groups, $K_i(R),\, i\geq 0$, I am then in a position to define the Dennis Trace Map, $\delta: (K_i(R)\rightarrow HH_i(R)) \,\forall i\geq 0$, and give explicit matrix examples.

\subsection{Grothendieck Groups\label{GG}}
%%%%%%%%%%%%%%%%%%%%%%%%%%%%%%%%%%%
$\,\,\,\,\,\,\,\,\,\,$In this report we are mainly interested in the Dennis Trace Map from the first K-group, $\delta: (K_1(R)\rightarrow HH_1(R))$. This section defines the Tace Map, $Tr: (K_0(R)\rightarrow R/[R,R])$, induced from the standard trace map on square matrices. We will see later that $R/[R,R]$ is just the zeroth Hochschild Homology Group of $R$, and indeed $Tr = \delta_0: (K_0(R)\rightarrow HH_0(R))$.

Firstly, recall relevant definitions:

\subsubsection{Definition (Grothendieck Group):}
\emph{The Grothendieck Group (G.G.) of a semigroup $S$ is an abelian group given in terms of generators and relations:\begin{itemize}
\item Generators $\{ [x] : x\in S\}$
\item Relations $x+y = z$ in $S\iff (\star)\,\,\,  [x] +[y] = [z]$ in G.G.
\end{itemize}}

This operation is easily seen to be functorial, with the maps between semi-groups giving rise to associated maps between Grothendieck Groups, and this makes it an interesting object to study. 

\subsubsection{Definition ($K_0(R)$):}
\emph{Let $R$ be a ring with unit, the zeroth K-group is defined to be:
$$K_0(R) = \text{G.G. of semigroup $Proj\,R$}$$
Where $\text{Proj }R$ is the semigroup of Isomorphism classes of finitely generated projective modules over $R$. In this case $(\star)$ splits to simplify the relations: $[x]+[y] = [x\oplus y]$}.
\bigskip

The most natural setting for understanding the Trace Map comes from a modification of this into idempotents.

Given a ring R, $P\in \text{Proj }R$ we may associate an element of $GL(R)$ to $P$. Since $P$ is a direct summand of the free group, $P\oplus Q = R^n$, say. Consider the map given on components,
$$\theta: (P\oplus Q\rightarrow P\oplus Q) = (id\vert_P\oplus0\vert_Q)$$
Then $\theta:(R^n\rightarrow R^n)$ is an idempotent, $\theta^2 = \theta$, and may be considered an element of $M_n(R)\subset M(R)$

It is easily seen that different idempotents give rise to the same projective modules. A natural question to ask, is to what extent does the module uniquely determine the idempotent $\theta$? The answer is given as an equivalence relation on idempotents ensuring Isomorphism of the corresponding modules.

\subsubsection{Proposition (Idempotent Equivalnce)}
\emph{For any ring $R$, $\text{Proj }R$ may be identified with the set of conjugacy orbits of $GL(R)$ on $Idem(R)$. Where the Semigroup operation is induced by $(p,q)\rightarrow \left( {\begin{array}{cc}
               p& 0\\
               0& q\\\end{array}}\right)$.}
               
       \bigskip

Since $K_0(R)$ is the G.G. of this semigroup we may consider mapping from class of idempotents (defined up to conjugation). The "trace map" is now given by a choice of matrix representing P, and taking it's trace (which is well defined):

\begin{eqnarray}
\nonumber Tr &:&K_0(R) \rightarrow R/[R,R]\\
\nonumber      &:& [P] \rightarrow [\text{Tr }P]\text{  class of idempotent matrix}
\end{eqnarray}
%%%%%%%%%%%%%%%%%%%%%%%%%%%%%%%%%%%
\subsection{Hochschild Homology of a Ring\label{HH}}
%%%%%%%%%%%%%%%%%%%%%%%%%%%%%%%%%%%
For $k$ a commutative ring, and $R$ a $k$-algebra (with unit), I consider an $R-R$ bimodule M (where $(r_1 m)r_2 = r_1(mr_2)$), and introduce an homology where both left and right actions are considered - the Hochschild Homology of M over R.

\subsubsection{Definition (Hochschild Homology as $Tor$-group):}
 \emph{The n-th Hochschild Homology of R with coefficients in the bimodule M is $HH_n(R,M) = Tor_n^{R\otimes R^{\text{op}}}(R,M)$.}
 
 \bigskip
 
 There is an explicit complex which may be used for computation:
 
 \subsection{Proposition (HH as Homology of a given Complex)\label{complex}:}
 \emph{Hochschild Homology is the homology of Hochschild Chain Complex, $\{C_*(R,M),d\}$ consisting of $$C_n(R,M) = k^{\otimes� n}\otimes_k M$$ and boundary maps $$d(r_1\otimes\dots\otimes r_n\otimes m) = r_2\otimes\dots\otimes r_n\otimes mr_1 +\sum_{i=1}^{n-1}{(-1)}^i r_1\otimes\dots\otimes r_i r_{i+1}\otimes \dots\otimes r_n\otimes m + {(-1)}^n r_1\otimes \dots \otimes r_{n-1}\otimes r_n m$$}
 
 We will be mainly concerned with $HH_0(R,M)$ and  $HH_1(R,M)$ which are computed from,
 \begin{eqnarray}
 \nonumber  \rightarrow R\otimes R\otimes M  &\xrightarrow{d}& R\otimes M\\
 \nonumber  r_1\otimes r_2\otimes m &\rightarrow& r_2\otimes mr_1 - r_1r_2\otimes m +r_1\otimes r_2 m
 \end{eqnarray}
  \begin{eqnarray}
 \nonumber  R\otimes M &\xrightarrow{d}& M\\
 \nonumber r\otimes m &\rightarrow& mr-rm
 \end{eqnarray}
  
  \subsubsection{Notation (Ring acting on Itself):}
  \emph{If $M=R$ with $R-R$ bimodule structure coming from left/right multiplication of elements in ring, we write $HH_n(R)$ for $HH_n(R,M)$.}
  
  \bigskip
  
  I now calculate the first few terms of the Hochschild homology sequence of a ring explicitly.
  
  \subsubsection{ Corollary (Calculation of Terms of Low Degree)\label{kahler}:}
  \emph{For a k-algebra R,
  \begin{itemize}
  \item $HH_0(R) = R/[R,R]$
  \item For $R$ commutative, $HH_1(R) \cong \Omega^1(R)$, the space of Kahler differentials
  \end{itemize}}
  
  \subsubsection*{Proof}
From the chain complex above, $HH_0(R) = R/d(R^{\otimes 2})$, and $d(r_1\otimes r_2) = r_1r_2 - r_2r_1 = [r_1,r_2]$. Hence,
$$HH_0(R) = R/[R,R] \,(= R\text{ if $R$ is commutative})$$

For $R$ commutative, $d(R^{\otimes 2}) = 0$, hence
\begin{eqnarray}
\nonumber HH_1(R) &=& R\otimes_k R / \text{Im} \,d(R^{\otimes3})\\
\nonumber                  &=& R\otimes_k R / < \{r_1r_2\otimes r_3 - r_1\otimes r_2 r_3 + r_3 r_1 \otimes r_2 \}> 
\end{eqnarray}

Writing, $a_1 \mathfrak d a_2$ for the image of $a_1 \otimes a_2$ in this quotient, then $\mathfrak d:a_2\rightarrow \mathfrak d a_2$ is a $k$-linear derivation.

I now give an example to show how this complex may be used for calculations:

\subsubsection{Example (Polynomial Ring)\label{laurent}:}
Consider the polynomial ring in one variable over $k$, $R = k[t]$ (the same arguments extend to case of Laurent Polynomials).
$R$ is free over $k$, taking as basis the set of monomials $\{t^i: \, i\geq 0\}$, hence is k-Projective. Since R is commutative, $R=R^{\text{op}}$, and $R\otimes R^{\text{op}}\cong k[t,s]$. Thus if we consider setting $t$ and $s$ to be equal, as a $k[t,s]$-module, $R\cong k[t,s]/(t-s)$.

Hence, $k[t,s]\xrightarrow{(t-s)} k[t,s] \twoheadrightarrow R$ is exact, and thus is a $R\otimes R^{\text{op}}$ resolution of $R$, giving:
\begin{eqnarray}
\nonumber HH_1(R)\cong HH_0(R) \cong R\\
\nonumber HH_i(R) = 0 \,\,\,\,\, \forall\,\, i>1
\end{eqnarray}
%%%%%%%%%%%%%%%%%%%%%%%%%%%%%%%%%%%
\subsection{Generalised Trace Map\label{GTM}}
%%%%%%%%%%%%%%%%%%%%%%%%%%%%%%%%%%%
This section extends the trace map from a map on matrices to the sum of diagonal entries to a map from tensor products of matrix spaces, and will be a key tool in the definition of the Dennis Trace Map.

\subsubsection{Definition (Generalised Trace Map):}
\emph{Let $N_1,\dots , N_n$ be R bi-modules. Then for all r, define generalised trace map,
\begin{eqnarray}
\nonumber &tr:& M_r(N_1)\otimes\dots \otimes M_r(N_n) \rightarrow N_1\otimes \dots \otimes N_n\\
\nonumber &&\text{as the unique k-linear extension of the map defined on elementary matrices,}\\
\nonumber &tr:& E_{i_0j_0}(a_0)\otimes E_{i_1j_1}(a_1)\otimes \dots \otimes E_{i_nj_n}(a_n)\rightarrow \Big( {\begin{array}{cc}
                a_0\otimes a_1\otimes \dots \otimes a_m, & \text{ if }j_0 = i_1, j_1=i_2, \dots ,j_m = i_0\\
                0, &\text{ otherwise}\\\end{array}}
                \end{eqnarray}}
                
\bigskip

Notice that for $n=1$, $tr$ recovers the usual trace map as a sum of diagonal elements ($j_0=i_0$). We now see how this leads to the Hochschild Homology being unchanged when we pass from working over a ring $R$ to the new ring of $r\,\text x \, r$ matrices over $R$:

\subsubsection{Proposition (Morita Equivalence):\label{Morita}}
\emph{The Trace map (rewritten as a sum over all $n+1$-tuples) acting on elements of Hochschild Complex:
\begin{eqnarray}
\nonumber Tr: M_r(R)^{\otimes n} \otimes M_r(M) &\rightarrow& R^{\otimes n} \otimes M\\
\nonumber : \alpha \otimes \beta \otimes \dots\otimes \eta &\rightarrow & \sum_{(i_0, i_1, \dots , i_n)}
\alpha_{i_0 i_1}\otimes \beta_{i_1 i_2} \otimes \dots\otimes \eta_{i_n i_0}\\
\nonumber \text{induces a Natural Isomorphism,}\\
\nonumber HH_n(M_n(R)) &\rightarrow& HH_n (R)
\end{eqnarray}
Hence Hochschild Homology is Morita Invariant.}
\bigskip

Since the functor $K_1$ is Morita Invariant (again proved using G.T.M.), for a map of functors $K_1(R)\rightarrow F(R)$ to be interesting, $F(R)$ must also be invariant. \ref{Morita} gives hope that such a map may exist between $K_n(R)$ and $HH_n(R)$



%%%%%%%%%%%%%%%%%%%%%%%%%%%%%%%%%%%
\subsection{Hochschild Homology of Group Ring as a Direct Sum over Conjugacy Classes\label{DS}}
%%%%%%%%%%%%%%%%%%%%%%%%%%%%%%%%%%%
The aim of this section is to relate Hochschild Homology of a group ring back to a set of Group Homologies corresponding to centralisers of elements in the group. The key result is:

\subsubsection{Propositon (Direct Sum Decomposition):\label {DSD}}
\emph{For G a finite group the Hochschild homology of the group algebra breaks up as the direct sum over conjugacy classes of the group homology of centrailsers of representatives of the conjugacy classes with trivial coefficients:
$$ HH_*(kG) = \bigoplus_{g_c\in C_G} H_*(Z(g_c))$$}
\bigskip

Proof of this in \cite{B} is done using Shapiro's Lemma and the Mackey Decomposition of a space into a sum over double cosets. I give a proof which isolates in \ref{lemma18}, and \ref{lemma19} what Benson needs from the above results.

\subsubsection{Lemma (Centralisers and Conjugacy Classes):\label{lemma18}}
\emph{Choosing representatives of conjugacy classes, $g_c\in C$, there exists an isomorphism of sets,
$$C\cong G/ Z(g_c)$$ where centraliser of $g_c$, $Z(g_c) = \{ h\in G \vert h = g h g^{-1}\}$}
\bigskip

This follows from the First Isomorphism Theorem, applied to,
\begin{eqnarray}
\nonumber G &\twoheadrightarrow& C\\
\nonumber g &\rightarrow& g g_c g^{-1}, \text{ whose kernel } \cong Z(g_c)
\end{eqnarray}

\subsubsection{Lemma (Induction of Centralisers):\label{lemma19}}
\emph{$$H_*(G,k(G/Z(g_c))) \cong H_* (Z(g_c), k)$$}
%\bigskip

Shapiro's Lemma for $H\leq G$, and an $H$-module $M$ follows by explicitly manipulating resolutions and gives a relation between homology of $H$ with coefficients in $M$, and the homology of $G$ with coefficients in the induced module $M \uparrow_H^G = M\otimes_{kH} kG$, and states 
$$H_*(H,M) = H_*(G, M \uparrow_H^G )$$
The Lemma \ref{lemma19} follows immediately from Shapiro noting
\begin{itemize}
\item $ Z(g_c) \leq G$
\item $k(G/ Z(g_C)) \cong k \otimes_{k Z(g_c)} kG$
\end{itemize}
\bigskip
Using these Lemmas I now prove \ref{DSD} by showing how Hochschild complex splits up over conjugacy classes.

\subsubsection*{Proof of \ref{DSD}}
Partition $G$ into the union of conjugacy classes, $k G = \bigoplus_{C\in C_G} k C$. In Hochschild Chain Complex, each generating chain $c=g_1\otimes \dots \otimes g_n \otimes m$ can be thought of (canonically) as $g_1\otimes \dots \otimes g_n \otimes g_n^{-1}\dots g_1^{-1} g$ where the product of elements, $g = g_1 \dots g_n m$ is the "marker" of generating chain.

\subsubsection*{Claim (boundary map preserves markers):}
\emph{All generating chains occuring in the sum giving boundary, $d(c)$ have "markers" in $C(g)$ (in the same conjugacy class).}

\bigskip
This can be seen by considering each summand of the boundary in turn, for example, the first may be thought of as $g_2\otimes \dots \otimes g_n\otimes g_n^{-1}\dots g_2^{-1}(g_1^{-1} g g_1)$ with "marker" $g_1^{-1} g g_1 \in C(g)$.

\bigskip

Thus, we can separate the chains with markers in a particular conjugacy class. For $C\in C_G$, a particular conjugacy class, denote by $C_*(kG, kG)_C$ the subset of $C_*(kG,kG)$ generated by chains with markers in $C$. This gives the decomposition: $C_*(kG,kG) \cong \bigoplus_{C\in C_G} C_*(kG,kG)_C$ inducing the sum over homology of $C$-components:

$$HH_*(kG,kG) \cong \bigoplus_{C\in C_G} HH_*(kG,kG)_C$$

To complete the proof we understand how Hochschild homology, may be reduced to group homology where group action on a left module incorporates both left/right action of bimodule.

For $N$ a $kG-kG$ bimodule, let $\overline N$ be the left $kG$-module with underlying abelian group N, and left action given by $g\circ m = (g \circ_l m )\circ_r g^{-1}$. Then isomorphims at resolution level - between \ref{complex}, and homogeneous bar complex (used to calculate group homology, see \cite{W}) or Shapiro's Lemma \ref{lemma19} give,

$$ HH_*(kG,N) = H_*(G,\overline N)$$

As a direct sum of left modules, $kG = \bigoplus_{C\in C_G} k C$, hence restricting action to conjugacy classes: 

$H_*(G, \overline{kG}) = \bigoplus_{C\in C_G} H_* (G, kC)$. \ref{lemma18}, and \ref{lemma19} give Hochschild Homology of Group Algebra in terms of conjugacy class representatoves $\{g_C\}$
\begin{eqnarray}
\nonumber HH_*(kG,kG) 	&=& \bigoplus_{C\in C_G} HH_* (kG,kG)_C\\
\nonumber                          	&=& \bigoplus_{C\in C_G} H_*(G, kG)\\
\nonumber   			&=& \bigoplus_{\{g_C\}} H_*(G, k(G/ Z(g_C)))\\
\nonumber   			&=& \bigoplus_{\{g_C\}} H_*(Z(g_C))
\end{eqnarray}
where \underline{$HH_*(kG,kG)_C$ corresponds to $H_*(Z(g_C))$}. This completes the proof of \ref{DSD}.

\bigskip

These ideas can be extended using the same proof to the case of semiconjugacy - for a group homomorphism $\phi$, define an equivalence relation $g_1 \sim g_2$ when $g_1 = g g_2 \phi (g^{-1})$ for some $g\in G$. Then the decomposition theory of "markers" passes unchanged, and denoting the set of "semiconjugacy classes" by $G_{C_\phi}$,

$$ HH_*(kG)= \bigoplus_{g_c\in C_{G_\phi}} H_*(k(g_c))_\phi$$

\subsubsection{Corollary (Zeroth Hochschild Homology):}
\emph{$HH_0(kG,kG) \cong kG$ the free abelain group generated by conjugacy classes.}

\subsubsection{Corollary (First Hochschild Homology)\label{FHH}:}
\emph{Hochschild Homology splits as a sum of abelianisations:
$$HH_1(kG,kG) \cong \bigoplus_{C\in C_G} H_1 (Z(g_C)) \cong \bigoplus_{C\in C_G} (Z(g_C))^{\text{ab}}$$}

These corollories explain the general strategy of the Introduction to get information on first K-groups, since abelianisations are easy to calculate, the image of the Dennis Trace Map is well understood. Thus, if we understand the map itself we can get information on the domain space, $K_1(kG)$.









%%%%%%%%%%%%%%%%%%%%%%%%%%%%%%%%%%%
\subsection{Equivalence of Definitions of First Hochschild Homology of a Group Ring, $HH_1(kG)$\label{EQ}}
%%%%%%%%%%%%%%%%%%%%%%%%%%%%%%%%%%%
We have 2 different interpretations of the first Hochschild Homology of a group ring, as Kahler differentials (\ref{kahler}), and also lying in a direct sum (\ref{DSD}). I give an isomorphism $\Theta$ connecting the approaches. For ease of notation I consider the case of G commutative, when each element forms it's own conjugacy class, whose centraliser is the whole group $G=G^{\text{ab}}$.

$$\begin{array}{ccccc}
       		&&HH_1(kG)=<\{g\otimes h\}>/{(\sim)} &&\\
              	&\swarrow&&\searrow&\\
               	<\{g\mathfrak d h\}>&&\xrightarrow{\Theta}&& \bigoplus_{g\in G} G\\\end{array}$$
	
The concept of "markers" introduced in the proof of \ref{DSD} gives which conjugacy class each term lies in, and thinking of $g\mathfrak d h$ as $g \mathfrak d g^{-1} (gh)$, the mapping into direct sum is (unique) linear extension to group algebras of 
$$g \mathfrak d g^{-1} (gh)\xrightarrow{\Theta} (1,\dots, 1, g, 1,\dots,1),$$
with the identity in each component except for $g$ lying in $gh$-position

The derivative property of $\mathfrak d$ follows immediately from the following commutative diagram, since both $g\mathfrak d hk$ and $h\mathfrak d gk$ have the same marker, $ghk$.

$$\begin{array}{ccccc}
       		&(\dots,1,gh,1,\dots)  &=&(\dots,1,g,1,\dots).(\dots,1,h,1,\dots)&\\
              	&\downarrow\Theta&\circlearrowright&\downarrow\Theta&\\
               	&gh\mathfrak d k & = & g\mathfrak d hk +h\mathfrak d kg & \\\end{array}$$
	
I complete this section by giving another, perhaps more familiar interpretation of Kahler Differentials from that given in \ref{kahler}.

For a ring $R$, let I be the kernel of the augmentation map $R\otimes R\rightarrow R$, induced from $a\otimes b\rightarrow ab$. Then there is an \underline{isomorphism from Kahler differentials to the quotient of ideals $I/I^2$}:

\begin{eqnarray}
\nonumber <\{a\mathfrak d b\}>/(\sim) &\cong& I/I^2\\
\nonumber  a\mathfrak d b &\xrightarrow{\psi}& ab\otimes 1 - a\otimes b
\end{eqnarray}

This is clear except for it being a map on the quotient, or equivalently that $\mathfrak d$ is a derivation:
\subsubsection*{Claim ($\psi$ respect derivation):}
\begin{eqnarray}
\nonumber \psi((a\mathfrak d bc) - (ab\mathfrak d c + ac\mathfrak d b)) 	&\equiv& (abc\otimes 1 - a\otimes bc)-(abc\otimes 1 +ab\otimes c +abc\otimes 1-ac\otimes b)\\
\nonumber 										&\equiv& (abc\otimes 1-a\otimes bc + ab\otimes c - ac\otimes b)\\
\nonumber										&\equiv& (a\otimes 1)(b\otimes 1 -1\otimes b)(c\otimes 1-1\otimes c)\in I^2\\
\nonumber										&\equiv& 0
\end{eqnarray}

Hence, $\psi(a\mathfrak d bc) \equiv \psi(ab\mathfrak d c + ac\mathfrak d b)$, as required.			

%%%%%%%%%%%%%%%%%%%%%%%%%%%%%%%%%%%