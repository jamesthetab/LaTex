
\section{Introduction}

In this section we define a cohomology for profinite groups, $G$,  which recovers the usual group cohomology when $G$ is finite and see how this profinite cohomology  is built up from finite pieces in \ref{SHG}.

I then consider completing the ground ring, and then define a cohomology  - this is achieved using the completed tensor product in \ref{CTP}, and i relate back to the cohomology of profinite groups in a surprisingly simple way in \ref{HCGR}.

In \ref{TEG} I give a duality result for $"Tor"$ and$"Ext"$ groups over completed rings, and together with the usual application of Pontryagin duality the theories of homology and cohomology are dual, and I use them interchangeably.

These techniques give the ground work for \ref{CHH} where I demonstrate the Hochschild Homology of the completed (Iwasawa) algebras as a completion of the Hochschild homologies of the finite group ring.




%%%%%%%%%%%%%%%%%%%%%%%%%%%%%%
%%%%%%%%%%%%%%%%%%%%%%%%%%%%%%

\section{Definition of Cohomology}

I outline key constructions without proof - see (\cite{W}, 6.11) or (\cite{Se}) for details.

For G profinite, an appropriate notion of module is that of "discrete $G$-module. A discrete $G$-module is a $G$-module $A$ such that when $A$ is given the discrete topology, the action of $G$ on $A$: the multiplication map $G\,\text{x}\, A\rightarrow A$ is continuous.

The cohomology groups $H^*(G,A)$ of a profinite group $G$ with coefficients in a discrete $G$-module $A$ are the right derived functors of the functor $\mathfrak D\rightarrow \text{Ab}$ sending $A$ to $A^G$, applied to $A$.

Thus $H^0(G,A) = A^G$, and so when $G$ is a finite group, dimension shifting arguments give $H^*(G,A)$ agree with usual group cohomology.

Many properties are inherited - $H^*(G,A)$ is contravariant in $G$ via restriction maps, $H^*$ is functorial, and when the map $G\rightarrow G/H$ is continuous (when $H$ is a closed normal subgroup of $G$) we have inflation maps
$$H^*(G/H,A^H)\rightarrow H^*(G,A)$$

\subsection*{Cochains and Cocycles} 

I now give a resolution to calculate these groups.

If $A$ is a discrete $G$-module, let $C^n(G,A)$ denote the set of continuous maps from $G^n$ to $A$ (defining $C^0(G,A) = A$) - i.e. the maps $\phi:G^n\rightarrow A$ which are locally constant - each point in $G^n$ has a neighbourhood on which $\phi$ is constant.

Endowed with pointwise addition, $C^n(G,A)$ is an abelian group. Moreover, it is easily seen that $C^n(G,A) = \varinjlim_U  C^n(G/U , A^U)$ where $U$ runs through all open normal subgroups of $G$. This gives the key theorem:

\subsection*{Theorem (Definition of Cohomology - \cite{W}, 6.11.13):}
\emph{Let $G$ be a profinite group and $A$ a discrte $G$-module. Then,
\begin{eqnarray}
\nonumber    H^*(G,A)   &\cong& H^*(C^*(G,A))     \\
\nonumber                       &\cong& \varinjlim_U H^*(G/U, A^U)
\end{eqnarray}}

%%%%%%%%%%%%%%%%%%%%%%%%%%%%%%
%%%%%%%%%%%%%%%%%%%%%%%%%%%%%%


\section{Induction}

Recall, a $G$-module $A$ is called \textbf acyclic  if $H^n (G,A)
= 0$ for all $n>0$. A is called \textbf{cohomologically trivial/
welk/flasque} if $$H_n(H,A) = 0$$ for all closed subgroups, $H<G$,
and all $n>0$.

For our purposes the most important examples of cohomologically
trivial $G$-modules are those that are \textbf induced, given by
$$\text{Ind}_G(A) = \text{Map}(G,A)$$ where $A$ is any $G$-module.

The most natural definition is as continuous functions $x:
G\rightarrow A$, with the discrete topology on A, and where
$\sigma\in G$ acts via $(\sigma x) (\tau) = \sigma x (\sigma^{-1}
\tau)$. We will use the isomorphism:

$$\text{Ind}_G(A) \cong A\otimes \ZP  [G]$$ given by $x\rightarrow
\sum_{\sigma\in G} x(\sigma^{-1})\otimes \sigma$, where $\ZP [G]$
is the \textbf{group ring} of G.

The following proposition from \cite{N}, chapter 1 gives the
important properties of Induced modules required to handle
dimension shifting arguments in homological algebra:

\subsection{Proposition (Properties of Induction): \label{propind}}
\emph{\begin{enumerate}
\item The functor $A\rightarrow \text{Ind}_G(A)$ is exact ($\ZP [G]$
free, and hence projective as a $\Z$-module)
\item An induced $G$-module $A$ is also an induced $H$-module for
every closed subgroup $H$ of $G$, and if $H$ is normal, then $A^H$
is an induced $G/H$-module.
\item If one of the $G$-modules $A$ or $B$ is induced, then so are
$A\otimes B$ and $\text{Hom}(A,B)$, provided that in the case of
$Hom(A,B)$ when $A$ is induced, $G$ is finite.
\item If $U$ runs through the open normal subgroups of $G$, then
$$\text{Ind}_G(A) = \varinjlim_U \text{Ind}_{G/U}
(A^U)$$
\item The induced G-modules are cohomologically trivial.
\end{enumerate}}




%%%%%%%%%%%%%%%%%%%%%%%%%%%%%%
%%%%%%%%%%%%%%%%%%%%%%%%%%%%%%





\section{Structure of Homology Groups\label{SHG}}

The cohomology groups $H^n(G,A)$ of a profinite group $G$ with coefficients in a $G$-module $A$ are built up in a simple way from those of the finite factor groups of $G$. 

Let $U,V$ run through the open normal subgroups of $G$. If $V\subset U$, then the projections
$$G^{n+1} \rightarrow (G/V)^{n+1} \rightarrow (G/U)^{n+1}$$ induce homomorphisms,

$$C^n(G/U,A^U) \rightarrow C^n(G/V,A^V)\rightarrow C^n(G,A)$$
which commute with the operators $\delta^{n+1}$ and we therefore obtain homomorphisms,

$$H^n(G/U,A^U) \rightarrow H^n(G/V,A^V)\rightarrow H^n(G,A)$$

The groups $H^n(G/U,A^U)$ thus form a direct system and we have a canonical homomorphism \\ $\varinjlim_U H^n(G/U,A^U)\rightarrow H^n(G,A)$, moreover this may be shown to be an isomorphism:

\subsection{Proposition (Decomposition of Cohomology - \cite{N}, 1.2.6):\label{decomp}}
\emph{$$\mathbf{\varinjlim_U H^n(G/U,A^U)\cong H^n(G,A)}$$}

\subsection{Corollary (Decomposition of Homology - Dually):}

\emph{$$\varprojlim_U H_n(G/U,A_U)\cong H_n(G,A)$$}



\subsection{Change of the Group G\label{COGS}}

I investigate how this affects the cohomology groups $H^n(G,A)$. A general situation is for 2 profinite groups, $G$ and $G'$, a $G$-module $A$ and a $G'$-module $A'$, and 2 homomorphisms:
$$\phi:G'\rightarrow G,\,\, f:A\rightarrow A'$$
if such that the homomorphisms are "compatible": $\mathbf{f(\phi(\sigma')a) = \sigma' f (a)}$, then we obtain a map of chain complexes which commutes with boundary maps, giving a map of cohomologies:
$$H^n(G,A)\rightarrow H^n(G',A')$$

Moreover the map on cohomology groups $H^n(G,A)$ is functorial in both $G$ and $A$ simultaneously.

Let $(G_i)_{i\in I}$ be a projective system of profinite groups and let $(A_i)_{i\in I}$ be a direct system, where each $A_i$ is a $G_i$-module and the transition maps,

$$G_j\rightarrow G_i, A_i\rightarrow A_j$$
are compatible in the sense defined above. Then combining the induced homomorphisms $H^n(G_i, A_i)\rightarrow H^n (G_j, A_j)$ the cohomology groups $H^n(G_i, A_i)$ form a direct system of abelian groups. We may generalise \ref{decomp} to

\subsection{Proposition (Limits of Cohomology Groups - \cite{N}, 1.5.1):\label{LCG}}
\emph{If $G = \varprojlim_{i\in I} G_i$ and $A = \varinjlim_{i\in I} A_i$, then
$$H^n(G,A) \cong \varinjlim_{i\in I} H^n(G_i, A_i)$$}


\subsection{Corollary (Limits of Homology Groups): \label{LHG}}
\emph{If $G = \varprojlim_{i\in I} G_i$ and $A = \varprojlim_{i\in I} A_i$, then
$$H_n(G,A) \cong \varprojlim_{i\in I} H_n(G_i, A_i)$$}


\subsection*{Proof of \ref{LCG}}
The compatible pairs of maps give canonical homomorphisms $\kappa: C^n(G_i, A_i) \rightarrow C^n(G,A)$. Hence a homomorphism,
$$\kappa: \varinjlim_{i\in I} C^n(G_i, A_i) \rightarrow C^n(G,A)$$ which commutes with the boundary homomorphisms. Thus it is sufficient to show that $\kappa$ is an isomorphism which is achieved by using compactness arguments to reduce to finite quotients.

\begin{itemize}
\item SURJECTIVITY:

Let $y:G^n\rightarrow A$ be the inhomogeneous cochain associated to $x\in C^n(G,A)$. Since $G^n$ is compact, $A$ discrete and $y$ continuous, $y$ takes only finitely many values and factors through $\overline y : (G/U)^n\rightarrow A$ for a suitable open normal subgroup $U$. The finitely many values are represented by elements of some $A_i$ - $\overline y$ is the ccomposite of a function $\overline y_i: (G/U)^n\rightarrow A_i$ with $A_i\rightarrow A$. Also, there exists $j>i$ such that the projection $G\rightarrow G/U$ factors through the canonical map $G_j\rightarrow G/U$, and this gives the inhomogeneous cochain $y_j:G^n_j \rightarrow A_j$ as the composite
$$G_j^n \rightarrow (G/U)^n\xrightarrow{\overline y_i} A_i\rightarrow A_j$$
such that the composiite $G^n \rightarrow G^n_j \xrightarrow{y_j} A_j\rightarrow A$ is $y$. If $x_j\in C^n(G_j,A_j)$ is the homogeneous cochain associated to $y_j$ then its image in $C^n(G,A)$ is $x$. This gives the surjectivity of $\kappa$ 
\item INJECTIVITY:

Let $x_i\in C^n(G_i, A_i)$ be a cochain which becomes zero in $C^n(G,A)$, so that the composite
$$G^{n+1}\rightarrow G_i^{n+1}\xrightarrow{x_i}A_i\rightarrow A$$ is zero. Since $x_i$ has only finitely many values, there exists a $j\geq i$ such that the composite

$$G_j^{n+1}\rightarrow G_i^{n+1}\xrightarrow{x_i}A_i\rightarrow A_j$$ is already zero. Thus $x_i$ becomes zero in $C^n(G_j,A_j)$ and hence represents the zero class in \\ $\varinjlim_{i\in I} C^n(G_i,A_i)$. This gives the injectivity of $\kappa$.

\end{itemize}


%%%%%%%%%%%%%%%%%%%%%%%%%%%%%%
%%%%%%%%%%%%%%%%%%%%%%%%%%%%%%



\section{Completed Tensor Product \label{CTP}}

Firstly, recall that every compact $\ZP[[G]]$-module is the
projective limit of finite modules, and the category $\mathfrak C$
of compact modules has sufficiently many projectives and exact
inverse limits.

Also, every discrete $\ZP[[G]]$-module is the direct limit of
finite modules, and the category $\mathfrak D$ of discrete modules
has sufficiently many injectives and exact direct limits.

A tensor product for compact $\ZP [[G]]$-modules is defined by its
universal property. Explicitly, let M be a compact right and N be
a compact left $\ZP [[G]]$-module. Then the \textbf{complete
tensor product} is a compact $\ZP$-module $M\widehat \otimes_{\ZP
[[G]]} N$ coming with an $\ZP [[G]]$-bihomomorphism $\alpha$ (i.e.
$\alpha$ is a continuous $\ZP$-homomorphism such that $\alpha
(m\lambda, n) = \alpha (m, \lambda n)$ for all $m\in M, \, n\in N$
and $\lambda \in \ZP [[G]]$):

$$\alpha:M\, \text x \, N \rightarrow M\widehat \otimes_{\ZP
[[G]]} N$$

with the following property: given any $\ZP [[G]]$-bihomomorphism
$f$ of $M\, \text x \, N$ into a compact $\ZP$-module $R$, there
is a unique $\ZP$-module homomorphism $g: M \widehat \otimes N
\rightarrow R$ such that $f = g \circ \alpha$.

The complete tensor product may be constructed in the
following way:

$$M\widehat \otimes_{\ZP [[G]]} N = \varprojlim_{U,V} M/U
\otimes_{\ZP[[G]]} N/V$$

where $U$ (respectively $V$) run through the open
$\ZP[[G]]$-submodules of $M$ (respectively $N$). In \ref{completed} we will explicity identify submodules in calculation of Hochschild homology. Since $M/U$ and $N/V$ are both finite, each $M/U \otimes_{\ZP[[G]]} N/V$ is finite, and thus $M\widehat \otimes_{\ZP [[G]]} N$ is a compact $\ZP$-module. Moreover, the natural quptient maps $M\, \text x \, N \rightarrow M/U \otimes N/V$ induce the desired bihomomorphism $\alpha: M\, \text x \, N \rightarrow M\widehat \otimes N$ when passing to the limit. We see from the exact sequence,

$$0\rightarrow \text{im}(M\otimes V + U\otimes N)\rightarrow
M\otimes N \rightarrow M/U \otimes N/V\rightarrow 0$$ that infact
$M\widehat\otimes N$ is a completion of $M\otimes
N$ arising from the topology with fundamental system of open
neighbourhoods of zero given by $\text{im}(M\otimes V + U\otimes N)$.



%%%%%%%%%%%%%%%%%%%%%%%%%%%%%%
%%%%%%%%%%%%%%%%%%%%%%%%%%%%%%





\section{Homology of Completed Group Rings\label{HCGR}}

We are now in a position to define the Tor groups over the Iwasawa
algebra: $Tor^{\ZP[[G]]}_\bullet (-,-)$ is defined as the left
derived functors of (the right exact functor)
$-\widehat\otimes_{\ZP[[G]]} - $.

Induction takes the form, $M_G \cong \ZP
\widehat\otimes_{\ZP[[G]]} M$ giving the following crucial
theorem:

\subsection{Proposition (Completed "Tor" Groups and Group Homology - \cite{N} chapter V, 5.2.6):\label{4.6.1}} \emph{There
are canonical isomorphisms for all $i\geq 0$ and all compact
modules $M\in \mathfrak C$: $$H_i (G,M)\cong Tor_i^{\ZP[[G]]}
(\ZP,M)$$}

\subsection*{Proof}

The above functors agree for $i=0$. It is sufficient to show that a free $\ZP[[G]]$-module $F$ has trivial $G$-homology. Using ********* we reduce to the case $F=\ZP[[G]]$. We then have a compatible inverse system with the groups $G/U$ acting on the module $\ZP[G/U]$. By \ref{completed} we have,

$$H_i(G, \ZP[[G]]) =  \varprojlim_{U\subset G} H_i(G/U , \ZP [G])$$ where $U$ runs through the open normal subgroups of $G$. The key observation is that the $G/U$-module $\ZP[G] \cong \ZP\otimes_\Z \Z [G/U]$ is induced and hence is cohomologically trivial by \ref{propind}.


We can similarly define $"Ext"$ groups, and the 2 are related using
pontryagin duality, a self inverse mapping from an abelian profinite group, A (abelian compact group) to discrete abelian group (discrete abelian torsion group), given by considering a space of maps under compact-open topology:  where

$$A^V = \text{Hom}_{cont} (A,\QP /  \ZP)$$ where $\QP / \ZP$ is given quotient topology, (see \cite{N}, 5.2.9 for details),

\subsection{Theorem (Relationship Between "Tor" and "Ext" Groups):\label{TEG}}
\emph{For $M\in \mathfrak D$ a discrete module and $N\in \mathfrak C$
a compact module (guaranteeing both groups defined) there are
canonical isomorphisms for all $i\geq 0$: $$Tor_i^{\ZP[[G]]} (M,
N^V) \cong Ext^i_{\ZP[[G]]} (M,N)^V$$ where "$\,^V$" denotes the
Pontryagin dual}

Thus the theories of homology and cohomology are dual and we may use Pontryagin duality to easily switch between the two.


%%%%%%%%%%%%%%%%%%%%%%%%%%%%%%%%%%%%
%%%%%%%%%%%%%%%%%%%%%%%%%%%%%%%%%%%%
%%%%%%%%%%%%%%%%%%%%%%%%%%%%%%%%%%%%
%%%%%%%%%%%%%%%%%%%%%%%%%%%%%%%%%%%%
%%%%%%%%%%%%%%%%%%%%%%%%%%%%%%%%%%%%
%%%%%%%%%%%%%%%%%%%%%%%%%%%%%%%%%%%%

\subsection{Profinite Spaces and Profinite Groups}
%%%
\subsubsection{Lemma}
\emph{For a Hausdorff topological space $T$ the following conditions are equivalent:\begin{enumerate}
\item $T$ is the (topological) inverse limit of finite discrete spaces.
\item $T$ is compact and every point of $T$ has a basis of neighbourhoods consisting of subsets which are both open and closed.
\item $T$ is compact and totally disconnected.
\end{enumerate}}

The inverse limit of an inverse system of topological groups is just the inverse limit of groups together with the inverse limit topology on the underlying topological space.

\subsubsection{Definition}
\emph{A space $T$ is called a \textbf {profinite space} if it satisfies the equivalent conditions above.}

\subsubsection{Proposition}
\emph{For a Hausdorff topological group G the following conditiions are equivalent:
\begin{enumerate}
\item $G$ is the (topological) inverse limit of finite discrete groups.
\item $G$ is compact aand the unit element has a basis of neighbourhoods consisting of open and closed normal subgroups.
\item $G$ is compact and totally disconnected.
\end{enumerate}}

\subsubsection{Definition}
\emph{A Hausdorff Topological Group $G$  is called a \textbf {profinite group} if it satisfies the equivalent conditions above.}

Assume homos between profinite groups are continuous and subgroups closed. Since a subgroup is complement of non trivial cosets, open subgroups are closed, and closed subgroups opne iff it is of finite index. Can try to pass finite group theory to profinite group theory:

\subsubsection{Definition}
\emph{Let $G$ be a profinite group. A \textbf{topological} G-module $M$ is an abelian Hausdorff topological group M which is endowed with the structure of a G-module such that the action $G\, X \, M \rightarrow M, \, (g,m)\rightarrow g(m)$, is continuous. The term G-module, without the word topological will always refer to a discrete module - \textbf{the topology on M is the discrete topology}}.

\subsubsection{Lemma}
\emph{Let G be a profinite group and let M be a \textbf{discrete} G-module, then the following holds:
\begin{enumerate}\item For every $m\in M$ the subgroup $G_m = \{g\in G \vert g(m) = m\}$ is open.
\item $M = \bigcup M^U$ where U runs through the open subgroups of G\. - clear since $m\in M^{G_m}$.
\end{enumerate}}

\textbf{The groups $\Z, \, \Q, \, \Z/n\Z,\, \FP$ are always viewed as trivial discrete G-modules, modules with trivial action of G.}

\subsubsection{Definition}
\emph{We call the group $$A^V  = Hom_{cont} (A, \R / \Z)$$ the Pontryagin dual of A.}

Given locally compact topological spaces X,Y the set of continuous maps $Map_{cont} (X,YU)$ carries a natural topology, the compact-open topology, with basis the subsets:
$$U_{K,U} = \{ f\in Map_{cont}(X,Y_\vert f(K) \subset U\}$$ where K is compact subset of X and U is open subset of Y.

\subsubsection{Theorem - Pontryagin Duality}
\emph{If A is a Hausdorff abelian locally compact topological group, then the same is true for $A^V$ endowed with the compact open topology. There is a canonical homomorphism:
$$A\rightarrow {(A^V)}^V$$
given by
$$a\rightarrow \tau_a : A^V\rightarrow \R / \Z, \phi \rightarrow \phi (a)$$ is an isomorphism of topological groups. Commutes with limits and induces:
$$\text{(abelian compact groups)} \leftrightarrow \text{(discrete abelian groups)}$$ 
$$\text{(abelian profinite groups)} \leftrightarrow \text{(discrete abelian torsiongroups)}$$}

Given family $\{X_i\}_{i\in I}$ of Hausdorff, abelian topological groups let $Y_i\subset X_i$ be given for almost all $i\in I$.

\subsubsection{Definition}
\emph{The \textbf{restricted product}
$$\prod_{i\in I} (X_i, Y_i)$$ is the subgroup of $\prod_{i\in I} (X_i)$ such that $x_i\in Y_i$ for almost all i.}

The direct product is such an example. The restricted product is again a Hausdorff, abelian topological group.

\subsubsection{Proposition}
If all the $X_i$ are locally compact and almost all $Y_i$ are compact, then again the restricted product is a locally compact group. There is an isomorphism:

$$(\prod_{i\in I} (X_i, Y_i))^V \cong \prod_{i\in I} (X_i^V, (X_i/Y_i)^V)$$




%%%
\subsection{Definition of the Cohomology Groups}
%%%
Let $$X^n = X^n (G,A) = Map(G^{n+1},A)$$
Take $$d_i:G^{n+1}\rightarrow G^n: (\sigma_1,\dots \sigma_n)\rightarrow (\sigma_1,\dots \hat{\sigma_i},\dots  \sigma_n)$$
Which in turn induce $d_i^*$ on the $X^{i-1}\rightarrow X^i$. Now define, $$\delta^n = \sum_{i=0}^n (-1)^i d_i^*$$

\subsubsection{Proposition}
\emph{The sequence
$$0\rightarrow A\rightarrow^{\delta^0} X^0\rightarrow^{\delta^1}X^1\rightarrow^{\delta^2}X^2\rightarrow^{\delta^3}X^3\rightarrow\dots$$ is exact}

Hence we have a resolution of $A$ by G-modules, applying fixed module functor leads to cohomology groups:

\subsubsection{Definition}
\emph{For $n\geq 0$ the factor group
$$H^n(G,A) = Z^n(G,A) / B^n(G,A)$$ is called the n-dimensional cohomology group of G with coefficients in A.}

Below trick reduces number of variables in computation by one.

Also often refine defintion of 0th cohom groups in the following sense:
For G a finite group, the norm residue group,
$$\hat H^0 (G,A) = A^G / N_G A$$ where $N_G A$ is the image of the norm residue map $$N_G:A\rightarrow A, \, N_G a = \sum_{\sigma\in G} \sigma a$$

The modified cohomology groups are usual ones except 0th is the norm residue group, equivalent to extending the resolution:

\subsubsection{Proposition}
\emph{We have an exact sequence
$$0\rightarrow \hat H_0(G,A)\rightarrow H_0(G,A)\rightarrow^{N_G}H^0(G,A)\rightarrow \hat H^0 (G,A)\rightarrow 0$$}

Now let G be a profinite group and A a G-module. For every pair of normal subgroups $V\leq U$ of G, we have homos:
$$\hat H^0(G/V,A^V)\rightarrow \hat H^0(G/U,A^U)$$
$$\hat H_0(G/V,A^V)\rightarrow \hat H_0(G/U,A^U)$$
Induced from $id:A^G\rightarrow A^G$, respectively $N_{U/V}:A^V\rightarrow A^U$.
Now define:
$$\hat H^0(G,A) = \underleftarrow{Lim}_U H^0 (G/U, A^U)$$

Similarly for homology giving relations of the form:
$$0\rightarrow N_G A \rightarrow A^G \rightarrow  H^0(G,A)$$ for $N_G A = \underleftarrow{Lim} N_{G/U} A^U$.

The complete standard resolution of A is defined as the sequence $X^{-1-n} = Hom (X_n,A)$ giving:
$$\dots\rightarrow X^{-2}\rightarrow X^{-1}\rightarrow X^0\rightarrow X^1\rightarrow X^2\rightarrow\dots$$

Now, for every $n\in \Z$ the n-th cohomology group $\hat H^n (G,A)$ is defined as the homology group of the complex
$$\hat C^\bullet (G,A) = ((X^n)^G)_{n\in \Z}$$
Can interpret small cohomology groups in terms of crossed homomorhisms and pointed sets.

The cohomology groups $H^n(G,A)$ of a profinite group G with coefficients in a G-module A are built up in a simple way from those of the finite factor groups of G.

$$G^{n+1}\rightarrow (G/V)^{n+1}\rightarrow (G/U)^{n+!}$$ induces maps commuting with differentials:
$$C^n(G//U,A^U)\rightarrow C^n(G/V, A^V)\rightarrow C^n(G,A)$$ giving homos:
$$H^n(G//U,A^U)\rightarrow H^n(G/V, A^V)\rightarrow H^n(G,A)$$therefore the groups form a direct system, with homo:
$$\mathbf{\underleftarrow {Lim} H^n (G/U , A^U) \rightarrow H^n (G,A)}$$
which is infact an IOM




%%%
\subsection{The Exact Cohomology Sequence}
%%%
\subsubsection{Definition}
\emph{A G-module A is called acyclic if $H^n(G,A) =0$ for all $n>0$, and cohomologically trivial $$H^n(H,A) =0$$ for all closed subgroups H of G, and all $n>0$. INDUCED MODULES ARE COHOM TRIV.}

If G is a finite group, $Ind_G(A) = Map(G,A) \cong A\circ \Z [G]$

\subsubsection{Proposition}\begin{enumerate}
\item The functor $A\rightarrow Ind_G(A)$ is exact.
\item An induced G-module A is also an induced H-module for every closed subgroup H of G, and for H normal, $A^H$ is an induced $g/H$-module.
\item If one of A or B is induced so are $a\circ B$ and $Hom(A,B)$
\item For U running through open subgroups of G,
$$Ind_G(A) = \underleftarrow{Lim} Ind_{G/U} (A^U)$$\end{enumerate}

This leads to
\subsubsection{Proposition}
The induced G-modules are cohomologically trivial, and for G-finite we have moreover the norm residue group, $\hat H^n (G,M) = 0$ for all $n\in \Z$.

We can now perform \textbf{dimension shifting}:












%%%
\subsection{The Cup-Product}
%%%
%%%
\subsection{Change of the Group G}
%%%
%%%
\subsection{Basic Properties}
%%%
%%%
\subsection{Cohomological Triviality}


%%%%%%%%%%%%%%%%%%%%%%%%%%%%%%%%%%%%


