\section{Lie Groups}

Before going on to define the cohomology of Lie groups and Lie
algebras I would like to recall the key properties of these
structures and of Lie Theory - the method of passing from a Lie
group to its Lie algebra. Later, this report will concentrate on
the relation of the cohomology of a Lie group to the cohomolgy of
its associated Liealgebra.

\subsection{Manifolds}\label{df2.1}
I will recall the definition of tangent space which will be needed
for Lie Theory later.

\subsection{Analytic Manifolds}\label{df2.1.1}
Recall, for X a topological space, a chart c on X is a triple
$(U,\varphi, n)$ such that:
\begin{enumerate}
    \item $U\subset X$ is open
    \item $n\in \Z$ and $n\geq 0$
    \item $\varphi : U \mapsto \varphi U \subset k^n$ is open and
    $\varphi$ is a homeomorphism.
\end{enumerate}
Charts allow considerations of continuity, analyticity,.. of maps
to be passed to questions about the well understood maps $k^n
\mapsto k^m$ using homeomorphism $\varphi$. Two charts $c,c'$ are
compatible if they behave well on their intersection $V = U \cap
U'$ - if the maps $\varphi' \circ \varphi^{-1} |_{\varphi (V)}$
and $\varphi \circ \varphi'^{-1} |_{\varphi' (V)}$ are analytic.

An atlas is a collection of compatible charts which "span" X,
meaning that $\bigcup U$ cover X. This leads to the notation of
compatibility of atlases - where charts are pairwise compatible.

We can now define X as an analytic manifold if it has the extra
structure of an equivalence class of compatible atlases.

To understand a morphism $f$ of manifolds we again use charts to
pass to simple maps $k^n \mapsto k^m$ and require that $f$ induces
a continuous map which is "locally given by analytic functions".
So, for a sufficiently refined chart, the maps on coordinates are
analytic.

\subsection{Tangent Spaces}\label{df2.1.2}

For $x\in X$, define $T_x X = {( {\textbf m_x}/{\textbf m^2_x})}^*
= $ tangent space of X at $x$, where $\textbf m_x$ denotes
functions vanishing at $x$.

\subsection*{Claim} $T_x X$ is canonically isomorphic to the
space $\underline{C_x}$ of "tangent classes of curves at $x$".

Taking $\underline{F'_x} = \{$pairs $(N, \phi):\, 0\in N \subset
k$ is an open neighborhood, $\phi:N\mapsto X$ such that $\phi (0)
= x \}$. \\
We define an equivalence relation on $\underline{F'_x}$. We say
$(N_1, \phi_1)$ is equivalent to $(N_2, \phi_2)$ if $D(\varphi
\circ \phi_1)(0) = D(\varphi \circ \phi_2)(0)$ - this is a valid
definition since $\varphi \circ \phi_i$ is defined in the
neighborhood $N_i\cup \phi_i^{-1} (U)$ of 0.

Then the tangent space, $\underline{C_x}$ is the set of
equivalence classes of $\underline{F'_x}$. The definition of
$\underline{C_x}$, and of its induced vector space structure
(corresponding to adding linear maps given by differentiating) is
independent of the choice of chart.

For $x\in X$, denote $\underline{F_x}$ for the set of pairs
$(U,\varphi)$ where U is an open neighborhood of $x$, and
$\varphi$ is an analytic function on U.

We define the pairing:
$$\underline{F'_x} \, X\, \underline{F_x} \mapsto k\,:\, (N,\phi)
\, X \, (V,f) \mapsto D(f\circ \phi )(0) \in k$$ This induces a
pairing $\underline{C_x}\, X\, T_x^* X \mapsto^\omega k$ which is
bilinear and gives required duality - $\underline{C_x}$ is the
dual of $T_x^* X$.

Of course, the pairing $\omega$ is simply differentiation of a
function in the direction of the tangent to a curve.

As the simplest example, for a vector space V, and for any point
$x\in V$ (it doesn't matter which point we choose by homogeneity),
$T_x V = L(k,V) = V$ - the tangent to a curve through V is a
vector in V.

Later, we will consider tangent spaces at $x=1\in G$, and analytic
group when producing its associated Lie algebra $\mathcal G$.





\subsection{Analytic Groups}\label{df2.2}

\subsection{Definition}\label{df2.2.1}

For G a topological group and an analytic manifold over k (a field
complete with respect to a non-trivial absolute value). G is said
to be an \textbf{analytic group} or \textbf{lie group} if the
following hold:
\begin{enumerate}
    \item The map $G\,X \, G \mapsto G:\,(x,y)\mapsto xy$ is a
    morphism.
    \item The map $G\mapsto G: x\mapsto x^{-1}$ is a morphism.
\end{enumerate}
These two conditions immediately give structure to the group, and
allow us to study any neighborhood as a translation by (1) of a
neighborhood of the origin. For example, since taking a chart, G
is locally isomorphic to an open subset of $k^n$ (some n), the
intersection of neighborhoods of the identity if $\{ 1\}$, and it
is a standard result that this gives G is Hausdorff.

There are two examples which become important later in this
report:
\begin{enumerate}
    \item \textbf{General Linear Groups}

For a finite dimensional algebra R over k, we denote $G_m(R)$ for
its group of invertible elements. It is clear that multiplication
is a morphism since multiplication in R is bilinear, and has an
obvious inverse in the group.

Familiarly, for $R = End(V)$, the endomorphism ring of a finite
dimensional vector space V/k, we call $G_m(R)$ the general linear
group of V, $GL(V)$.

When $V=k^n$, we use the notation $GL(V) = GL(n,k) = {GL}_n(k)$,
whose elements may be represented by invertible matrrices. For a
valuation ring $A/k$, let $GL(n,A)\subset GL(n,k)$ be defined by
$$GL(n,A) = \{ (\alpha_{i,j}) | (\alpha_{i,j})\in A\text{ is an auto, } det(\alpha_{i,j})\in\text{ units of }A
\}$$ Then the group $GL(n,A)$ is open and closed in $E(k^n)$ and
hence an analytic group. When k is locally compact, $GL(n,A)$ is a
compact open subgroup of $GL(n,k)$.
    \item \textbf{Lie Group $\QP$}

    Consider $\QP$ as a degenerate Lie group over the field $\QP$
    with inherited multiplication and inverse. We have the theorem
    that if k is locally compact then:
    \subsection{Theorem}\label{df2.2.2}\label{df2.2.3}
    $GL(n,A)$ is a maximal compact subgroup of $GL(n,k)$ and, if G
    is a maximal compact subgroup of $GL(n,k)$, then G is a
    conjugate of $GL(n,A)$.

    In this case we can use maximality of $\ZP$ in $\QP$ to reduce
    the study to that of discrete Lie groups and algebras over
    $\ZP$.

    \end{enumerate}


\subsection{Formal Groups}\label{df2.3}

When $R=k$, a complete field, we shall use formal groups to define
a functor T:
$$\textbf{Analytic Groups $\rightarrow$ Lie Algebras}$$
Central to this technique will be the 1-1 correspondence between
Lie algebras and formal groups which i will now describe.

In a way, Lie Groups are "locally" formal groups(or equivalently,
Lie algebras) - which are a kind of "linearisation" of the system.

\subsection*{Definition}
Let R be a commutative ring with a unit, and consider the Formal
Power Series ring, $R[[X_1, \dots ,X_n]] = R[[X]]$ in n-variables,
and let $Y= (Y_1, \dots Y_n)$ be a set of a further n variables.

Then, \textbf{A Formal Group Law} in n variables is an n-tuple $F
= (F_i )$ of formal power series, $F_i \in R[[X,Y]]$, such that:

\begin{enumerate}
    \item $F(X,0) = X,\, F(0,Y) = Y$
    \item $F(U, F(V,W)) = F(F(U,V),W)$, a kind of associativity.
\end{enumerate}

This immediately gives tight restrictions on structure - each
$F_i$ has the form:
$$F_i(X,Y) = X + Y +\sum_{|\alpha|\geq 1, |\beta|\geq 1}
c_{\alpha, \beta} X^\alpha Y^\beta$$

We can think of a Formal Group Law as a bifunctor, which
approximates to first order (possibly after shifting the origin)
to \textbf{addition}.

We could simply take $F(X,Y) = X+Y$.

Alternatively, we could consider multiplication of 2 elements
taking origin as the multiplicative identity: $(1+X).(1+Y) = 1+X +
Y + XY$ gives rise to the Formal Group Law $F(X,Y) = X+ Y +XY$.

Formal Groups are very important in the study of the Group Law for
rational points on Elliptic Curves:


\subsection{Group Law for Points on Elliptic
Curves}\label{df2.3.1} Given a Weierstrass Equation in x,y
describing an Elliptic Curve \textbf E, we make the change of
coordinates:
$$z=x/y\text{ and } w=-1/y$$
ie.
$$x=z/w\text{ and } y=-1/w$$
The Weierstrass Equation then becomes:
$$w=z^3 +a_1 zw +a_2 z^2 w + a_3 w^2 +a_4 z w^2 +a_6 w^3$$
We then substitute this equation recursively (and checking
convergence) to give:
\subsection{}\label{df2.3.2}
$$ w(z) = z^3 (1+A_1 z +A_2 z^2 + \dots)\in \Z [a_1, \dots ,
a_6][[z]]$$ where each $A_n \in \Z [a_1, \dots , a_6]$ is
homogeneous of weight .

\ref{df2.3.2} gives rise to Laurent series for x and y:
\begin{eqnarray}
\nonumber x(z) &=& \frac{z}{w(z)} = 1/z^2 - a_1/z - a_2 - a_3z -
(a_4+a_1 a_3)z^2 - \dots\\
\nonumber y(z) &=& \frac{-1}{w(z)} = -1/z^3 +a_1/z^2 +a_2/z +a_3
+(a_4 +a_1a_3)z +\dots
\end{eqnarray}
We now express the additive group law on \textbf E using these
power series in place of x and y.

Let $z_1,z_2$ be independent indeterminants, define $w_i = w(z_i)$
for $i=1,2$. Standard calculations on this new Elliptic Curve
(defined by its Weierstrass Equation) gives the z-coordinate of
$\ominus (z_1 \oplus z_2)$, called $z_3$ as
\begin{eqnarray}
\nonumber z_3 &=& z_3(z_1, z_2)\\
\nonumber     &=& -z_1 - z_2 +\frac{a_1 \lambda +a_3 \lambda^2 -
a_2\nu -2a_4 \lambda\nu - 3 a_6 \lambda^2\nu}{1+a_2\lambda
+a_4\lambda^2 +a_6 \lambda^3}\\
\nonumber &\in& \Z[a_1,\dots , a_6][[z_1,z_2]]
\end{eqnarray}
Inverting this point, we get an expression for $z(z_1\oplus z_2)$:
\begin{eqnarray}
\nonumber F(z_1,z_2) &=& i(z_3(z_1,z_2))\\
\nonumber            &=& z_1 +z_2 - a_1z_1z_2 - a_2 (z_1^2 z_2
+z_1z_2^2)+\dots \\
\nonumber            & & -(2a_3z_1^3z_2 - (a_1a_2 -3a_3)z_1^2z_2^2
+ 2a_3z_1z_2^3)+\dots \\
\nonumber        & &   \in \Z[a_1,\dots ,a_6][[z_1,z_2]]
\end{eqnarray}

We define this power series as a formal group law $F(z_1,z_2)$ -
all required properties of associativity and commutivity are
inherited from the geometric interpretation of the group law.

The following lemma, whose proof is analogous to the calculation
of example \ref{df2.3.8} is important in handling torsion points.

\subsection{Lemma}\label{df2.3.3}
\emph{Let $a\in R^*$ and $f(T) \in R[[T]]$ a power series starting
$$f(T)=aT+\dots$$
Then there is a \textbf{unique} power series $g(T)\in R[[T]]$ such
that $f(g(T)) = T$. It further satisfies $g(f(T)) = T$.}

We can now prove the central position in handling addition in
these new coordinates:

\subsection{Proposition}\label{df2.3.4}
\emph{Let $\mathcal F$ be a formal group over R, and let $m\in
\Z$.\begin{enumerate}
    \item $[m](T) = mT + (\text{higher order terms})$.
    \item If $m\in R^*$, then $[m]:\mathcal F\mapsto \mathcal F$
    is an isomorphism.
\end{enumerate}}
\subsection*{Proof}
\begin{enumerate}
    \item For positive m it is immediate from the formal power
    series $F$. For negative m observe that the inversion $i$ negates
    the dominant term.
    \item This follows from the above lemma, \ref{df2.3.3}. $g$
    provides a two-sided inverse to $f$, and hence $[m]$ has an
    inverse and is thus an isomorphism.
\end{enumerate}
The following plays an important role in proving the Mordell-Weil
Theorem on the structure of rational points on an Elliptic curve -
that, \emph{The Group E(K) is finitely generated}.

\subsection{Proposition}\label{df2.3.5}
\emph{Let p be the characteristic of k (p=0 being allowed). Then
every torsion element of $\mathcal F (\mathcal M)$ has order a
power of p.}
\subsection*{Proof}
We need to only consider torsion elements of order prime to p by
multiplying by an arbitrary torsion element of an appropriate
value of p. Using this trick, let $m\geq 1$ such that $(m,p)=1$,
$x\in \mathcal F(\mathcal M)$ an element such that $[m](x)=0$ and
we need to show $x=0$.

Since m is prime to p, $m\notin \mathcal M$, hence from
\ref{df2.3.4} $[m]$ is an isomorphism of formal groups and induces
$[m]: \mathcal F(\mathcal M) \cong \mathcal F(\mathcal M)$.

This has trivial kernel, hence $x=0$, as required.






\subsection{Formulae}\label{df2.3.8}
We now study general expressions for the structure of Formal
Groups. We use $O(d^0 \geq n)$ to denote a Formal Power Series
whose homogeneous parts of degree strictly less than n vanish.

\begin{itemize}
    \item The structure from the definition may be written:
    $$F(X,Y) = X+Y +B(X,Y) + O(d^0\geq 3)$$ where B is a bilinear
    form. We now set
    $$[X,Y] = B(X,Y) - B(Y,X)$$ and I shall refer to this
    bifunctor as the "Lie Bracket" associated to the Formal Group.
    \item Consider the construction of the inverse power series
    $\varphi$ of Theorem \ref{df2.2.3} satisfying $F(X,
    \varphi(X)) = 0 = F(\varphi(X),X)$

    Let $\varphi_i(X)$ be the i-th homogeneous part of
    $\varphi(X)$.

    Since $F(X,\varphi(X)) = X+\varphi_1(X) + O(d^0\geq 2) = 0$.

    We have $\varphi(X) = -X +\varphi_2(X)+ O(d^0\geq 2)$.

    \begin{align}
    \nonumber \text{Similarly } F(X,\varphi(X)) &= X+(-X+\varphi_2(X) +\dots) +
    B(X,-X+\dots)+\dots\\ \nonumber &=\varphi_2(X) - B(X,X) + O(d^0\geq 3).
\end{align}

    Hence, $\varphi_2(X) = B(x,X)$.

    Combining gives:
    $$\varphi(X) = -X +B(X,X) + O(d^0\geq 3)$$

\end{itemize}

The following are preliminary to a key theorem on the
    associativity of the "Lie Bracket" we have constructed.

\subsection{Lemma}\label{df2.3.9}
$$XYX^{-1} = Y+ [X,Y] +O(d^0\geq 3)$$
This follows by careful book keeping when expanding the brackets.
Similar results hold for $Y^{-1}$, $X^{-1}Y^{-1}XY$, and together
lead to an identity of Hall:

\subsection{Proposition: Hall}\label{df2.3.10}
$$(X^Y,(Y,Z)).(Y^Z,(Z,X)).(Z^X,(X,Y)) = 0$$

Finally examining this identity to order 3 recovers Jacobi's
identity, justifying my description of $[X,Y]$ as a Lie Bracket:

\subsection{Identity: Jacobi}\label{df2.3.11}
$$[X,[Y,Z]]+[Y,[Z,X]]+[Z,[X,Y]] = 0$$




\subsection{Formal Groups Giving Rise to Analytic
Groups}\label{df2.4}

Let k be a complete Ultrametric field, and let L be the valuation
ring of k. Denote the maximal ideal of A by \textbf m. For
$F(X,Y)$ a formal group law over A, we construct an analytic
group:

\emph{Let $G=\{(x_1,\dots , x_n): x_i \in \textbf{m}\}$ and define
multiplication on G by the formula $xy=F(x,y)$.}

A group G arising from the above construction is known as
\textbf{standard}.

\subsection*{Definition}
Recall that an \textbf{analytic group chunk} is a topological
space X together with a distinguished element $i\in X$, and an
open neighborhood U of i in X, together with a pair of maps
$\varphi:U\,X\,U\mapsto X$, and $\phi:U\mapsto U$ such that:
\begin{enumerate}
    \item For some neighborhood $V_1$ of e in $U$, $x\in V_1 \Rightarrow$
    $$x= \varphi(x,e) = \varphi(e,x)$$
    \item For some neighborhood $V_2$ of e in $U$, $x\in V_2
    \Rightarrow$
    $$e=\varphi(x,\phi x) = \varphi (\phi x, x)$$
    \item For some neighborhood $V_3$ of e in $U$,
    $\varphi(V_3\,X\, V_3) \subset U$, and for all \\ $x,y,z\in V_3$:
    $$\varphi(x,\varphi(y,z)) = \varphi(\varphi(x,y),z)$$
\end{enumerate}
Clearly $\varphi, \phi$ arise locally from a multiplication, and
an inverse, and by shrinking U can we can assume this is true on
the whole of U.

The following theorem shows how analytic groups naturally have a
substructure of a formal group.



\subsection{Theorem}\label{df2.4.1}
\emph{Any analytic group chunk contains an open subgroup which is
standard.}

The proof proceeds by shrinking G so that the Formal Power Series
Converges, and then the strict unit ball gives the desired open
subgroup.

As an immediate corollary we see that in fact any analytic group
chunk is equivalent (ie. there exists local homeomorphisms between
spaces such that ordered pairs of composition are equivalent to
the identity) to an analytic group.





\section{Lie Theory}\label{df2.5}

\subsection{Lie Algebra of an Analytic Group
Chunk}\label{df2.5.1}

For a group chunk G/k, define as set isomorphisms $L(G) = \mathcal
G = T_eG$, the tangent space at the identity, see \ref{df2.1.2}. I
now define a Lie Algebra structure on $\mathcal G$.

To get a handle on the calculations take a chart $c = (U, \varphi,
n)$ of G at e. The group law on G is induced (via $\varphi$) from
a formal group law $F$ on $k^n$. Denoting the isomorphism
$\mathcal \cong k^n$ induced by $\varphi$, by
$\overline{\varphi}$. Define

$$\overline{\varphi} {[x,y]}_c = {[\overline \varphi x, \overline
\varphi y]}_F$$

A homomorphism between charts passes to a Lie Algebra homomorphism
and hence the induced Lie Bracket is independent of c.

\subsection{Definition}\label{df2.5.2}
$\mathcal G$ together with it's canonical Lie Algebra structure is
the \textbf{Lie Algebra of G}.

This construction provides a converse to Theorem \ref{df2.4.1},
since from a given group chunk with given formal group we can
construct the associated standard algebra. When we are working
with analytic groups, the local functions $\phi, \varphi$ of group
chunks are global and induce, via charts the formal group $F$ on
$k^n$, this in turn induces the Lie algebra.

\subsection*{Example}

Let R be an associative algebra of finite dimension over k. From
\ref{df2.2.1}, $G_m(R)$ is an analytic group. $T_1G_m(R) = R$.
Multiplication in $G_m(R)$ has the form $(1+x)(1+y)=1+x+y+xy$.
Taking the chart
\begin{eqnarray}
\nonumber\varphi &:& R\mapsto k^n\\
\nonumber &:&z\mapsto z-1
\end{eqnarray}
We have
\begin{eqnarray}
\nonumber \overline{\varphi}{[x,y]}_c &=& {[x-1,y-1]}_F\\
\nonumber          &=& 1+{[x,y]}_c \\
\nonumber          &=& (1+x)(1+y)\\
\nonumber          &=& 1+x+y+xy
\end{eqnarray}
Hence, \emph{$F(x,y)=x+y+xy = x+y+B(x,y)$}. Hence, the Lie algebra
structure on $T_1G_m(R) =R$ is given by
$$[x,y]=xy-yx$$
Note that in the case where R is an endomorphism ring we recover
the usual Lie algebra structure.

\subsection{Linear Action}\label{df2.5.3}
Consider a Lie Group G acting on a vector space V. For the
continuation, the most important case to us is deducing the action
of it's Lie Algebra on V (and hence calculating the Lie Algebra
Cohomology with coefficients in V).

Formally, a \emph{Linear Representation of G in V} is an analytic
group homomorphism $\sigma: G\mapsto GL(V)$. The, an element $g\in
G$ acts on V via $\sigma$:
$$g.v = \sigma(g)(v)\,\forall v\in V$$

The induced homomorphism $\overline\sigma : L(G)\mapsto E(V)$
gives an induced representation of $L(G)=\mathcal G$ on V.

\subsection{Examples of Linear Representations}\label{df2.5.4}
I now give two examples of this, there will be more in my own
calculations, see chapter 7.

\begin{enumerate}
    \item \textbf{Determinants:}

    Let $G=GL(V)$. Then $det:G\mapsto G_m(k) = k^*$ the
    determinant map is an analytic homomorphism.
    Take $x$ as a representative of an element $1+x\in \mathcal G
    = L(G)$. We have
    \begin{eqnarray}
    \nonumber det(1+x) &=& 1+ tr(x) + \dots + det(x) \\
    \nonumber          &=& 1+tr(x) + O(d^0\geq 2)
    \end{eqnarray}
    Hence, \emph{$L(det)(x) = tr(x)$} as Lie Theory picks out the
    dominant term only (replaces curves by their linear tangents).
    This is a good example of how the Lie Algebra is a
    linearisation of the Lie Group, and so is often simpler. This
    simplicity makes associated calculations of cohomology easier,
    while the results of Lazard allow us to pass back ti get
    information on the hard calculations of cohomology of Lie
    groups.

    Conversely, we may turn this process round, since from
    \cite{S2}, p152, there is a theorem of Lie:
    \subsection{Theorem: Third Theorem of Lie}\label{df2.5.5}
    For any Lie Algebra $\mathcal G$ there exists a connected and
    simply connected analytic group G such that $L(G)=\mathcal G$.
    This allows us to use the theory of Lie groups to treat Lie
    algebras.
    \item \textbf{Tensoring:}

    Let $V_1,\dots , V_n$ be vector spaces. Take
    $V=V_1\otimes\dots\otimes V_n$. Then $G=\prod_{i=1}^n GL(V_i)$
    acts on V in a natural way via
    \begin{eqnarray}
    \nonumber \Theta &:& E(V_1)\, X\dots X\, E(V_n)\mapsto E(V)\\
    \nonumber        &:& (u_1,\dots , u_n) \mapsto u_1\otimes
    \dots \otimes u_n
    \end{eqnarray}
    To work with the Lie Algebra we take a representative in the
    neighborhood of 1: $\Theta (1+x_1, \dots , 1+ x_n) = 1
    +\sum_{i=1}^n 1\otimes \dots \otimes x_i \otimes \dots \otimes
    1 +O(d^0\geq 2)$. Hence, translating the origin:
    $$L\Theta (x_1, \dots , x_n) = \sum_{i=1}^n 1\otimes \dots
    \otimes x_i\otimes \dots \otimes 1$$
\end{enumerate}








%%%%%%%%%%%%%%%%%%%%%%%%%%%%%%%%%%%%%%%%%%%%%%%%%%%%%%%%%%%%%%%%%%%%%%%%%%%%%%%%%%%%%%%%%%%%%%%%%%%%%%%%%%%%%%%%%%%%%%%
%%%%%%%%%%%%%%%%%%%%%%%%%%%%%%%%%%%%%%%%%%%%%%%%%%%%%%%%%%%%%%%%%%%%%%%%%%%%%%%%%%%%%%%%%%%%%%%%%%%%%%%%%%%%%%%%%%%%%%%
%%%%%%%%%%%%%%%%%%%%%%%%%%%%%%%%%%%%%%%%%%%%%%%%%%%%%%%%%%%%%%%%%%%%%%%%%%%%%%%%%%%%%%%%%%%%%%%%%%%%%%