\cite{CFKSV} explains how characteristic elements lying in the Whitehead group of a Localized Iwasawa algebra code interesting information for non-commutative Iwasawa Theory.

In this section I consider whether the generalized Dennis Trace Maps discussed in \ref{general dtm} behaves well with respect to localization by the canonical Ore set, the image of the characteristic element under this map lying in the first Hochschild homology of a localized Iwasawa algebra. Hence, the study of how $K_1$ and $HH_1$ behave with respect to localization of the ring is relevant.

\subsection{Commutative Localization}
For $R$ a commutative ring, let $S$ be a subset of $R$ which is multiplicatevly closed. The idea is to invert elements of $S$ to give a ring $R_S$ whose units are precisely the elements of $S$.

If $S$ contains zero divisors, wjen we invert $S$, the elements of $R$ which kill an element of $S$ must be zero in $R_S$. Define the assasinator of $S$ as follows:

$$\text{ass}(S) = \{ x\in R:xs=0\text{ for some } s\in S\} = \bigcup_{s\in S} \text{ann}(s)$$


Then a localisation of $R$ at $S$ is a ring $R_S$ with a homomorphism $\phi:R\rightarrow R_S$ (and if $R_S$ exists it is unique up to isomorphism given these conditions)
\begin{enumerate}
\item $\phi(s)$ is a unit in $R_S$ for all $s\in S$.
\item Every element of $R_S$ can be written in the form $\phi(r) \phi(s)^{-1}$ for soome $r\in R$ and $s\in S$.
\item ker$\phi = \text{ass}(S)$, so $\phi$ is not in general injective
\end{enumerate}

The localisation may be thought of as a ring of fractions $r/s <-> (r,s)$ modulo the equivalence relation:
$$(r,s)\sim (u,v) \text{if and only if} rvt=uvt \text{ for some } t\in S$$

$R$-modules may be localised to $M\otimes_R R_S$ which we may think of as the set of equivalence classes $\{m/s : m\in M, s\in S\}$ in $M$ x $S$ modulo the equivalence:

$$(m,s)\sim (n,t) \text{ if and only if } mtu = nsu \text{ for some } u\in S$$

One of the most important properties of localization is that it is \emph{exact}, meaning it takes exact sequences to exact sequences. Given $f:M\rightarrow N$ a map of $R$-modules, we define a map $f_S:M_S\rightarrow N_S$ by setting
$$f_S(m/s) = f(m)/s \text{ for all } m\in M, s\in S$$

\subsection{Non-Commutative Localization}
There are much tighter conditions on $S$ for a localization to exist. Suppose a right localisation (considering right fractions $rs^{-1}$) exists, then for all $r\in R$, $s\in S$ the element $\phi(s)^{-1}\phi (R)$ must lie in $R_S$, and so is expressable as
$$\phi(s)^{-1}\phi (r) = \phi(a) \phi (b)^{-1}$$

Hence $\phi(rb-sa) = 0$ so there exists a $t\in S$ such that $r(bt) - s(at)$, and $bt\in S$ since $S$ is mult. closed.

\subsubsection*{Definition (Ore Set)}
\emph{A mult. closed set $S$ is a right Ore set, if and only if for all $r\in R, s\in S$ there exist $r'\in R, s'\in S$ such that $rs'=sr'$}. 

This is automatically satisfied in Commutative rings.

I have already explained that if the right localisation exists, then $S$ must be an Ore set, but conversely if $S$ is an Ore set then ass$(S)$ is a two-sided ideal of $R$, and provided $\overline S$ consists of regular elements in $\overline R = R / \text{ass} (S)$ then $R_S$ exists.






\subsection{Localization of $K_1$-groups}
We showed in \ref{localgroup} that the Iwasawa algebra is local (with unique maximal ideal the kernel of the composition of augmentation and reduction modulo $p$: $\Lambda_G\rightarrow \ZP\rightarrow \FP$), hence semilocal and the elements of $K_1$ are all realized as units, see \ref{whitehead semilocal}.

Using that Localisation is exact, and a technical Lemma it is established in \cite{CFKSV}, Proposition 4.2,  that $(\Lambda_G)_S$ is semi-local allowing us to again apply \ref{whitehead semilocal} so the elements of $K_1[(\Lambda_G)_S]$ are all realized as units.

However we are interested in characteristic elements lying in $K_1[(\Lambda_G)_S^*]$ where $S^* = \bigcup p^n S$. \cite{CFKSV} explains that this Ring need no longer be semilocal. In False Tate Curve, $\Lambda_G = \ZP[[U,V]]$ where $U+1$, $V+1$ correspond to fixed topological generators, and $S$ is the complement of $(p,U)$ in $R$. From above, $R_S$ is a local ring of dimension 2. So $R_{S^*}= R_S [1/p]$ is a ring of dimension 1. We see for $g$ any irreducible distinguished polynomial in $\ZP[U]$, $gR$ is a prime ideal, so $gR_{S^*}$ is also prime ($S^* \cap gR$ is empty) giving $R_{S^*}$ has infinitely many prime ideals, so is not semilocal.

However, the statement of Vasserstein's theorem on $K_1$ still holds.

\subsubsection*{Theorem ($K_1$ of Localisation of Completed Group Rings - see \cite{CFKSV} 4.4):}
\emph{Assume that $G$ has no element of order $p$. Then the natural map,
$$(\Lambda_G)_{S^*}^*\rightarrow K_1[(\Lambda_G)_{S^*}] \text{ is surjective}$$}

Kato shows, for $G$ an open subgroup of the Galois group corresponding to the False Tate extension with associated finite group $\Delta$ (see \ref{section14}),

$$K_1(\ZP[[G]]_{S^*}) \cong K_1(\ZP [[G]]_{S}) \oplus \Z^{|\Delta|}$$

Thus the study of $K_1(\ZP[[G]]_{S^*})$ is reduced to the study of $K_1(\ZP [[G]]_{S})$. Finally, after localising the ideals $A_{G,n}$ of \ref{section14} he produces a group $\phi_S(G)$, and a homomorphism (\cite{K}, 8.14), 
$$\theta_{G,S} : K_1(\ZP [[G]]_{S}) \rightarrow \phi_S(G)$$
which is conjectured to be bijective, and shows how this depends on the bijectivity of a simpler map.


%%%%%%%%%%%%%%%%%%%%%%%%%%%%%%%%%%%%%%%
%%%%%%%%%%%%%%%%%%%%%%%%%%%%%%%%%%%%%%%
%%%%%%%%%%%%%%%%%%%%%%%%%%%%%%%%%%%%%%%
%%%%%%%%%%%%%%%%%%%%%%%%%%%%%%%%%%%%%%%
%%%%%%%%%%%%%%%%%%%%%%%%%%%%%%%%%%%%%%%

\subsection{Localization of $HH_1$-groups \label{12.4}}
\subsubsection*{Theorem (Localisation of $Tor$-groups - see \cite{W} 8.7.3):}
\emph{Let $S$ be an Ore Set in ring $R$, and $M$ and $N$ two $R$-modules. Then
$$Tor_*^{R_S}(M_S,N_S)\cong Tor_*^R(M_S,N)\cong  [Tor_*^R (M,N)]_S$$}

Since the Hochschild homology may be considered as group homology with coefficients in group ring where group acts by conjugation we have

\subsubsection{Corollary ( Hochshild Homology of Localization of Group Rings):\label{homiscool}}
\emph{Let $S$ be an Ore Set in ring $R$, a $k$-algebra. Then
$$HH_*((kG_S))\cong H_*^{(kG)_S}(\overline{(kG)_S}) \cong [HH_*(kG)]_S$$ }

\subsubsection*{Corollary (Hochshild Homology of Localisation of Completed Group Rings):}

$$\varprojlim_{U\leq G} [HH_*(k[G/U])]_S \cong \varprojlim_{U\leq G}HH_*((k[G/U])_S)$$

Thus understanding whether inverse limits commute with localization by $S$ will lead to a connection between the object of interest, $HH_*((\Lambda_G)_S)$, above, and $(\varprojlim HH_* (k[G/U]))_S = [HH(\Lambda_G)]_S$ which is easy to calculate.

\subsubsection*{Localization and taking Inverse Limits does not Commute}
 However, Universal Property of Inverse Limit gives a map $ S^{-1}\varprojlim R_U \rightarrow \varprojlim S^{-1} R_U$ (not necessarily onto).

Take $S=\{1,p, \dots\}$, and $R= \ZP = \varprojlim_n \ZP / p^n\ZP = \varprojlim_n \Z / p^i\Z$ (more generally, for $F$ a Weierstrass polynomial, the injection $\mathfrak O [T]\hookrightarrow \ \mathfrak O [[T]]$ induces an isomorphism $\mathfrak O [T] / F \mathfrak O [T]  \rightarrow \mathfrak O [[T]] / F \mathfrak O [[T]]$, see \cite{N} 5.3.3). 


Consider image of $m\in  \Z / p^i\Z$ in  $S^{-1}  \Z / p^i\Z$. Certainly $p^i m = 0$, but $\frac{1}{p^i}\in S^{-1}$ so $\frac{1}{p^i} (p^i m) =0$ hence $\overline m = 0 \in S^{-1} M \,\forall \,m \in  \Z / p^i\Z$. This gives the result since
$$\varprojlim_i [ S^{-1} (\Z / p^i\Z)] = 0 \neq \QP = S^{-1} \varprojlim_i  (\Z / p^i\Z)$$

So we see that the Torsion elements map to zero, and non-torsion elements are not in kernel of Localization.

\subsubsection{Proposition (Hochschild Homology of Localised Iwasawa Algebras of pro-$p$ Groups):}
\emph{For $G$ a pro-$p$ group, $HH_1((\Lambda_G)_T)$ is trivial, and thus by commuting diagram of \ref{commutingdiagramcool}, $HH_1((\Lambda_G)_{S^*})$ is trivial.}

\subsubsection*{Proof}
\begin{eqnarray}
\nonumber HH_1[(\Lambda_G)_T]   &=& HH_1[ (\varprojlim \ZP [G/U])_T ] \\
\nonumber \text{Localization commutes with the inverse limit} && \text{(no torsion in group algebra)}\\
\nonumber     						&=& HH_1[\varprojlim ( \ZP [G/U])_T] \\
\nonumber     						&=& \varprojlim [HH_1( \ZP [G/U])_T] \text{ see } \ref{CHH} \\
\nonumber     						&=& \varprojlim [HH_1( \ZP [G/U])]_T \text{ see } \ref{homiscool}\\
\nonumber						&=& \varprojlim [\bigoplus_{\gamma\in\text{ccl}(G/U)}[Z(\gamma)]^{\text{ab}}]_T
\end{eqnarray}

$G$ is pro-$p$ implies $G/U$ is a $p$-group, all elements have order some power of $p$, and thus the subgroups $Z(\gamma)$ also have this property, meaning they are $T=\{1,p,p^2,\dots \}$-torsion and are killed when we localize - $HH_1[(\Lambda_G)_T]  = \varprojlim \bigoplus_\gamma \overline{id} = \overline {id}$ in $\bigoplus_{\gamma\in\text{ccl}(G/U)}[Z(\gamma)]^{\text{ab}}$.






%%%%%%%%%%%%%%%%%%%%%%%%%%%%%%%%%%%%%%%%%%%%%%%%%%%%%%%%%%%%
%%%%%%%%%%%%%%%%%%%%%%%%%%%%%%%%%%%%%%%%%%%%%%%%%%%%%%%%%%%%
%%%%%%%%%%%%%%%%%%%%%%%%%%%%%%%%%%%%%%%%%%%%%%%%%%%%%%%%%%%%
%%%%%%%%%%%%%%%%%%%%%%%%%%%%%%%%%%%%%%%%%%%%%%%%%%%%%%%%%%%%
%%%%%%%%%%%%%%%%%%%%%%%%%%%%%%%%%%%%%%%%%%%%%%%%%%%%%%%%%%%%
%%%%%%%%%%%%%%%%%%%%%%%%%%%%%%%%%%%%%%%%%%%%%%%%%%%%%%%%%%%%

