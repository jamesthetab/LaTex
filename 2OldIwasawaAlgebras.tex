%%%%%%%%%%%%%%%%%%%%%%%%%%%%%%%%%%%
%%%%%%%%%%%%%%%%%%%%%%%%%%%%%%%%%%%
Iwasawa Algebras are a kind of Completed Group Rings. Let $G$ be a finitely generated pro-p group. For $M\leq N$ open normal subgroups of G the natural map $$G/M\rightarrow G/N,$$
induces an  epimorphism of group algebras
$$\ZP[G/M]\rightarrow \ZP[G/N],$$
giving an inverse system of $\ZP$-algebras, whose inverse limit is denoted
$$\ZP[[G]] = \underleftarrow{\text{Lim}}_{N\vartriangleleft_o G}(\ZP[G/N])$$
and is called the Iwasawa Algebra of $G$.  This can also be defined as a completion of the group ring and is known as the completed group algebra of $G$.



Writing $R=\ZP[G],\, G_k= P_k(G)$ (lower p-series) take $I_k$ to be the kernel of the natural epimorphism:
$$I_k = (G_k - 1)R = \text{ker}(R\rightarrow \ZP[G/G_k]),$$

I now recall a result giving the tight correspondence between the algebraic and analytic structure of pro-p groups:

\subsubsection{Lemma (Proposition 1.16 \cite{DDMS})}
\emph{\,\,\,\,\,\,\,\,\,\,Let G be a finitely generated pro-p group, then $P_i(G)$ is open in $G$ for each $i$, and the set $\{P_i(g)\vert i\geq 1\}$ is a base for neighbourhoods of $1$ in $G$.}
%%%%%%%%%%%%%%%%%%%%%%%%%%%%%%%%%%%
\bigskip

Hence the family $(G_k)$ is cofinal with the family of all open normal subgroups of G, and we may identify $\ZP[[G]]$ with
$$\underleftarrow{\text{Lim}}_{k\in\N} (\Z/p^k\Z)[G/G_k] \cong \underleftarrow{\text{Lim}}(R/(I_k+p^kR))$$
There is a chain of ideals more suited to defining a norm than $(I_k)$, the cofinal series consisting of powers of the following ideal:
$$J = I_1 +pR = \text{Ker}(R\rightarrow \FP)$$
Cofinality follows from 
\subsubsection{Lemma (Lemma 7.1  \cite{DDMS})\label{cofinal}}
\emph{Let $k\geq 1$. Then
\begin{itemize}
\item $J_k\supset I_k +p^k R;$
\item for each $j\geq 1,\, I_k +p^jR\supset J^{m(k,j)}$ where $m(k,j) = j \vert G/G_k\vert$.
\end{itemize}}
\bigskip
These definitions give $\bigcap_{l=1}^\infty J^l = 0$, and $J^i J^j = J^{i+j}$ and we may now define a norm with respect to this chain of ideals:

\subsubsection{Definition}
\emph{The norm $\parallel - \parallel$ on $\ZP[G]$ is defined by
\begin{eqnarray}
\nonumber \parallel c\parallel &=& p^{-k} \text{    if }c\in J^k-J^{k+1}\\
\nonumber \parallel 0\parallel &=& 0
\end{eqnarray}}

Writing $\hat{R}$ for the completion of the group algebra with respect this norm, $\hat{R}$ may be thought of as $\underleftarrow {\text{Lim}}_k R/J^k$, and Lemma \ref{cofinal} gives
$$\hat{R} \cong \underleftarrow{\text{Lim}}_k (\Z/ P^k \Z) [G/G_k] \cong \ZP[[G]],$$
This justifies the name "Completed Group Algebra of $G$" for the Iwasawa Algebra, $\ZP[[G]]$.

Observe, since $g-1\in J = \text{ker: }\ZP [G]\rightarrow \FP$, each element $g\in G$ satisfies $\parallel g - 1\parallel \leq p^{-1}$.
%%%%%%%%%%%%%%%%%%%%%%%%%%%%%
%%%%%%%%%%%%%%%%%%%%%%%%%%%%%

\subsection{Modules over Iwasawa Algebras}
In this section I introduce the topological Nakayama lemma to deduce properties of Iwasawa modules. The structural results we have for projective modules will be important when it comes to constructing projective resolutions in order to calculate Hochschild homology.

Let $\mathfrak m = p \ZP$.  Recall that $\ZP / \mathfrak m$ is a finite field (of characteristic $p$). Therefore $\ZP [[G]]$ is an inverse limit of finite discrete (hence Artinian) rings, and denoting the kernel of the natural epimorphism $\ZP[[G]]\twoheadrightarrow \ZP[[G/U]]$ by $I(U)$, we have that the ideals
$$\mathfrak m^n \ZP [[G]] +I(U),\,\, n\in\mathbb N, \, U\subset G\text{ open normal}$$
are a fundamental system of neighbourhoods of $0\in \ZP [[G]]$. 

We denote by $Rad_G\subset \ZP[[G]]$ the radical of $\ZP [[G]]$, i.e. the inverse limit of the radicals of $\ZP/ \mathfrak m ^n [G/U]$ - the intersection of all left maximal ideals. Then $Rad_G$ is a closed two-sided ideal which is the intersection of all open left maximal ideals. The powers $(Rad_G)^n,  \, n\geq 1$, define a topology on $\ZP[[G]]$, which we call the \textbf{R-topology}.

\subsubsection{Proposition (Different Topologies \& Properties of Iwasawa Algebras - \cite{N}, 5.2.16):\label{localgroup}}
\emph{\begin{enumerate}
\item The $R$-topology is finer than the canonical topology on $\ZP [[G]]$, in particular it is Hausdorff.
\item The following assertions are equivalent
\begin{itemize}
\item The $R$-topology coincides with the canonical topology on $\ZP[[G]]$
\item $Rad_G\subset \ZP[[G]]$ is open.
\item $\ZP[[G]]$ is a semi-local ring.
\item $(G:G_p)<\infty$, where $G_p$ is a $p$-Sylow subgroup in $G$.
\end{itemize}
\item $\ZP[[G]]$ is a local ring if and only if $G$ is a pro-$P$-group. In this case, the maximal ideal of $\ZP[[G]]$ is equal to $\mathfrak m \ZP[[G]] +I_G$, the kernel of the composition of augmentation with reduction modulo $p$:\\ $\ZP[[G]]\xrightarrow{g=1} \ZP\xrightarrow {\text{mod } p} \FP$.
\item More generally, for $\mathfrak O$ a commutative ring, complete in its $\mathfrak m$-adic topology, for $\mathfrak m$ a maximal ideal, and $\mathfrak k = \mathfrak O / \mathfrak m$ a finite field of characteristic $p$ - $\mathfrak O$ compact. $\mathfrak O [[G]]$ is a semi-local ring if and only if $G$ is pro-p. The global dimension, $\text{gl} \mathfrak O[[G]]$ equals $\text{cd}_pG+\text{gl}\mathfrak O$, where $\text{cd}_p$ denotes $p$-cohomological dimension. By a result of Serre, $\text{cd}_p G$ is finite if and only if $G$ does not contain an element of order $p$.
\item Assume that $\mathfrak O$ is finitely generated as a $\ZP$-module and let $G$ be a compact $p$-adic Lie group. Then $\mathfrak O [[G]]$ is Noetherian.
\end{enumerate}}

Now assume that $M$ is a compact $\ZP[[G]]$-module. Then in addition to the given topology there are two other topologies on $M$:

\begin{enumerate}
\item The topology given by the sequence of submodules $\{\mathfrak m ^n M +I(U)M\}_{n,U}$ where $n\in\mathbb N$, and $U$ runs through the open normaal subgroups of $G$. We call this topology the \textbf{$(\mathfrak m, I)$-topology}.
\item The $R$-topology, which is given by the sequence of submodules $\{ (Rad_G)^n M\}_{n\in \mathbb N}$.
\end{enumerate}

From above, the $R$-topology is obviously finer than the $(\mathfrak m, I)$-topology, and both coincide if $G$ is a pro-$P$-group.

\subsubsection{Proposition}
\emph{\begin{enumerate}
\item The $(\mathfrak m, I)$-topology is finer than the original toplogy on $M$. In particular, the $(\mathfrak m, I)$- and the $R$-topology are Hausdorff.
\item If $M$ is finitely generated, then the $(\mathfrak m, I)$-topology coincides with the original topology on $M$.
\end{enumerate}}

Thus the topologies are equivalent in the case of interest here, of the False Tate Curve.



\subsubsection{Corollary (Nakayama's Lemma):}
\emph{\begin{itemize}
\item If $M\in \mathfrak C$ and $Rad_G M = M$, then $M=0$.
\item Assume that $G$ is a pro-$p$-group, hence \textbf{$\ZP[[G]]$ is local with maximal ideal $\mathit M$}. Then $M$ is generated by $x_1, \dots , x_r$ if and only if $x_i +\mathfrak M M, \, i=1,\dots , r$, generate $M/\mathfrak M M$ as an $\ZP[[G]] / \mathfrak M$ - vector space.
\end{itemize}}


\subsubsection{Corollary}
\emph{Let $G$ be a pro-$p$-group and let $P\in \mathfrak C$ be finitely generated. Then $P$ is a free $\ZP[[G]]$-module if and only if $P$ is projective.}

This will be an essential tool when we manipulate projective resolutions to calculate Hochschild Homologies of the Iwawawa algebra in \ref{CFTC}.



%%%%%%%%%%%%%%%%%%%%%%%%%%%%%%%%%%%
\subsection{Iwasawa Algebras formed by Power Series Expansion}
%%%%%%%%%%%%%%%%%%%%%%%%%%%%%%%%%%%
By refining the filtration $\{J^i \}$, \cite{DDMS} produce a complete, seperated and exhaustive (increasing) filtration $F_\bullet \Lambda$ of $\mathfrak O [[G]]$ (for $G$ $p$-valuable) such that the induced grading,
\begin{eqnarray}
\nonumber \text{gr }\mathfrak O[[G]] 	&=& \mathfrak k [X_0, X_1, \dots, X_d] \text{  if }\mathfrak O \text{ is a DVR}\\
\nonumber 					&=& \mathfrak k [X_1, \dots, X_d] \text{  if }\mathfrak O \text{ is a finite field}
\end{eqnarray}

This is key to producing a criterian for any Linear Combination of group elements to be in the Iwasawa Algebra in terms of convergence of a power series (see \cite{V} 2.1), and I then quote the norm of such an element as a function of power series coefficients. I recall these formula from \cite{DDMS} section 7:

\subsubsection{Notation (Generating Set):}
\emph{Let $\{a_1, \dots a_d\}$ be a topological generating set for $G$, where the dimension is $d=d(G)$. Define,
$$b_i = a_i-1\text{  for  } i=1, \dots , d$$
For $\alpha = (\alpha_1\dots, \alpha_d)\in \N^d$ and $\mathbf v = (v_1,\dots , v_d)$ any d-tuple, denote,
\begin{eqnarray}
\nonumber <\alpha> = \alpha_1+\dots \alpha_d\\
\nonumber \mathbf v^\alpha = v_1^{\alpha_1}\dots v_d^{\alpha_d}
\end{eqnarray}}

The structure of the Iwasawa Algebra is given by

\subsubsection{Theorem (Power Series Expansion of Iwasawa Algebra):\label{PS}}
\emph{\begin{enumerate}
\item If $G$ is powerful then each element of $\ZP [[G]]$ is equal to the sum of a convergent series
$$\sum_{\alpha\in \N^d} \lambda_\alpha \mathbf b^\alpha$$
with $\lambda_\alpha\in\ZP$ for each $\alpha$.
\item If $G$ is uniform then the above series is uniquely determined by its sum
\end{enumerate}}

Given such a power series expansion we can calculate its norm just by looking at coefficients:
\subsubsection{Theorem (Norm of Power Series):}
\emph{Assume that $G$ is uniform.If $c=\sum_{\alpha\in\N^d} \lambda_\alpha \mathbf b^\alpha \in \ZP [G]$, where $\lambda_\alpha\in \ZP$ for each $\alpha$, then
$$\parallel c \parallel = \text{sup }_{\alpha\in \N^d}p^{-<\alpha>}|\lambda_\alpha |.$$}

\subsubsection{Claim (Topological Generation):}
\emph{$$a_1^{\lambda_1}\dots a_d^{\lambda_d} = \sum_{\alpha\in\N^d}
\left(\begin{array}{c}\lambda_1\\ \alpha_1\\ \end{array}\right)  \dots \left(\begin{array}{c}\lambda_d\\ \alpha_d\\ \end{array}\right) \mathbf b^{\alpha}$$ where, for $\lambda\in\ZP$, and $1\leq r\in\N$, 
$$\left( \begin{array}{c}\lambda\\ r\\ \end{array}\right)  = \frac{\lambda (\lambda - 1)\dots (\lambda-r+1)}{r!} \text{ belongs to }\ZP$$}

Thus any element in the group ring $\ZP [G]$ converges to a power series, of form \ref{PS}, and we verify it lies in the Iwasawa Algebra $\ZP[[G]]$.

%%%%%%%%%%%%%%%%%%%%%%%%%%%%%%%%%%%
\subsection{False Tate Curve\label{false tate}}
%%%%%%%%%%%%%%%%%%%%%%%%%%%%%%%%%%%
I now specialise to a specific example arising from number theory and show how this gives a counter example to any attempt to give a general construct for the completion of the Hochschild Homology of a Group Algbera as the Hochschild Homology of an Iwasawa Algebra.

For $\alpha\in k^*$ a unit of $k$ which is not a root of unity, we define a Galois extension of a number field $k$ which arises from the $p$-adic Galois representation coming from the Tate Elliptic Curve. Assuming $k$ already contains the $p$-th roots of unity, $\mu_p$, denote it's cyclotomic $\ZP$-extension, $k_{cycl} = k(\mu_{p^\infty})$, and $k_\infty$ as the field arising from adjoining the $p$-power roots of $\alpha$, and thus all $p$-power roots of unity, $k_\infty = k(\mu_{p^\infty} , p^{p^{-n}})$. By Kummer Theory, the Galois Group, $G= G(k_\infty / k_{cycl})$ is isomorphic to the (non-abelian) semi-direct product of 2 copies $F,H$ of $(\ZP,+)$. 

The arithmetic of $k_\infty / k$ depends on the choice of $\alpha$, \cite{V} describes the case of $\alpha = p$:

\subsubsection{Lemma (Ramification Properties):}
\emph{Let $k=\Q (\mu_p)$ and $k_\infty = k (\mu_{p^\infty}, p^{p^{-\infty}})$. Then the extension $k_\infty / k$
\begin{enumerate}
\item totally ramifies at the unique place over $p$, in particular there is just one prime of $k_\infty$ above $p$,
\item unramified outside $p$.
\end{enumerate}}

I explicitly construct the semi-direct product, and simplify the power series expansion for elements in $\ZP [[G]]$.

Let $F,H = (\ZP,+)$, where ,
\begin{eqnarray}
\nonumber \rho:H  &\rightarrow& \text{Aut } G = (\ZP^*,*)\\
\nonumber           1 &\rightarrow& 1+p\\
\nonumber           n &\rightarrow& (1+p)^n   
\end{eqnarray}

\subsubsection{Notation (FTC):\label{FTC}}
\emph{ \begin{eqnarray}
\nonumber G &=& F  \rtimes H\\
\nonumber     &=& \{(f,h)| (f,h).(f',h') = (f+\rho(h).f', h+h')\}
\end{eqnarray}
Thus $(f,h)^{-1} = (-\rho(-h).f, -h)$, and we denote the \textbf{topological generators of $G$ as $f=(1,0)$ and $h=(0,1)$}}

\bigskip

We can already see that the group is powerful - that $G/G^p$ is abelian, or equivalently that $[G,G]\subset G^p$. Observe $(p.n,0)=(n,0)^p\in G^p\,\,\forall n\in \ZP$. Thus, a general element in the commutator, (for details of calculation see calculation of conjugacy classes below),
\begin{eqnarray}
\nonumber (a,b)^{(g,h)} (a,b)^{-1} 	&=& (\rho(-h) (a+(\rho(b)-1)g), b)(-\rho(-b).a,-b)\\
\nonumber  					&=& ((\rho(-h)-1)a+\rho(-h)(\rho(b)-1)g,0)\\
\nonumber \text{Since }p|( \rho(n)-1) \,\forall n\in \ZP\\
\nonumber 					&=& (m.p,0), \text{ some }m\in\ZP\\
\nonumber					&\in& G^p
\end{eqnarray}

Hence G is powerful - a non-commutative pro-finite group with associated Lie Algebra $F \text{ X } H$, the commutative Cartesian product. This follows from the good behaviour under taking powers (if $(g,h)^n = (a,b)$, then $h=b/n$, and $g=a.\frac{1-(1+p)^{b/n}}{1-(1+p)^b}$).

Let $\mathbf b_1 = f - 1\in \ZP[G]$, and $\mathbf b_2 = g - 1\in \ZP[G]$, then \ref{PS} gives 

\subsubsection{Theorem (Power Series Expansion of FTC):\label{PSFTC}}
\emph{For $G = F  \rtimes H$ each element of $\ZP [G]$ is equal to the sum of a uniquely determined convergent series
$$\sum_{(\alpha_1,\alpha_2)\in \N^2} \lambda_{(\alpha_1,\alpha_2)} \mathbf b_1^{\alpha_1} \mathbf b_2 ^{\alpha_2}$$
with $\lambda_{(\alpha_1,\alpha_2)} \in\ZP$. }

\bigskip

We will now see how this is a special case of a well understood theory:
\subsubsection{Definition (Skew Power Series Rings):}
Let $R$ be a ring, $\sigma:R\rightarrow R$ a ring endomorphism and $\delta:R\rightarrow R$ a $\sigma$-derivation of $R$ - a group homomorphism satisfying
$$\delta(rs) = \delta(r) s + \sigma(r) \delta (s) \text{  for all}r,s\in R.$$
The the (formal) skew power series ring
$$R[[X;\sigma,\delta]]$$
is defined to be the ring whose underlying set consists of the usual power series $\sum_{n=0}^{\infty} r_n X^n$, with $r_n \in R$. Where the multiplication of two such power series is defined by the formula
$$Xr = \sigma(r) X +\delta(r)$$
Hence, all products are convergent, and if $R$ is complete with respect to the $I$-adic topology, $I$ some $\sigma$-invariant ideal such that $$\delta(R)\subset I,\,\,\,\delta(I)\subset I^2$$
then multiplication is uniquely defined.


\subsubsection{Proposition (FTC as Skew Power Series):\label{FTCPS}}
\emph{Writing, $X=\mathbf b_1$, and $Y = \mathbf b_2$, we have $$\ZP [[G]] \cong \ZP [[X]]  [[Y; \sigma, \delta]]$$ the skew power series ring, where $R = \ZP[[X]]$, and for $\epsilon = \rho(h)$ the ring automorphism $\sigma$ is induced by $X \rightarrow (X+1)^\epsilon$, and $\delta = \sigma - \text{ id}$.}

\subsubsection{Corollary (FTC as commutator):} 

\subsubsection*{Proof}
$R$ is complete with respect to the topology induced by it's maximal ideal $\mathfrak m$, generated by $X$ and $p$. $\sigma$ is just choosing another generator of $H$, so $\sigma(\mathfrak m)= \mathfrak m$. Convergence follows if we can verify $\delta(\mathfrak m^k)\subset \mathfrak m^{k+1}$ for $k=0,1$. Since $\delta$ and $\sigma$ are $\ZP (\FP)$-linear, we need only show for $r=X$. But, following \cite{V},
\begin{eqnarray}
\nonumber \delta X &=& \sigma(X) - X\\
\nonumber               &=& (X+1)^\epsilon - 1- X\\
\nonumber    	       &=& \sum_{i\geq 1} \left( \begin{array}{c}\epsilon \\ i\\ \end{array}\right) X^i - X\\
\nonumber 	       &=& (\epsilon - 1) X^k +\text{ terms of higher degrees}
\end{eqnarray}
Since $p| (\epsilon - 1)$ it follows $\delta(X)\subset \mathfrak m ^2$, as required.

To complete the proof we must verify that $\sigma$ and $\delta$ describe how coefficients fail to commute with variables, that $YX = \sigma(X) Y + \delta(X)$, by linearity it is sufficient to prove:
$$YX = \sigma(X) Y + \delta (X)$$

\begin{eqnarray}
\nonumber \text{LHS} 	&=& ((0,1)-1)((1,0)-1)\\
\nonumber			&=& (0+(1+p).1,1) - (0,1) - (1,0) + (0,1)\\
\nonumber 			&=& (1+p,1) - (1,0) - (0,1) + (0,0)\\
\nonumber \text{RHS}	&=& ((1+p,0)-1)((0,1)-1)+(1+p,0)-(1,0)\\
\nonumber			&=& (1+p+1.0,1) - (0,1) - (1+p,0) +1+ (1+p,0) - (1,0)\\
\nonumber			&=& \text{LHS, as required}
\end{eqnarray}

I now investigate conjugation in this semi-direct product, let $(a,b),(g,h)\in G$ then
\begin{eqnarray}
\nonumber (a,b)^{(g,h)} 	&=& (g,h)^{-1} (a,b) (g,h)\\
\nonumber 			&=& (\rho(-h)g, -h) (a,b)(g,h)\\
\nonumber 			&=& (\rho(-h)g, -h) (a+\rho)b) g, b+h)\\
\nonumber			&=& (\rho(-h)g+\rho(-h).(a+\rho(b)g),b)\\
\nonumber 			&=& (\rho(-h) (a+(\rho(b)-1)g), b)
\end{eqnarray}

Thus $(1,0) = f \sim (\ZP^*,0)$, since $\rho = (h\rightarrow \rho (-h)(1+0)): \ZP \twoheadrightarrow \ZP^*$ . And conjugating by the set $(0,\N)\subset G$, we see that \textbf{the infinte sequence $f^1, f^{(1+p)}, f^{(1+p)^2}, f^{(1+p)^3}, \dots$ all lie in the same conjugacy class.}

\subsubsection{FTC as Skew Power Series Recap}

\begin{eqnarray}
\mathfrak O_K[[L]] 	&=& \mathfrak O_K [[\epsilon - 1]] <<\ZP>>\\
				&=& \mathfrak O_K [[X]] <<Y, \sigma>>
				\end{eqnarray}
Where $Y$ corresponds to $\gamma -1$ for $\gamma$ a topological generator of $\ZP$.

To understand multiplication of $\sum r_n X^n$ it is sufficient to understand

\begin{eqnarray}
Yr 	&=& \sigma(r). Y + \delta(r) = \sigma(r).Y+\sigma(r) - r\text{ so,}\\
(1+Y). r	&=&\sigma(r). (1+Y)
\end{eqnarray}
where $\sigma: (X+1)\rightarrow (X+1)^{\chi(\gamma^{-1})} - 1 = (X+1)^{1+p}\text{ WLOG}$.

i.e
$$\Lambda(G)\cong  \mathfrak O_K [[X]] <<Y, \sigma>> \cong\{ \sum r_n Y^n \,|\, (1+Y). r	=\sigma(r). (1+Y) \text{ and } \sigma: (X+1)= (X+1)^{1+p} \}$$






\subsubsection{Proposition (Construction of Completion):\label{COC}}

%%%%%%%%%
We recall the construction of a completion of Hochschild homology for Novikov rings. Describing elements in $\ZP[[G]]$ as formal linear combinations of elements, an n-chain used to calculate $HH_n(\ZP [[G]])$ has form:

$$\sum_{g_1\in G}n_{g_1} g_{n+1} \otimes\dots\otimes \sum_{g_{n+1}\in G}n_{g_{n+1}}g_{n+1}$$

For finite elements, in $\ZP [G]$, this may be simplified to an element of $C_n({\Z G},{\Z G})$, a finite sum:

$$(\star)\,\,\,\,\, \sum_{g_1,\dots, g_{n+1}\in G} n_{g_1}\dots n_{g_{n+1}} g_1\otimes \dots \otimes g_{n+1}$$

Although a general element taken from $\widehat{\Z G}_\chi$ would be an infinite sum of tensors which does not give a well defined element of $C_n({\ZP G},{\ZP G})$, the homology of the iwasawa algebra lies in a completion of $HH_*(\ZP G)$ iff the following holds:

\emph{"Given a conjugacy class $\gamma\in \Gamma$ there are only finitely many nonzero summands in $(\star)$ with marker in $\gamma$: such that $g_1\dots g_{n+1}\in \gamma$."}

Notice that since $\ZP[[G]]$ is a ring, if this property holds for calculation of the zeroth homology - in other words that an element of $\lambda\in \ZP[[G]]$ can only finite support in any one conjugacy class, then the same holds for the formal product $\lambda_1...\lambda_{n+1}$ since $\lambda_1..\lambda_{n+1}\in \ZP[[G]]$.

\subsubsection{Proposition (Iwasawa Completion does not Decompose Over Conjugacy Classes):\label{no compose over conj}}
\emph{The Hochschild homology of a general Iwasawa algebra as a completion of the Hochschild homology of the group ring is not contained in the direct product over conjugacy classes ($ HH (        \widehat{(\Z G)}) \subset \prod_{\gamma , \text{conj. class}} HH (\Z G)_\gamma$). Indeed, for $G$ the non-commutative semi-direct product of two copies of $\ZP$ arising from the False Tate Curve, there exists an element of the Iwasawa Algebra with infinte support in a single conjugacy class.}

\bigskip

From above, this is equivalent to demonstrating an element in an Iwasawa Algebra $\lambda\in \ZP[[G]]$ with a conjugacy class having infinite support.

Consider the False Tate Curve, $G= F\rtimes H$, by \ref{FTCPS}, for $X=f-1$, 
\begin{eqnarray}
\nonumber \lambda &=& \sum_{n\geq 0} \lambda_n X^{{(1+p)}^n}\\
\nonumber \text{where, } \lambda_n &=& p^n\\
\nonumber \text{thus, } \lambda &=&  1.(f-1)+ p.(f-1)^{(1+p)}+p^2.(f-1)^{(1+p)^2}+\dots
\end{eqnarray}

Then the series is convergent, thus $\lambda\in \ZP[[G]]$. Where
\begin{eqnarray}
\nonumber  \text{coeff. of } f^1 &=& 1+p.\mu_{1,1} +p^2.\mu_{1,2} + p^3.\mu_{1,3} +...\in \ZP^*\\
\nonumber  \text{coeff. of } f^{(1+p)} &=& p+p^2.\mu_{2,1} +p^3.\mu_{2,2} + p^4.\mu_{2,3} +...\in p\ZP^*\\
\nonumber  \text{coeff. of } f^ {(1+p)^2}&=& p^2+p^3.\mu_{3,1} +p^4.\mu_{3,2} + p^5.\mu_{3,3} +...\in p^2\ZP^*\\
\nonumber 					&\vdots&
\end{eqnarray}
Where $\mu_{i,j}\in \ZP$, thus coeffs. of $g^1, g^{(1+p)^2}. g^{(1+p)^3},\dots$ are all nonzero (contained in $p^n   \ZP^*$), and hence \textbf {there exists an element of the Iwasawa Algebra with infinte support in a conjugacy class}, as a completion over conjugacy classes described as a direct sum requires completion of the group algebra itself in each coordinate, and does not lie in the direct product over conjugacy classes of the usual group algebra.

However, when $G$ is infinte it is no longer true that $HH_1(G,k) = G^{ab}$, and so although we have shown,  $ HH (        \widehat{(\Z G)}) \subset \prod_{\gamma , \text{conj. class}} HH (\Z G)_\gamma$, does does not imply $HH_1 \widehat{(\Z G)}) \neq \prod_{\gamma , \text{conj. class}} {Z(\gamma)}^{ab}$. This question will be resolved below.

