%%%%%%%%%%%%%%%%%%%%%%%%%%%%%%%%%%%%
\section{Idempotents}
%%%%%%%%%%%%%%%%%%%%%%%%%%%%%%%%%%%%


\begin{theorem}(see \cite{fulton}, 2.30, \textbf{Characters form an Orthonormal
Basis})

The number of irreducible representations of $G$ is equal to the number of
conjugacy classes of $G$. Equivalently, their characters $\{ {\chi}_V \}$
form an orthonormal basis for $\C _{\text{class}} (G)$.

\end{theorem}

Any irreducible representation is isomorphic to a (minimal) left ideal in
$\C G$. These left ideals are generated by idempotents. In fact, we can
interpret the projection formulas for character table in the language of
the group algebra: the elements

$$ dim \, W . \frac{1}{|G|}\sum_{g \in G} \overline{\chi_W (g)} . e_g \in
\C G$$

are idempotents in the group algebra corresponding to the direct sum
decomposition of the complex group algebra,

$$\C G \cong \bigoplus \text{End}(W_i).$$





%%%%%%%%%%%%%%%%%%%%%%%%%%%%%%%%%%%%
\section{Incidence Matrices and Class Algebra Constants}
%%%%%%%%%%%%%%%%%%%%%%%%%%%%%%%%%%%%

\section{Patterns in Centralisers and growth of Conjugacy Classes}

Using the key idea that
$$\phi(Z(xG_n)) = Z(xG_{n+1}) \cap G_2$$
we defined $b_n$ and $c_n$ as multipliers as constants giving valency and multiplier in the tree which eventually stabilises.

Chris mentioned the work of Andre Jaikin-Zaparion who looks at the orbits on the duals of Lie Algebras.

This leads into the world of $p$-adic valuation. This raises another interesting question - since the valuation given to $p$ is usually $p ? 1$ it is not clear that it should always have the value $1$ in Number Theory.

By thinking about multiplication in the centre we can think go the incidence matrices as expressing (class sum)*(class sum) as a linear combination of (class sums).

Over Easter we expressed the coefficients in the Linear Comb in terms of Double Cosets, and later in terms of idempotents.


%%%%%%%%%%%%%%%%%%%%%%%%%%%%%%%%%%%%
\section{Kirillov Orbit Method}
%%%%%%%%%%%%%%%%%%%%%%%%%%%%%%%%%%%%

\section{Sizes of Conjugacy Classes \& Degrees of Representations over $\C$}

In the process of calculating the index of the span of idempotents inside the span 
of rows of the character table we need to consider an index:

$$\prod_i \frac{|G|}{\chi_i (1) \sqrt{|Z(C_i)|}}$$

Where $\chi_i$ run through the irreducible representations, and $C_i$ run through the conjugacy classes of $G$. Modulo the size of the group, $|G|$, this is equivalent to studying the ratio, $r$:

$$\prod_i \frac{|ccl(C_i)|}{\chi_i (1)^2}$$

This seems like a natural object to study given the tight relationship between the additive version of this comparison: $\sum_i |ccl(C_i)| = |G| = \sum_i \chi_i (1)^2$. Thus,

$$\sum_i |ccl(C_i)| - \chi_i (1)^2 = 0.$$

As an example we can look at $D_6$, the third Dihedral group, symmetries of an equilateral triangle: the conjugacy classes are of orders $\{1,2,3\}$ and the irreducible representations have degrees $\{1,1,2\}$.

$$r= \frac{1.2.3}{1.1.2} = 3 >1$$

We can also view this ratio as comparing the product along the top row of the normalised character table (multiply each column by the square root of the size of the conjugacy class it represents)  with the product along the left hand column.

We made a couple of conjectures which have both found to be false by counter-example:

\begin{enumerate}
\item \textbf{$G$ is Abelain is equivalent to $r=1$}

The $73^{rd}$ group of order $64$ in Magma is non-abelian, but enjoys the property that the set of the sizes of conjugacy classes is equivalent to the set of squares of degrees of the irreducible representations (counting multiplicities), so in particular $r=1$. In fact, it has $8$ representations of degree $1$ and $14$ representations of degree $2$; and has $8$ elements in its centre, and $14$ conjugacy classes of order $4$.

\item \textbf{The ratio $r\geq 1$}

The group $SL(2,5)$, the special linear group of degree $2$: $2$-by-$2$ matrices of determinant now over the field of fine elements provides a counter example. 

It has conjugacy classes of sizes $\{ 1,1,12,12,12,12,20,20,30\}$ and irreducible representations over $\C$ of degrees $\{1,2,2,3,3,4,4,5,6 \}$.

Where, $\prod_i |ccl(C_i)| = 248,832,000$, while $\prod_i \chi_i(1)^2 = 298,598,400$, giving $r<1$.
\end{enumerate}

A remaining question about this ratio is what conditions are required to give $r=1$?




%%%%%%%%%%%%%%%%%%%%%%%%%%%%%%%%%%%%
\section{Two types of Commutator and the Group Algebra Hierarchy}
%%%%%%%%%%%%%%%%%%%%%%%%%%%%%%%%%%%%

$[a,b]_{\text{Group}}$ and $[a,b]_{\text{Lie}}$
Gives rise to $2$ invariant objects: one where conjugate differences drops the level: \\ $g\circ_{G} \theta (x) = \theta (g^{-1} x g) - \theta (x)$, and the other where comparing left and right multiplication drops the level  \\$g\circ_L \theta (x) = \theta (x g) - \theta (gx)$, using that $$g\circ_L \theta (x) \cong g\circ_G \theta (gx).$$

Understanding the Lie difference is the same as understanding the Group difference and left multiplication. As these are defined up to conjugacy class we need to understand multiplication of conjugacy classes which leads to the world of Incidence Matrices. 

Firstly, we consider a hierarchy of functions which behaves well with respect to the group difference. In particular I give a filtration, 

$$\Sigma_0 \subset \Sigma_1 \subset \Sigma_2 \subset \Sigma_3 \subset \dots$$ 

With the property that for any $g\in G$, the conjugacy difference $\theta_g$ (where $\theta_g f(x) \equiv f (x^g) - f(x)$) drops at least one level in the hierarchy:

$$\theta_g \Sigma_n \subset \Sigma_{n+1}$$

Moreover, there is also nice behaviour with respect to the $p$-valuation:

$$p. \Sigma_n \subset \Sigma_{n+1}$$

\begin{definition}(Function Hierarchy)
\begin{itemize}
\item Let $\Sigma_0$ represent the function which is constantly $0$ across the whole of $G$.

\item Let $\Sigma_1$ represent the $p$-torsion class functions (taking values in $\frac1p \ZP / \ZP$) which is constantly $0$ across the whole of $G$.

\item Let $\Sigma_2$ represent the $p^2$-torsion "almost class functions" (taking values in $\frac1{p^2} \ZP / \ZP$) with the property that the $p$-torsion part is the bit that fails to be a class function, but which becomes a class function when we take any conjugacy difference (or multiply by $p$ to kill off the $p$-torsion).

\item In general $\Sigma_n$ is a collection of $p^n$-torsion "almost class functions" (taking values in $\frac1{p^n} \ZP / \ZP$) with the property that the $p$-torsion is annihilated by taking conjugacy differences $n$ times, the $p^2$ not $p$-torsion is annihilated by taking conjugacy differences $n-1$ times, and so on. Finally, the $p^n$ not $p^{n-1}$-torsion is a class function, and is annihilated by taking conjugacy differences once.

\item We may think of elements of $\Sigma_{n+1}$ arising from elements $f\in \Sigma_n$ by taking the function $f$ with values in $\frac1{p^n} \ZP / \ZP$ and dividing by $p$ to give $f'$ with values in $\frac1{p^{n+1}} \ZP / \ZP$. This would still be annihilated by n conjugacy differences. We can use the extra freedom to now move around on the top layer, and give p-torsion differences for each generator.

\item This construction ensures that $\theta_g \Sigma_n \subset \Sigma_{n+1}\, \forall \, g\in G$ and that $p. \Sigma_n \subset \Sigma_{n+1}$.
\end{itemize}
\end{definition}

\begin{proposition}(Structure of Hierarchy)
\end{proposition}

\newpage
%%%%%%%%%%%%%%%%%%%%%%%%%%%%%%%%%%%%
\section{The Algebra of Aardakov and Wadsley}
%%%%%%%%%%%%%%%%%%%%%%%%%%%%%%%%%%%%

%%%%%%%%%%%%%%%%%%%%%%%%%%%%%%%%%%%%
\section{A New Completed Normed Algebra}
%%%%%%%%%%%%%%%%%%%%%%%%%%%%%%%%%%%%