
In \cite{S}, Schutz uses the one-parameter fixed point theory of Geoghegan and Nicas to get
information about the closed orbit structure of transverse gradient flows of closed 1-forms
on a closed manifold $M$. He defines a noncommutative zeta function in an object related to
the first Hochschild homology group of the Novikov ring associated to the 1-form, essentially an image under the Dennis Trace Map and relates this to the torsion of a natural chain homotopy equivalence between the Novikov complex and a completed simplicial complex of $\overline M$ , the universal cover of $M$.

Recall, from \ref{DTM}, for any ring R there is a Dennis trace homomorphism $DT : K_1(R) \rightarrow  HH_1(R)$ from K-theory to Hochschild homology (in all dimensions, but only dimension 1 is used).
They use a variant of $DT$,, the Dennis Trace map, denoted $\mathfrak{DT} :  \overline W \rightarrow \widehat{HH_1(\Z G)_\xi}$. Here $\overline W$ is a subgroup of $K_1(\overline {\Z G}_\xi)$ containing the torsion of $\phi(v)$, and $\widehat{HH_1(\Z G)_\xi}$ is a completion of $HH_1(\Z G)$, see \ref{HHNR} which is related to the Hochschild homology of the Novikov ring by a natural homomorphism $\theta: HH_1(\widehat{\Z G_\xi})\rightarrow \widehat{HH_1(\Z G_\xi)}$.

The main theorem says that if one applies this modified Dennis trace homomorphism to the torsion of the equivalence $\phi(v)$, one gets the (topologically important part of the) closed
orbit structure of the flow induced by $v$ in a recognizable form - that of a (noncommutative)
zeta function. In other words the �Dennis trace� of the torsion equals the zeta function.



\subsection*{Theorem (DTM, Torsion Equivalence and Zeta Functions -  \cite{S} 1.1):}
\emph{Let $\omega$ be a Morse form on a smooth connected closed manifold $M^n$. Let $\xi:G\rightarrow \R$ be induced by $\omega$ and let $C_*^\Delta(\overline M)$ be the simplicial $\Z G$ complex coming from a smooth triangulation of $M$. For every transverse $\omega$-gradient, $\nu$
there is a natural chain homotopy equivalence $\phi(v) : \widehat{\Z G_\xi}\otimes_{\Z G} C_*^\Delta(\overline M) \rightarrow C_*(\omega, \nu)$ whose torsion $t(\phi(v))$ lies in $\overline W$ and satisfies
$$\mathfrak {DT}(t(\phi(v))) = \zeta (-v)$$}

I now ask if the same in true in the Iwasawa completed case - Does the completed Dennis Trace
Map from \ref{general dtm}, performed on the characteristic elements in K-theory of \cite{CFKSV} carry important information?






%%%%%%%%%%%%%%%%%%%%%%%%%%%%%%%%%%%%%%%%%%%%%%%%%%%%%%%%%%%%
%%%%%%%%%%%%%%%%%%%%%%%%%%%%%%%%%%%%%%%%%%%%%%%%%%%%%%%%%%%%
%%%%%%%%%%%%%%%%%%%%%%%%%%%%%%%%%%%%%%%%%%%%%%%%%%%%%%%%%%%%
%%%%%%%%%%%%%%%%%%%%%%%%%%%%%%%%%%%%%%%%%%%%%%%%%%%%%%%%%%%%
%%%%%%%%%%%%%%%%%%%%%%%%%%%%%%%%%%%%%%%%%%%%%%%%%%%