\section{Group Cohomology}

The next two sections study cohomology of groups and algebras. A
reference for this is Lazards paper [Lz]. I
recall the key points i will need for my calculations.

Following the treatment in \cite{HS} I shall define, for an
abstract group G (over the integers) the homology groups
$H_n(G,B)$ and the cohomology groups $H^n(G,B)$, for $n\geq 0$,
where A is a left, and B a right G-module. This follows from the
theory of derived functors, taking the group ring $\Z G$ as the
"ground ring" $\Lambda$".

We begin by giving a precise defintion of the group ring, which we
define by giving its underlying abelian group and specifying the
composition of 2 elements. Note, in this chapter all groups are
written multiplicatively.

\subsection{Group Ring $\Z G$}\label{df3.1}
\subsection{Definition}\label{df3.1.1}
The \textbf{Integral Group Ring $\Z G$} of G has as it's
underlying abelian group the free abelain group with basis the set
of elements of G, and the product of 2 basis elements is induced
from the product in G.

ie.
\begin{itemize}
    \item Elements are of the form $\sum_{x\in G} m(x)\, x$ (where
    $m:G\mapsto \Z$ is zero on all but a finite number of elements
    of G).
    \item Multiplication is given by
    $$ \left ( \sum_{x\in G} m(x)\, x \right ). \left (\sum_{y\in G} m'(y)\,
    y\right ) = \sum_{x,y \in G} (m(x).m'(y))xy$$
\end{itemize}

Consider the natural embedding
$$i:G\mapsto \Z G:g\mapsto \sum
m(x)\, x\text{ where }m(x) = 
\begin{cases}
  1 & \text{ if } x=g \\
  0 & \text{ otherwise}
\end{cases}$$
 I now give a universal property which will be
important when we come to generalise group cohomology to Lie
algebras.

\subsection{Proposition: Extension of Nice Multiplicative
Functions}\label{df3.1.2} Let R be a ring. Given any function
$f:G\mapsto R$ which interacts well with the multiplicative
structure of the group and ring such that: $f(xy)=f(x).f(y)$ and
$f(1_G) = 1_R$ then there exists a unique ring homomorphism $f':\Z
G\mapsto R$ such that $f'i = f$. Clearly
$$f'(\sum_{x\in G} m(x)\, x) = \sum_{x\in G} m(x) f(x)$$ is the
only possible choice and this commutes.

This proposition gives an interpretation of a G-module A as a $\Z
G$-module (which is a G-module via the embedding $i$) and so we
may use these terms interchangeably.

\subsection{Augmentation Map}\label{df3.1.1}
Consider the trivial map from G into integers $\Z$, given by $x\in
G \mapsto 1\in\Z\,\forall\, x\in G$. \ref{df3.1.2} gives a unique
ring homomorphism $\varepsilon: \Z G \mapsto \Z$, the augmentation
of $\Z G$ defined by:
$$\varepsilon \left ( \sum_{x\in G} m(x)\, x \right ) = \sum_{x\in G}
m(x)$$

Denote the kernel of $\varepsilon$, called the
\textbf{Augmentation Ideal of G} by $IG$. It is central to the
theory of group cohomology, see for example \ref{df3.4.3}.


\subsection{Construction of (Co)Homology Groups}\label{df3.2}

For a left G-module A, thinking of $\Z$ as a trivial G-module, we
define the \textbf{n-th cohomology group of G with coefficients in
A} by
$$H^n(G,A) = Ext^n_G (\Z,A)$$
And dually, for a right G-module B, the homology groups are
$$H_n(G,A) = Tor^G_n (B,\Z)$$
From now on i will concentrate on calculations relating to
cohomology although for each of the results there is an analogous
dual result for homology.

\subsection{Explicit Calculation}\label{df3.3}
\begin{itemize}
    \item Take a G-progective resolution \textbf P of the trivial
    (left) module $\Z$.
    \item Form the complexes $Hom_G(\textbf P, A)$ and $B\otimes_G
    \textbf P$.
    \item Compute their homology
\end{itemize}
As an example i give an explicit description of a resolution of
$\Z$ over our given group, known as the \textbf{Homogeneous Bar
Resolution}. We need to define G-modules to form the projection:
\begin{enumerate}
    \item Let $\overline{B_n},\, n\geq 0$ be the free abelian
    group on all $(n+1)$-tuples \\ $(Y_0,y_1,\dots , y_n),\, y_i\in
    G$. Define the left G-module strucure in $\overline{B_n}$ by
    $$y(y_0, y_1, \dots , y_n) = (yy_0, yy_1, \dots , yy_n)\,
    y\in G$$
    Hence $\overline{B_n}$ is a free G-module, with base
    consisting of the $(n+1)$-tuples $(1,y_1, \dots , y_n)$.

    \item Define the differential maps of the sequence
    $$\overline{\textbf B}:\dots \rightarrow \overline{B_n}
    \rightarrow^{\rho_n} \overline{B_{n-1}}\rightarrow
    \dots\rightarrow \overline{B_1}\rightarrow^{\rho_1}
    \overline{B_0}$$
    by the boundary formula
    $$\rho_n(y_0,y_1,\dots ,y_n) = \sum_{i=0}^n {(-1)}^i
    (y_0,\dots , \widehat{y_i}, \dots , y_n)$$
    \item It is easily seen that $\rho_{n-1}\rho_n = 0 \text{ for
    } n\geq 2$ and recalling the augmentation map $\varepsilon$ we
    also have $\varepsilon \rho_1 = 0$.
    \item Explicit calculation, or contracting homotpoies from
    topology give that
    $$\dots \rightarrow \overline{B_n}\rightarrow^{\rho_n}
    \overline{B_{n-1}}\rightarrow \dots \rightarrow
    \overline{B_1}\rightarrow^{\rho_1}\overline{B_0}
    \rightarrow^\varepsilon \Z\rightarrow 0$$
    is a free G-resolution of $\Z$.
\end{enumerate}


\subsection{Exact Sequences}\label{df3.4}
From the long exact sequences associated to derived functors, see
\ref{df1.5.3} we have, for a short exact sequence
$A'\rightarrowtail A\twoheadrightarrow A''$ of G-modules there
exists a long exact cohomology sequence:
\begin{eqnarray}
\nonumber 0 \rightarrow H^0(G,A')\rightarrow H^0(G,A) \rightarrow
H^0(G,A'')\rightarrow H^1(G,A')\rightarrow \dots\\
\nonumber \dots \rightarrow H^n(G,A')\rightarrow H^n(G,A)
\rightarrow H^n(G,A'')\rightarrow H^{n+1}(G,A')\rightarrow \dots
\end{eqnarray}
The following are other properties of Group Cohomology easily
deduced from the theory of derived functors:

\subsection{}
If A is injective then $H^n(G,A) = 0 \text{ for all } n\geq 1$.

\subsection{}
Hence, if $A\rightarrowtail I\twoheadrightarrow A'$ is an
injective presentation of A, then the long exact sequence
collapses to give
$$H^{n+1}(G,A) \cong H^n(G,A') \text{ for } n\geq 1$$

\subsection{Changing the Group Acting}\label{df3.4.3}

I will show $H^n(G,A) \cong Ext_g^{n-1}(IG,A)$ hence we can
equally well work with the group or its augmentation ideal. This
is a special case of a more general shifting technique:

\subsection*{Lemma}
\emph{Let $0\rightarrow K \rightarrow P_k \rightarrow ^\phi \dots
\rightarrow P_0\rightarrow \Z \rightarrow 0$ be an exact sequence
of left G-modules with $P_0, \dots , P_k$ projective. Then the
following sequence is exact and specifies the cohomology groups of
G:\begin{enumerate}
    \item $Hom_G(P_k,A)\rightarrow Hom_G(K,A) \rightarrow
    H^{k+1}(G,A) \rightarrow 0$
    \item $H^n(G,A) \cong Ext_G^{n-k-1} (K,A) \text{ for } n\geq
    k+2$
\end{enumerate}}
\subsection*{Proof}
\begin{enumerate}
    \item This is just the definition of $H^{k+1}$ as the coker of
    the induced map $Hom(P_k,A)\mapsto Hom(K,A)$.
    \item Let $\dots \rightarrow B_1 \rightarrow B_0
    \rightarrow^\varphi K\rightarrow 0$ be a projective resolution
    of K used to calculate $Ext_G^{n-k-1}(K,A)$. Then, since $\mathbf{Im \varphi = K = ker \phi}$ we can piece
    together with our original resolution to get a new projective
    resolution of $\Z$:
    $$\dots B_{n-k-2}\rightarrow^\alpha B_{n-k-1}\rightarrow^\beta
    B_{n-k} \rightarrow \dots \rightarrow B_0\rightarrow
    P_k\rightarrow \dots \rightarrow P_0 \rightarrow \Z\rightarrow
    0$$
    Then,
    $$Ext_G^{n-k-1}(K,A) = {ker\beta}/{im\alpha} = H^n(G,A)$$ as
    required.
\end{enumerate}
In particular, using the resolution $0\rightarrow IG\rightarrow \Z
G\rightarrow \Z\rightarrow 0$, we have, for $n\geq2$
$$H^n(G,A) \cong Ext_G^{n-1} (IG,A)$$

