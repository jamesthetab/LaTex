I begin this chapter by working with the projective resolutions of the group ring to show finiteness of integral group homology. I then apply this to the decomposition of Hochschild homology and show that the Mittag-Leffler condition holds. I use the vanishing of $lim^1$ to comment on the connection between the homology of the completed resolution and the completion of the homologies of the finite resolution.

I then interpret these results in terms of cofinal sequences, and for the $n=1$ explain the connection with the Lie Bracket.

Finally, I introduce the concept of a free product and explain how our result has implications for the topology here.


%%%%%%%%%%%%%%%%%%%%%%%%%%%%%%%%%%%%
\section{Finiteness}
%%%%%%%%%%%%%%%%%%%%%%%%%%%%%%%%%%%%

In order to calculate the group homology via $Tor$ groups we use a canonical resolutions $B_\star$ and $B_\star^u$ of the trivial $G$-module $\Z$ by free left $\Z G$-modules.

\begin{definition} (\textbf{The \textit{Normalised} and \textit{Unnormalised} Bar Resolutions \label{bar}})

$$0\leftarrow \Z \xleftarrow{\epsilon}B_0\xleftarrow{d} B_1\xleftarrow{d} B_2 \xleftarrow{d} \dots .$$

$$0\leftarrow \Z \xleftarrow{\epsilon}B_0^u\xleftarrow{d} B_1^u\xleftarrow{d} B_2^u \xleftarrow{d} \dots .$$

Where $B_0^u$ is $\Z G$. Let the symbol $[\,]$ denote $1 \in \Z G$, and $\epsilon: B_0^u \rightarrow \Z$ the map which sends $[\,]$ to $1$.

For $n\geq 1$, $B_n^u$ is the free $\Z G$-module on the set of all symbols $[ g_1 \otimes \dots \otimes g_n]$ with $g_i \in G$, while $B_n$ is the free $\Z G$-module on the (smaller) set of all symbols $[ g_1 | \dots | g_n ]$ with the $g_i \in G - \{1 \}$.

\end{definition}

We may identify $B_n$ with the quotient $B_n^u$ by the submodule $S_n$ generated by the set of all \\ symbols $[g_1 \otimes \dots \otimes g_n]$ with some $g_i$ equal to $1$.

\begin{definition} (\textbf{Differentials in normalised and unnormalised resolutions})

For $n\geq 1$, define the differential $d: B_n^u \rightarrow B_{n-1}^u$ to be $d = \sum_{i=0}^n (-1)^i d_i$, where:
\begin{eqnarray}
d_0([g_1\otimes\dots\otimes g_n]) &=& g_1 [g_2 \otimes \dots \otimes g_n]; \nonumber\\
d_i([g_1\otimes\dots\otimes g_n]) &=& [g_1 \otimes \dots \otimes g_i g_{i+1} \otimes \dots \otimes g_n]\text{ for } i=1, \dots, n-1;\nonumber\\
d_n([g_1\otimes\dots\otimes g_n]) &=& [g_1 \otimes \dots \otimes g_{n-1}].\nonumber
\end{eqnarray}

Similarly, the differential for $B_\star$ is given by the following:
$$d_i([g_1|\dots|g_n]) = [[g_1|\dots|g_i g_{i+1} | \dots |g_n ] \text{ when }g_ig_{i+1}\neq 1, \text{ and } =0\text{ when } g_ig_{i+1}=1.$$
\end{definition}

\begin{theorem} (\textbf{Bar resolutions})
The sequences in \ref{bar} above are exact. Thus both $B_\star$ and $B_\star^u$ are resolutions of $\Z$ by free left $\Z G$-modules.
\end{theorem}

\begin{example}(\textbf{First Group Homology with Coefficients in Integers})

For every right $G$-module $A$, $H_\star (G;A)$ is the homology of the chain complex $A\otimes_{\Z G} B_\star$. In particular we see that $H_1(G; \Z)$ is the free abelian group on the symbols $[g], \, g\in G$, with the relations $[1]=0$, and $[f]+[g] = [fg]$ for ll $f,g\in G$. This recovers the calculation earlier that 

$$H_1(G; \Z ) = G/ [G,G].$$

\end{example}

%%%
\begin{theorem} (see \cite{Weibel}, 6.5.8, \textbf{(Co)Homology Groups are Torsion})

Let $G$ be a finite group with $m$ elements. Then for $n \neq 0$ and every $G$-module $A$, both $H_n(G;A)$ and $H^n(G;A)$ are annihilated by $m$, that is they are $\Z /m$-modules.

\end{theorem} 


6.5.8

\begin{proof}

Let $\eta$ denote the endomorphism of $B_*$, which is multiplication by
$(m-N)$ on $B_0$ and multiplication by $m$ on $B-n$, $\neq 0$. We claim
that $\eta$ is null homotopic. So applying $A\otimes$ or $Hom( - , A)$,
will then yield a null homotopic map, which must become zero upon taking
homology, proving the above theorem.

For definitions of "null homotopic" and "chain contraction" refer to \ref{df1.2.3}.

Define $\nu_n : B_n \rightarrow B_{n+1}$ by the formula

$$ \nu_V([g_1 | \dots | g_n ]) = (-1)^{n+1} \sum_{g\in G} [ g_1 | \dots
|g_n |g].$$

Setting $\omega = [g_1 | \dots |g_n])$ and $\epsilon = (-1)^{n+1}$, we
compute for $n\neq 0$

$$d\nu_n(\omega) = \epsilon \sum \{ g_1[\dots | g] + \sum (-1)^i [\dots
|g_i g_{i+1}| \dots | g] - \epsilon [\dots | g_{n-1} | g_n g] + \epsilon
\omega \}$$

$$\nu_{n-1} d (\omega) = - \epsilon \sum \{ g_1[\dots | g] + \sum (-1)^i
[\dots |g_i g_{i+1}| \dots | g] - \epsilon [\dots | g_{n-1} | g] \}.$$

As the sums over all $g \in G$ of $[\dots | g_n g ]$ and $[\dots | g]$
agree, we see that $(dv+vd)(\omega)$ is $\epsilon^2 \sum \omega =
m\omega$. Now $d\nu_0([]) = d(-\sum [g]) = 9m-N){}$, where $N = \sum g$ is
the norm. Thus $\{ \nu_V \}$ provides the chain contraction needed to make
$\nu$ null homotopic.

\end{proof}


%%%


%%%
\begin{corollary} (see \cite{Weibel}, 6.5.10, \textbf{Finitely generated modules give finite (Co)Homology groups}\label{6.5.10})

If $G$ is a finite group and $A$ is a finitely generated $G$-module, then $H_n(G;A)$ and $H^n(G;A)$ are finite abelian groups for all $n\neq 0$.

\end{corollary}

\begin{proof}
Each $A \otimes_{\Z G} B_n$ and $Hom_G(B_n , A)$ is a finitely generated
abelian group. Hence $H_n(G;A)$ and $H^n(G;A)$ are finitely generated $\Z
/ m$-modules when $n\neq 0$, hence finite.
\end{proof}

In particular for a finite group $G$ and $A = \Z$, the integral homology groups $H_n(G;\Z)$ are finite for $n \geq 1$. 

\section{Complete Resolution for Calculating the Hochschild Homology of an Iwasawa Algebra}

%%%%%%%%%%%%%%%%%%%%%%%%%%%%%%%%%%%%%%%%%%%%%%
I now present a filtration on the Hochschild Complex of the group algebra, $\ZP[G]$, see \ref{hochschildcomplex}, so that the quotients $C_{\ZP[G]} / F_p C_{\ZP[G]}$ recover the Hoschschild Complex for finite quotients, $C_{\ZP[G/G_n]}$. 

Taking the completion, $\varprojlim C_{\ZP[G]} / F_p C_{\ZP [G]} \cong \varprojlim C_{\ZP[G/G_n]}$ gives the complex for the Iwasawa Algebra, $\Lambda_G$, where the tensor in the original complex is now replaced with a completed tensor, see \ref{completedtensor}.

At level $0$, the filtration of $C_0 = \ZP[G]$, is given by $F_p C_0 = I_p$, the $p^{th}$ Augmentation Ideal, where (see \ref{3.2.5}),

$$ C_0 / F_p C_0 \cong \ZP [G/G_n].$$

At level $1$, the filtration of $C_1 = \ZP[G] \otimes \ZP[G]$, is given by $F_p C_1 = I_p \otimes \ZP[G] + \ZP[G] \otimes I_p$ where,

\begin{eqnarray*}
C_1 / F_p C_1 &\cong& (\ZP[G] \otimes \ZP[G]) / ( I_p \otimes \ZP[G] + \ZP[G] \otimes I_p)\\
				&\cong& (\ZP[G] \otimes \ZP[G/G_n]) / ( I_p \otimes \ZP[G/G_n])\\
				&\cong& (\ZP[G/G_n] \otimes \ZP[G/G_n]).
\end{eqnarray*}

\begin{proposition}(Structure of Completed Complex)

The Hochschild Complex, $\{C_n\}$, of the group algebra $\ZP[G]$ possesses a filtration $\{F_pC\}$ where
\begin{eqnarray*}
F_p C_n 	&\cong& I_p \otimes \ZP[G]^{\otimes n} + \text{ symmetric rearrangements}\\
		&\cong& \sum_{i=0}^n \ZP[G]^{\otimes i} \otimes I_p \otimes \ZP[G]^{\otimes n-i}
\end{eqnarray*}

Moreover, these filtrations are preserved under the boundary map, $b$, since $I_p$ is an ideal. Hence maps on Quotients,

$$C_{\ZP[G]} / F_p C_{\ZP[G]} \cong C_{\ZP[G/G_n]},$$

are well defined and we may form the inverse limit, 

$$\varprojlim C_{\ZP[G]} / F_p C_{\ZP [G]} \cong \varprojlim C_{\ZP[G/G_n]}$$

which then gives the Hochschild Complex for the Iwasawa Algebra, $\Lambda_G$, where the tensor in the original complex is now replaced with a completed tensor.
\end{proposition}

In particular, if we take $C$ to be the chain of simplicial modules used to calculate Hochschild Homology of the group ring of $G$, $F_nC$ be the image for the group of $p^n$-th powers, then we have the Iwasawa Algebra appearing as the completion of $C/F_nC$:

$$0 \rightarrow  \underleftarrow{\text{lim}^1} HH_{n+1}(\Z_p[G/G^m]) \rightarrow HH_n(\Lambda_G) \rightarrow  \underleftarrow{\text{lim}}\,H_n(\Z_p[G/G^m]) \rightarrow 0.$$ 

Specialising for $n = 0,1$:

$$0 \rightarrow  \underleftarrow{\text{lim}^1} HH_{1}(\Z_p[G/G^m]) \rightarrow HH_0(\Lambda_G) \rightarrow  \underleftarrow{\text{lim}}\,H_0(\Z_p[G/G^m]) \rightarrow 0.$$ 

$$0 \rightarrow  \underleftarrow{\text{lim}^1} HH_{2}(\Z_p[G/G^m]) \rightarrow HH_1(\Lambda_G) \rightarrow  \underleftarrow{\text{lim}}\,H_1(\Z_p[G/G^m]) \rightarrow 0.$$ 

Giving,

$$ \underleftarrow{\text{lim}}\,H_0(\Z_p[G/G^m]) =    HH_0(\Lambda_G) / \underleftarrow{\text{lim}^1} HH_{1}(\Z_p[G/G^m])$$

Which might tie in the work by Mahesh where he looks into different presentations of $\Z_p [Conj(G)]$ using traces.

$$ \underleftarrow{\text{lim}}\,H_1(\Z_p[G/G^m]) =    HH_1(\Lambda_G) / \underleftarrow{\text{lim}^1} HH_{2}(\Z_p[G/G^m])$$

%%%%%%%%%%%%%%%%%%%%%%%%%%%%%%%%%%%%%%%%%%%%%%
\begin{proposition}(Direct Proof of E-M Sequence for Hochschild Homology)

$$0 \rightarrow  \underleftarrow{\text{lim}^1} HH_{1}(\Z_p[G/G^m]) \rightarrow HH_0(\Lambda_G) \rightarrow  \underleftarrow{\text{lim}}\,H_0(\Z_p[G/G^m]) \rightarrow 0.$$ 
\end{proposition}

\begin{proof}
Taking $\Delta = i - \text{Projection}$, we have the onto maps, with kernels the inverse limits for each $n\in \mathbb{N}$:

$$0\rightarrow \varprojlim \ZP[G/G_n]^{\otimes 2} \rightarrow \prod \ZP[G/G_n]^{\otimes 2} \xrightarrow{\Delta}  \prod \ZP[G/G_n]^{\otimes 2} \rightarrow 0$$

and,

$$0\rightarrow \varprojlim \ZP[G/G_n] \rightarrow \prod \ZP[G/G_n] \xrightarrow{\Delta}  \prod \ZP[G/G_n]\rightarrow 0.$$

We may connect these sequences using the extension of the Lie algebra commutator map, $d$. By the definition of homology we have the following diagram using the snake lemma construction (see \ref{eightterm}), and denoting the kernel of a map by$Z(-)$:


\begin{tikzpicture}[>=triangle 60]
\matrix[matrix of math nodes,column sep={92pt,between origins},row
sep={60pt,between origins},nodes={asymmetrical rectangle}] (s)
{
|[name=kw]| 0  &|[name=ka]| Z(\Lambda_G^{\otimes 2}) &|[name=kb]| \prod Z(\ZP[G/G_n]^{\otimes 2}) &|[name=kc]| \prod Z(\ZP[G/G_n]^{\otimes 2}) \\
%
|[name=kx]| 0  &|[name=A]| \varprojlim[G/G_n]^{\otimes 2} &|[name=B]| \prod \ZP [G/G_n]^{\otimes 2} &|[name=C]| \prod \ZP [G/G_n]^{\otimes 2}  &|[name=01]| 0 \\
%
|[name=02]| 0 &|[name=A']| \varprojlim[G/G_n] &|[name=B']| \prod \ZP [G/G_n] &|[name=C']| \prod \ZP [G/G_n] & |[name=ky]| 0 \\
%
&|[name=ca]| HH_0 (\Lambda_G) &|[name=cb]| \prod {HH}_0 (\ZP [G/G_n]) &|[name=cc]| \prod {HH}_0 (\ZP [G/G_n]) & |[name=kz]| 0\\
};
\draw[->] (ka) edge (A)
          (kb) edge (B)
          (kc) edge (C)
          (A) edge (B)
          (B) edge (C)
          (C) edge (01)
          (A) edge node[auto] {\(d\)} (A')
          (B) edge node[auto] {\(d\)} (B')
          (C) edge node[auto] {\(d\)} (C')
          (02) edge (A')
          (A') edge  (B')
          (B') edge (C')
          (A') edge (ca)
          (B') edge (cb)
          (C') edge (cc)
;
\draw[->,gray] (ka) edge (kb)
               (kb) edge (kc)
               (ca) edge (cb)
               (cb) edge (cc)
               (kw) edge (ka)
               (kx) edge (A)
               (C') edge (ky)
               (cc) edge (kz)
;
\draw[->,gray,rounded corners] (kc) -| node[auto,text=black,pos=.7]
{\(\partial\)} ($(01.east)+(.5,0)$) |- ($(B)!.35!(B')$) -|
($(02.west)+(-.5,0)$) |- (ca);
\end{tikzpicture}

Hence, the snake lemma gives:

$$\prod Z(\ZP [G/G_n]^{\otimes 2} ) \xrightarrow{\Delta} \prod Z(\ZP [G/G_n]^{\otimes 2} ) \xrightarrow {\partial} HH_0 (\Lambda_G) \rightarrow \prod HH_0 (\ZP [G/G_n] ) \xrightarrow{\Delta} \prod HH_0 (\ZP [G/G_n] ) \rightarrow 0$$

By definition of the inverse limit we also have the exact sequence:

$$0\rightarrow \varprojlim HH_0 (\ZP[G/G_n]) \rightarrow \prod HH_0 (\ZP [G/G_n] ) \xrightarrow{\Delta} \prod HH_0 (\ZP [G/G_n] ) \rightarrow 0$$

Hence,
\begin{eqnarray}
Ker\, \Delta 	&\cong& \varprojlim HH_0 (\ZP[G/G_n]) \\
			&\cong& HH_0(\Lambda_G) / Im(\partial) \\
			&\cong& HH_0(\Lambda_G) / Coker ( \Delta) \\
			&\cong& HH_0(\Lambda_G) / \limone (Z(\ZP[G/G_n]^{\otimes 2}))
\end{eqnarray}

Referring to the proof of \ref{3.5.8} I recall that if we denote $B_i \subset Z_i \subset C_i$ to be the sub complexes of boundaries and cycles in the complex $C_i$, then $Z_i/B_i$ is the chain complex $H_*(C_i)$, and by sandwiching, $\limone Z_i \cong \limone H_*(C_i)$. This gives  $\limone (Z(\ZP[G/G_n]^{\otimes 2})) \cong \limone HH_1(\ZP[G/G_n])$. Hence, 
$$\varprojlim HH_0 (\ZP[G/G_n]) \cong HH_0(\Lambda_G) / \limone (HH_1(\ZP[G/G_n])),$$
or equivalently,
$$0\rightarrow  \limone (HH_1(\ZP[G/G_n])) \rightarrow HH_0(\Lambda_G) \rightarrow \varprojlim HH_0 (\ZP[G/G_n])\rightarrow 0.$$


\end{proof}
%%%%%%%%%%%%%%%%%%%%%%%%%%%%%%%%%%%%%%%%%%%%%%


The finiteness result, \ref{6.5.10} applies when G is the centraliser of an element the finite quotient of an uniform pro-$p$ group, and combining \ref{decomposition} with \ref{emsequences} we get:

\begin{corollary}(\textbf{Completions and Hochschild Homology}\label{finitehh})
\begin{enumerate}
\item By finiteness, the Mittag-Leffler condition applies giving:
$$\limone HH_m (\ZP [G/G_n]) = \limone \left \{ \bigoplus _{g_n \in \text{ccl } G/G_n } H_m (Z(g_n); \Z) \right \} = 0, \forall\, m\geq 1$$

\item Hence, \ref{emsequences} reduces to:

$$ 0 \rightarrow 0 \rightarrow HH_m \left ( \varprojlim_n \ZP [G/G_n] \right )  \rightarrow  \varprojlim_n \left ( HH_m ( \ZP [G/G_n])\right ) \rightarrow 0, \forall m\geq 0$$

We may write in terms of the Iwasawa Algebra, $\Lambda_G$:

$$HH_m (\Lambda_G) \cong  \varprojlim_n (HH_m ( \ZP [G/G_n]), \forall m\geq 0.$$
 
\end{enumerate}
\end{corollary}

\begin{example} (Intuition for the Approach of this Chapter)


Our intuition for this was built up in several stages:

To begin with we realised that the acts of taking homology and taking inverse limits need not commute, and we thought that we might be able to combine the derived functors of taking the inverse limit with the derived functors of homology (Hoschschild Homology in this case) to construct a term of total degree 1. Thus the first level might consist of the zeroth level of the inverse limit with the first level of the homology and vice versa.

Things then looked a lot more difficult when we realised that since the Inverse Limit was defined as the set of Invaraints (under shift map $\Delta$), it gave rise to a homology theory and so the higher derived functors had negative homological degree. This meant that the level of total degree 1 might consist of first homology with zeroth inverse limit, second homology and first derived functor of the inverse limit, third homology and second derived functor of the inverse limit... .

This situation was simplified by the idea that when the indexing set is the natural numbers, all the higher derived functors of the inverse limit vanish ($\varprojlim^n A_i = 0$ for all $n\geq 2$, see \ref{simplification}). So we were left to combe the inverse limit of $HH_n$ and the first derived functor of the inverse limit applied to $HH_{n+1}$. This is precisely how we use the Eilenberg-Moore Filtration, see \ref{emsequences}.
  
\end{example}
%%%

%%%%%%%%%%%%%%%%%%%%%%%%%%%%%%%%%%%%
\section{Cofinal Completions}
\newpage


%%%%%%%%%%%%%%%%%%%%%%%%%%%%%%%%%%%%

%%%%%%%%%%%%%%%%%%%%%%%%%%%%%%%%%%%%
\section{Interpreting as Commutivity of Inverse Limits and Lie Commutator Bracket}

Let $G$ be a Uniform pro-$p$ group, and $G_n$ the Frobenius powers, let $d$ be the commutator map used in associative Lie algebras: $d(a \otimes b ) \equiv ab - ba$, and denote the Iwasawa algebra, $\varprojlim \ZP [G/G_n]$, by $\Lambda_G$.


\begin{proposition}($lim^1$: Comparing Inverse Limits and Lie Bracket)
\begin{eqnarray*}
\limone HH_1 (\ZP [G/G_n]) &\cong& \varprojlim \{d \{\ZP[G/G_n] \otimes \ZP[G/G_n] \}\} / d \{ \varprojlim \{\ZP[G/G_n] \otimes \ZP[G/G_n] \}\} \\
					&\cong& \varprojlim \{d \{\ZP[G/G_n] \otimes \ZP[G/G_n] \}\} / d \{ \Lambda_G \widehat{\otimes} \Lambda_G \} \\
\end{eqnarray*}
\end{proposition}

\begin{proof}

Consider the exact sequence of towers arising from,

$$0 \rightarrow Ker_d(\ZP[G/G_n] \otimes \ZP[G/G_n]) \rightarrow (\ZP[G/G_n] \otimes \ZP[G/G_n])  \xrightarrow{d} d(\ZP[G/G_n] \otimes \ZP[G/G_n])  \rightarrow 0$$

Applying the eight term exact sequence of \ref{longderivedinv} we have:
\begin{eqnarray*}
0 \rightarrow \varprojlim Ker_d(\ZP[G/G_n] \otimes \ZP[G/G_n]) \rightarrow \varprojlim (\ZP[G/G_n] \otimes \ZP[G/G_n])  \xrightarrow{d} \varprojlim d(\ZP[G/G_n] \otimes \ZP[G/G_n])  \rightarrow \dots\\
\dots \rightarrow \limone Ker_d(\ZP[G/G_n] \otimes \ZP[G/G_n]) \rightarrow \limone(\ZP[G/G_n] \otimes \ZP[G/G_n])  \xrightarrow{d}  \limone d(\ZP[G/G_n] \otimes \ZP[G/G_n])  \rightarrow 0
\end{eqnarray*}

Since maps in the tower $\{  \ZP[G/G_n] \otimes  \ZP[G/G_n] \}$ are onto, by \ref{ML} we have that 

$$\limone \{  \ZP[G/G_n] \otimes  \ZP[G/G_n] \} = 0.$$ 

Hence,
\begin{eqnarray*}
0 \rightarrow \varprojlim Ker_d(\ZP[G/G_n] \otimes \ZP[G/G_n]) \rightarrow \varprojlim (\ZP[G/G_n] \otimes \ZP[G/G_n])  \xrightarrow{d}  \dots\\
\dots \xrightarrow{d} \varprojlim d(\ZP[G/G_n] \otimes \ZP[G/G_n])  \rightarrow \limone Ker_d(\ZP[G/G_n] \otimes \ZP[G/G_n]) \rightarrow 0
\end{eqnarray*}

By exactness,

$$ \limone Ker_d(\ZP[G/G_n] \otimes \ZP[G/G_n]) \cong  \varprojlim d(\ZP[G/G_n] \otimes \ZP[G/G_n])  / d(\varprojlim (\ZP[G/G_n] \otimes \ZP[G/G_n]))$$

The proof of \ref{3.5.8} gives  $\limone HH_1 (\ZP [G/G_n]) \cong  Ker_d(\ZP[G/G_n] \otimes \ZP[G/G_n])$ which yields the result.
\end{proof}

Prior to applying the Mittag-Leffler condition to $\limone HH_1(\ZP [G/G_n])$ I interpreted this in terms of a question about the commutivity of Inverse Limits and Lie Brackets as follows.  

I was interested in finding elements in 
$$\underleftarrow{lim}^1 \, HH_1(\mathbb{Z} _p [G/G_n]).$$
\\
Equivalently, to show $lim^1 \neq 0$, I needed to find an element in

$$\underleftarrow{lim} \,\{d (  \mathbb{Z}_p [G/G_n] \otimes
\mathbb{Z}_p{[G/G_n]} ) \} $$

which was not in

$$d ( \underleftarrow{lim} \,  \{\mathbb{Z}_p{[G/G_n]} \otimes
\mathbb{Z}_p{[G/G_n]} \}) \cong  d \{ \Lambda_G \widehat{\otimes} \Lambda_G \}.$$

From universal properties, 

$$\underleftarrow{lim} \,\{d (  \mathbb{Z}_p{[G/G_n]} \otimes
\mathbb{Z}_p{[G/G_n]} ) \} \rightarrow d ( \underleftarrow{lim} \,
\{\mathbb{Z}_p{[G/G_n]} \otimes
\mathbb{Z}_p{[G/G_n]} \})\rightarrow 0.$$

I hoped that in some interesting cases we would have
$$\underleftarrow{lim} \,\{d (  \mathbb{Z}_p{[G/G_n]} \otimes
\mathbb{Z}_p{[G/G_n]} ) \} \neq d ( \underleftarrow{lim} \,
\{\mathbb{Z}_p{[G/G_n]} \otimes
\mathbb{Z}_p{[G/G_n]} \}).$$

But in \ref{finitehh} above I showed $\limone HH_1 \{\ZP [G/G_n]\} =0$, hence

$$\underleftarrow{lim} \,\{d (  \mathbb{Z}_p{[G/G_n]} \otimes
\mathbb{Z}_p{[G/G_n]} ) \} \,\cong\, d ( \underleftarrow{lim} \,
\{\mathbb{Z}_p{[G/G_n]} \otimes
\mathbb{Z}_p{[G/G_n]} \}).$$


%%%%%%%%%%%%%%%%%%%%%%%%%%%%%%%%%%%%
\section{Interpreting vanishing of lim1 in terms of Hausdorff Property}
%%%%%%%%%%%%%%%%%%%%%%%%%%%%%%%%%%%%
I begin by recalling from \ref{emsequences}) that the term we have shown to be zero, $\limone H_{n+1}(C/F_pC)$ is infact the image of the filtration on homology, $\bigcap F_pH_n(C)$, measuring how far from being Hausdorff the inherited filtration on homology is.

I begin by expressing this for the n=0 case. Since the filtrations on the 1st level of the complex take the form $I_p + \ZP[G]$, we are looking at the images of homology groups $(I_p )/(d(I_p,\ZP[G])$ in $(\ZP[G])/(d(\ZP[G],\ZP[G])$, which is $(I_p+\ZP[G])/(d(\ZP[G],\ZP[G])$, and so we may state (from \ref{finitehh}) that 

$$\bigcap_{p=0}^{\infty} (I_p+\ZP[G])/(d(\ZP[G],\ZP[G])) = 0$$

Equivalently, 

$$\bigcap_{p=0}^{\infty} (I_p+\ZP[G]) =d(\ZP[G],\ZP[G])$$

It is tempting to think that just because our original complex is Hausdorff it's filtration has to be. We may rephrase this as if $\bigcap A_i = 0$, it seems that $\bigcap (A_i + M) = M$, or equivalently $\bigcap (A_i + M)/M = 0$.

This is true for $M$ finite when we can split up the sum into  $\bigcap (A_i /M) + \bigcap (M/M) = 0$, but there can be more interesting behaviour for $M$ infinite. Let $A_i = \prod_{n= - \infty}^{-i} \Z$, and $M = \bigoplus_{n= -\infty}^0\Z$. Then for each $i$, $A_i + M = \prod_{n= - \infty}^{0} \Z$. Hence, 

$$\bigcap (A_i + M)/M =  \left\{ \prod \Z \right\} / \left\{ \bigoplus \Z \right\} = \{\text{Uncountable}\} /\{\text{Countable}\} \neq 0.$$ 

