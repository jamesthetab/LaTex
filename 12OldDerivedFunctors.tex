\section{Derived functors and Homology}\label{df1}
\subsection{Introduction to Derived Functors}\label{df1.1}

Derived Functors are a central tool in Homological Algebra.

Given an additive functor $F:\mathbb{U} \mapsto \mathbb{B}$ from
the abelian category $\mathbb{U}$ to the abelian category
$\mathbb{B}$, we may form its left derived functors.
$L_nF:\mathbb{U} \mapsto \mathbb{B}$. This is subject to the
technical condition that $\mathbb{U}$ has enough projectives -
that every object of $\mathbb{U}$ has a projective presentation.

$L_nF(A)$ is then a function depending on the two variables F and
A. The key property is that these give rise to 2 Exact Sequences -
arising from varying A and F respectively.

2 cases are of special interest to us:
\begin{enumerate}
    \item We may study F as the tensor product $F_B: \mathbb M
    \mapsto \mathbb{Ub}$ where $F_B(A) = A\otimes_\Lambda B$, or
    equivalently, by symmetry of the tensor product $G_A: \mathbb M
    \mapsto \mathbb{Ub}$ where $G_A(B) = A\otimes_\Lambda B$.
    Which leads to $L_nF_B(A) \cong L_nG_A(B)$, which are both
    equal to the familiar functor $Tor_n^\Lambda$ which measures
    by how much the functor $\otimes$ fails to be right exact.

    \item  Similarly, provided there are enough injectives we can
    study right derived functors. The functor $Hom$ then leads to
    a generalisation of the Extension group to $Ext^n_\Lambda$. We
    will see how these groups degenerate to give cohomology groups
    of a group acting on one module, and we can then view these
    $Ext$-groups as a natural 2-module analogue of the standard
    cohomology groups. It is then natural ro study alternating
    products of their orders of the type used in Euler
    Characteristics of 1 module.
\end{enumerate}
I aim to give a survey of the theorems in this area and how they
fit together to  form the machinery rather than quoting each
proof.

\subsection{Homotopy}\label{df1.2}
\subsection{Background}\label{df1.2.1}
Let $\varphi,\psi:\textbf{C}\mapsto \textbf{D}$ be 2 chain maps
between chain complexes. $\varphi,\psi$ induce homomorphisms:
$H(\textbf{C})\mapsto H(\textbf{D})$ and it is an important
question when they are the same.

Homotopy gives a sufficient relation that $\varphi_* = \psi_* :
H(\textbf{C})\mapsto H(\textbf{D})$


\subsection{Definition}\label{df1.2.2}
A homotopy $\Sigma :\varphi\mapsto \psi$ for chain maps
$\varphi,\psi:\textbf{C}\mapsto \textbf{D}$ is a morphism of
degree $+1$ of graded modules $\Sigma: \textbf{C}\mapsto
\textbf{D}$ such that $\psi-\varphi = \delta\Sigma +\Sigma
\delta$. ie. such that for all $n\in \Z$
$$\psi_n - \varphi_n = \delta_{n+1} \Sigma_n +
\Sigma_{n-1}\delta_n$$

\subsection{Theorem}\label{df1.2.3}

\emph{If two chain maps $\varphi,\psi:\textbf{C}\mapsto
\textbf{D}$ are homotopic then
$H(\varphi)=H(\psi):H(\textbf{C})\mapsto H(\textbf{D})$.}

\subsection*{Proof}
Let $z\in Ker \delta_n$ be a cycle in $C_n$. If $\Sigma
:\varphi\mapsto \psi$, then
$$(\psi-\varphi)z=\delta \Sigma z + \Sigma \delta z = \delta
\Sigma z$$ since $\delta z = 0$. Hence $\psi(z)-\varphi(z)$ is a
boundary in $D_n$, ie. $\psi(z)$ and $\varphi(z)$ are homologous.








The homotopy relation is very powerful because its additive
structure behaves well with respect to chain maps and additive
functors: it is an equivalence relation, and book keeping shows
that:
\begin{enumerate}
    \item If $\varphi \cong \phi:\textbf{C}\mapsto\textbf{D}$ and
    $\varphi' \cong \phi':\textbf{D}\mapsto\textbf{E}$, then
    $\phi'\phi\cong\varphi'\varphi: \mathbf{d}\mapsto \mathbf{E}$
    \item Let $F: \mathbb M_\Lambda \mapsto \mathbb M_{\Lambda'}$
    be an additive functor. If $\textbf{C}$ and $\textbf{D}$ are
    chain complexes of $\Lambda$-modules and $\phi\cong \varphi
    :\textbf{C} \mapsto \textbf D$, then $F\phi \cong
    F\varphi:F\textbf C \mapsto F\textbf D$.
    \item If $\phi \cong \varphi: \textbf C \mapsto \textbf D$ and
    if F is an additive functor, then $H(F\phi) = H(F\varphi):
    H(F\textbf C) \mapsto H(F \textbf D)$.
\end{enumerate}
However, this description is only sufficient - I now quote an
example from \cite{HS} to demonstrate it is not necessary:



\subsection{Example}\label{df1.2.4}
Take ring $\Lambda = \Z$.
\subsection*{C chain:}

$$0\rightarrow C_1\rightarrow C_0\rightarrow 0$$

$$0\rightarrow (s_1) \rightarrow^{\delta:s_1\mapsto 2s_0} (s_0)
\rightarrow 0$$

Hence, $H_1(C) = 0,\, H_0(C) = \Z/2$

\subsection*{D chain:}

$$0\rightarrow D_1 \rightarrow 0$$

$$0\rightarrow (t_1) \rightarrow 0$$

Hence $H_1(D) = 0$

Define chain map $\varphi:C\mapsto D:s_1\mapsto t_1$. Both
$\varphi$, and the Zero chain map $0:C\mapsto D$ induce zero maps
in homology. But they are not homotopic using the 3rd
compatibility condition above. Take the functor to be $-\otimes
\Z/2$, we obtain
$$\begin{array}{ccc}
  H_1(\varphi\otimes \Z/2) & \neq & H_1(0\otimes \Z/2) \\
  || &  & || \\
  1:\Z/2 \mapsto \Z/2 &  & 0:\Z/2 \mapsto \Z/2 \\
\end{array}$$





\subsection{Projective Resolutions}\label{df1.3}
\subsection{Definition}\label{df1.3.1}
The positive chain complex $\textbf C:\dots\rightarrow
C_n\rightarrow C_{n-1} \rightarrow \dots \rightarrow C_1
\rightarrow C_0 \rightarrow 0$ (with $C_n = 0$ for $n<0$) is
\textbf{Projective} if $C_n$ is projective for all $n\geq 0$, and
is \textbf{Acyclic} if $H_n(\textbf C) = 0 \,\forall\, n\geq 0$.

Equivalently, \textbf C is acyclic if and only if
$\dots\rightarrow C_n\rightarrow C_{n-1}\rightarrow \dots
\rightarrow C_1\rightarrow C_0 \rightarrow H_0(\textbf C)
\rightarrow 0$ is \textbf{exact}.

A Projective Acyclic resolution \textbf P with an isomorphism
$H_0(\textbf P) \cong A$ is a \textbf{Projective Resolution of A}.
This is a basic tool in the construction of derived functors. The
following lemma shows why they are so commonly used.

\subsection{Lemma}\label{df1.3.2}
\emph{To every $\Lambda$-module A there exists a Projective
Resolution.}
\subsection*{Proof}
Choose a projective presentation $0\rightarrow R_1\rightarrow P_0
\rightarrow A\rightarrow 0$ of A: then a projective presentation
$0\rightarrow R_2 \rightarrow P_1\rightarrow R_1\rightarrow 0$ of
$R_1$, etc. Clearly, the complex
$$\textbf P:\dots \rightarrow P_n\rightarrow^{\delta_n} P_{n-1}
\rightarrow \dots \rightarrow P_0$$ where $\delta_n:P_n\rightarrow
P_{n-1}$ is defined by $0\rightarrow P_n \rightarrow R_n
\rightarrow P_{n-1}\rightarrow 0$, is a projective resolution of
A. Since it is projective, acyclic, and $H_0(\textbf P) = A$.

This proof is instructive as every projective resolution arises in
this way - as we may collapse down at any stage to a projective
presentation. Thus, the existence of a projective presentation and
projective resolution for all modules are equivalent.

\subsection{Proposition}\label{df1.3.3}
\emph{Two Projective Resolutions os A are canonically of the same
homotopy type.}

The proof builds on a result which, given a homomorphism between
the zeroth homology of a projective chain complex and an acyclic
chain complex builds a chain map, unique up to homotopy. We have
both conditions in both chains here and we can apply the result
twice in different directions to get uniqueness.

Later on we are using these resolutions in calculations,
uniqueness up to homotopy ensures that groups cohomology groups of
derived functors are well defined, see \ref{df1.2.3}.



\subsection{Derived Functors}\label{df1.4}
As mentioned in the introduction derived functors are a
generalisation of the much older theories of $Tor$ and $Ext$,

Let $T:\mathbb M_A \mapsto \mathbb U b$ be an additive covariant
functor(in the case of $Tor:\,T_B(A) = A\otimes B$, and
$Ext:G_B(a) = Hom(A,B)$). I will defines a sequence of functors
$L_nT:\mathbb M_A\mapsto \mathbb U b,\,n=0,1,2,\dots$, these are
the "\textit{Left Derived Functors of T}".

I will build up this definition in stages:
\begin{enumerate}
    \item For a $\Lambda$-module A, \textbf P a projective
    resolution of A we defined the abelian groups $L^\textbf P_n
    T(A),\, n=0,1,\dots$ by considering the homology of the following complex:
    $$T\textbf P : \dots \rightarrow TP_n\rightarrow TP_{n-1}
    \rightarrow \dots \rightarrow TP_0\rightarrow 0$$
    and set
    $$L_n^\textbf P T(A) = H_n(T \textbf P),\, n=0,1,\dots$$

    \item For T an additive functor, the homotopy $\eta$ of any two
    projective resolutions \textbf{P,Q} of A induces an isomorphism
    $\eta: L_n^\textbf P TA\cong L_n^\textbf Q TA$,
    $n=0,1,\dots$.
    \item Hence, we are able to drop the superscript P and write
    $L_nTA$ for $L_n^\textbf P TA$ - we have free choice in
    calculations to use any projective resolution of A.

    \item   Finally, given $\alpha:A\mapsto A'$, we need to define
    induce homomorphisms $\alpha_*:L_nTA\mapsto L_nTA',\,
    n=0,1,\dots$. To do this take projective resolutions
    \textbf{P,P'} of $A,A'$ and extend $\alpha$ to a chain map
    (using the same construction as in proving uniqueness of the
    projective resolution - both complexes as acyclic and
    projective) $\alpha(\textbf{P,P'})$. This gives directly a map
    of derived functors, $$\alpha_*:L_n^\textbf P TA \mapsto
    L_n^\textbf {P'} TA'$$
\end{enumerate}

\subsection{Examples}\label{df1.4.1}
I now give two examples straight from the definition.

\begin{enumerate}
    \item Recall, a covariant functor $T:\mathbb M_\Lambda \mathbb
    U b$ is right exact if, for any sequence $A'\rightarrow
    A\rightarrow A''\rightarrow 0$ the sequence $TA'\rightarrow
    TA\rightarrow TA''\rightarrow 0$ is exact. Considering
    $A\rightarrow A\oplus B\rightarrow B\rightarrow 0$ we see (using symmetry) that
    right exact functors are additive. As an example,
    $B\otimes_\Lambda -$ is additive.
    \item If P is a projective $\Lambda$-module, then clearly,
    $\textbf P: \dots \rightarrow 0\rightarrow P\rightarrow 0 $ is
    a projective resolution of P, and taking homology of a functor
    T applied to this resolution gives: $$L_nTP = 0 \text{ for
    }n=1,2,\dots \text{ and }L_0TP = TP$$
\end{enumerate}

\subsection{Long Exact Sequences of Derived Functors}\label{df1.5}

Firstly, in \ref{df1.5.2} I will vary the object in $\mathbb M_A$
whilst keeping the functor fixed to deduce an exact sequence. But
i need 2 preliminaries :the snake lemma for keeping track of
kernels in commutative diagrams, and a lemma on existence of
projective presentations.

\subsection{Snake Lemma}\label{df1.5.1}
\emph{Let the following diagram have exact rows.
$$\begin{array}{ccccccccc}
    &  & A   & \xrightarrow{\mu} & B  & \xrightarrow{\varepsilon}  & C & \longrightarrow & 0 \\
    &   & \downarrow\alpha &   & \downarrow\beta &    & \downarrow \gamma &  &  \\
  0 & \longrightarrow & A'  & \xrightarrow{\mu'} & B' &\xrightarrow{\varepsilon'}& C' &  &  \\
\end{array}$$
Then there is a "connecting homomorphism" $\omega:ker \gamma
\mapsto coker \alpha$ such that the following sequence is exact:
$$ker\alpha\xrightarrow{\mu_*}ker\beta\xrightarrow{\varepsilon_*}ker\gamma\xrightarrow{\omega}
coker\alpha\xrightarrow{\mu'_*}coker\beta
\xrightarrow{\varepsilon'_*}coker\gamma$$ If $\mu$ is monomorphic,
so is $\mu_*$; if $\varepsilon'$ is epimorphic, so is
$\varepsilon'_*$.}

Let $c\in ker\gamma$, we choose $b\in B$ with $\varepsilon b = c$.
Since $\varepsilon'\beta b = \gamma \varepsilon b = \gamma c = 0$
there exists $a'\in A'$ with $\beta b = \mu' a'$. Define $w(c) =
[a']$, the coset of $a'$ in $coker \alpha$. This clearly has the
required properties  and prove uniqueness by contradiction.

\subsection{Projective Presentation Lemma}\label{df1.5.2}
\emph{To a short exact sequence $ A'\rightarrowtail\varphi
A\twoheadrightarrow^\phi A''$ and to projective presentations
$\varepsilon':P'\twoheadrightarrow A'$ and
$\varepsilon'':P''\twoheadrightarrow A''$ there exists a
projective presentation $\varepsilon:P\twoheadrightarrow A$ and
homomorphisms $\iota:P'\rightarrow P$ and $\pi:P\rightarrow P''$
such that the following diagram is commutative with exact rows
$$\begin{array}{ccccc}
  P' & \rightarrowtail^\iota & P & \twoheadrightarrow^\pi & P'' \\
  \downarrow \varepsilon' &  & \downarrow \varepsilon &  & \downarrow \varepsilon'' \\
  A' & \rightarrowtail^\varphi & A & \twoheadrightarrow^\phi & A'' \\
  \downarrow &                 &\downarrow &
  &\downarrow\\
  0 & & 0 & & 0
\end{array}$$
}

The proof is constructive, taking $P = P'\oplus P''$, and uses the
canonical injection of $P'$ into $P$ and the canonical projection
of $P$ onto $P''$ as well as the property that $P''$ is projective
to give the existence of a compatible function to construct
$\varepsilon$.





\subsection{Theorem - L.E.S. from varying the object}\label{df1.5.3}
\emph{Let $T:\mathbb M_A\mapsto \mathbb U b$ be an additive
functor and let
$A\rightarrowtail^{\alpha'}A\twoheadrightarrow^{\alpha''}A''$ be a
short exact sequence. Then there exist connecting homomorphisms
$$\omega_n: L_nTA''\rightarrow L_{n-1}TA',\,n=1,2,\dots$$
such that the following sequence is exact:
$$\dots\rightarrow L_nTA'\rightarrow^{\alpha'_*} L_nTA
\rightarrow^{\alpha''_*}L_nTA''\rightarrow^{\omega_n}
L_{n-1}TA'\rightarrow\dots$$
$$\dots \rightarrow L_1TA''\rightarrow^{\omega_1}
L_0TA'\rightarrow^{\alpha'_*}L_0TA\rightarrow^{\alpha''_*}L_0TA''\rightarrow
0.$$}

\subsection*{Proof}

The Projective presentation Lemma, \ref{df1.5.2} is precisely what
we need to construct the following commutative diagram with exact
rows, where $P'_0$, $P_0$, and $P''_0$ are projective:
$$\begin{array}{ccccc}
  P'_0 & \rightarrowtail & P_0 & \twoheadrightarrow & P''_0 \\
  \downarrow \varepsilon' &  & \downarrow \varepsilon &  & \downarrow \varepsilon'' \\
  A' & \rightarrowtail ^{\alpha'} & A & \twoheadrightarrow^{\varepsilon''} & A'' \\
  \downarrow &  & \downarrow &  & \downarrow \\
  0 &  & 0 &  & 0 \\
\end{array}$$
$P_0 = P'_0 \oplus P''_0$ (by construction), and using snake lemma
\ref{df1.5.1}, the kernel sequence is exact: $ker \varepsilon'
\rightarrowtail ker \varepsilon \twoheadrightarrow ker
\varepsilon''$. Repeat this procedure replacing S.E.S. of the
theorem with S.E.S. of kernels. Induction gives the exact sequence
of complexes:
$$\textbf P \rightarrowtail^{\alpha'} \textbf P \twoheadrightarrow
^{\alpha''} \textbf P''$$

For each $n\geq 0$, $P_n = P'_n \oplus P''_n$, and $T$ is additive
hence we get short exactness of $0\rightarrow T\textbf P'
\rightarrow T\textbf P \rightarrow T\textbf P'' \rightarrow 0$.
Then the construction of long exact homology sequence of complexes
yields maps $\omega_n : H_n(T\textbf P'')\rightarrow H_{n-1}
(T\textbf P')$ and also the exactness of the sequence. Homotopy
considerations give independence from choice of resolution and
chain maps.


There is a similar theorem for varying the functor. To quote it
precisely we need first a definition.

\subsection{Definition}\label{df1.5.4}
A sequence $T'\rightarrow^{\varepsilon'}
T\rightarrow^{\varepsilon''} T''$ of additive functors
$T',T,T'':\mathbb M_\Lambda \rightarrow \mathbb Ub$ and natural
transformations $\varepsilon'. \varepsilon''$ is called
\textit{exact on projectives} if, for every projective
$\Lambda$-module P, the sequence $0\rightarrow
T'P\rightarrow^{\varepsilon'_P}
TP\rightarrow^{\varepsilon''_P}T''P\rightarrow 0$ is exact.


\subsection{Theorem - L.E.S. from varying the
functor}\label{df1.5.5} Let the sequence of additive functors
$T'\rightarrow^{\varepsilon'} T\rightarrow^{\varepsilon''} T''$ of
additive functors $T',T,T'':\mathbb M_\Lambda \rightarrow \mathbb
Ub$ be exact on projectives. Then, for every $\Lambda$-module A,
there are connecting homomorphisms $\omega_n:L_nT''A\rightarrow
L_{n-1}T'A$ such that the sequence
\begin{eqnarray}
\nonumber  &\dots&\rightarrow
L_nT'A\rightarrow^{\varepsilon'}L_nTA\rightarrow^{\varepsilon''}L_nT''A\rightarrow^{\omega_n}
L_{n-1}T'A\rightarrow\dots\\
\nonumber  &\dots&\rightarrow L_1T''A\rightarrow^{\omega_1}
L_0T'A\rightarrow^{\varepsilon'} L_0TA\rightarrow{\varepsilon''}
L_0T''A\rightarrow 0
\end{eqnarray} is exact.

\subsection*{Proof}
Choose a projective resolution \textbf P of A, then exactness on
projectives gives exactness of $0\rightarrow T'\textbf
P\rightarrow^{\varepsilon'_\textbf P} T\textbf
P\rightarrow^{\varepsilon''_\textbf P}T''\textbf P\rightarrow 0$
and applying Long Exact Homology Sequence of Complexes gives the
result.


\subsection{Application: the functor $Ext^n_\Lambda$}\label{df1.6}
\subsection{Definition}\label{df1.6.1}
The abelain groups $Ext_\Lambda^n(A,B)$ are calculated as follows:
\begin{itemize}
    \item Choose a projective resolution \textbf P of A.
    \item Form complex $Hom_\Lambda (\textbf P, B)$
    \item Take cohomology to find $Ext^n_\Lambda$ groups
\end{itemize}
ie. Consider the right derived functors of the additive covariant
functor \\ $Hom_\Lambda(-,B)$, and define:
$$Ext^n_\Lambda (-,B) = R^n(Hom_\Lambda (-,B)),\, n=0,1,2,\dots$$

Recall, the definition of $Ext(A,B)$ of extensions of A by B as
being the abelian group which gives exactness of \\
$\dots \rightarrow Hom_\Lambda(P_0,B)\rightarrow
Hom_\Lambda(R_1,B)\rightarrow Ext_\Lambda(A,B)\rightarrow 0$
\\
where $R_1\rightarrowtail P_0\twoheadrightarrow A$ is any
projective presentation of A.

From the right derived functor analogue of L.E.S. \ref{df1.5.3} we
have \\
$\dots \rightarrow Ext^0_\Lambda(P_0,B)\rightarrow
Ext^0_\Lambda(R_1,B)\rightarrow Ext^1_\Lambda(A,B)\rightarrow 0$

Observing that $Ext^0_\Lambda = Hom_\Lambda (A,B)$ (since the
functor $Hom$ is left exact) we get
$$Ext^1_\Lambda (A,B) \cong Ext_\Lambda (A,B)$$ and the notation
is justified.

The following characterises projective modules and the functor
$Ext^n_\Lambda$:

\subsection{Proposition}\label{1.6.2}
The following are equivalent:
\begin{enumerate}
    \item A is a projective $\Lambda$-module.
    \item $Hom_\Lambda (A,-)$ is an exact functor.
    \item $Ext^i_\Lambda (A,B)$ vanishes for all $i\neq 0$ and all
    B. In other words, A is \\ $Hom_\Lambda (-,B)$-acyclic for all
    B.
    \item $Ext_\Lambda^1 (A,B)$ vanishes for all B.
\end{enumerate}

\textbf{$1\Longrightarrow 4$} Using equivalence of $Ext^1$ and
$Ext$: $Ext_\Lambda(A,B)$ is in 1-1 correspondence with the set
$E(P,B)$, consisting of classes of extensions of the form
$B\rightarrowtail E\twoheadrightarrow A$. But from universal
property of projective A, short exact sequences of this form
split. Hence $E(A,B)$ consists of just one element, the zero
element.


 \textbf{$4\Longrightarrow 3$} The key here is that the vanishing
 of the first Ext group applies to all modules B (if it was just a
 particular one it is conceivable to continue the exact sequence
 back with a series of isomorphisms). This allows us to use
 dimension shifting, where we truncate a given projective
 resolution so that the second cohomology group of the original
 resolution is calculated by the first of the new one, and so by
 induction all Ext groups ($n\geq 0$) are calculated from the
 first.
 Let $$\textbf P: \dots \rightarrow P_2 \rightarrow _{a_2} P_1
 \rightarrow_{a_1} P_0 \twoheadrightarrow_{a_0}B \rightarrow 0$$
 be a projective resolution of B. For calculations, consider
$$\dots \rightarrow  Hom_\Lambda(A,P_2) \rightarrow _{{a_2}_*}
Hom_\Lambda(A,P_1) \rightarrow_{{a_1}_*} Hom_\Lambda(A,P_0)
\rightarrow 0$$ hypothesis (4) gives that
$$H_1(\textbf P)= Ext_\Lambda^1(A,B) = \frac{ker \{ Hom(A,P_2)\rightarrow
Hom(A,P_1)\}}{im \{ Hom(A,P_1)\rightarrow Hom(A,P_0)\}} =
\frac{ker\,{a_1}_*}{im\,{a_2}_*}=0$$.

We can amend the projective resolution to one of $P_0/B$ which
preserves the same maps:
$$\dots \rightarrow P_2 \rightarrow _{a_2} P_1
 \twoheadrightarrow_{a_1} P_0/B \rightarrow 0$$
 Then we can calculate, by hypothesis (4) the first homology of \textbf P',
$$H_1(\textbf P')=Ext_\Lambda^1(A,P_0/B) =0 = \frac{ker\,{a_2}_*}{im\,{a_3}_*}
= Ext_\Lambda^2(A,B)$$. Similarly by induction, we have
$Ext^i_\Lambda (A,B)=0\text{ for all } i\geq 0$, hence (3).

\textbf{$3\Longrightarrow 2$} Apply L.E.S. \ref{df1.5.3} to
$M'\rightarrowtail M \twoheadrightarrow M''$. All higher terms
vanish by (3), and so L.E.S. collapses to $Hom_\Lambda
(A,M')\rightarrowtail Hom_\Lambda (A,M)\twoheadrightarrow
Hom_\Lambda (A,M'')$ and hence functor $Hom_\Lambda (A,-)$ is
exact.

\textbf{$2\Longrightarrow 1$} Given that $Hom_\Lambda (A,-)$ is
exact and that we are given a surjection $g:B\rightarrowtail C$
and a map $\gamma :A\rightarrow C$. We can lift $\gamma\in
Hom_\Lambda (A,C)$ to $\beta \in Hom_\Lambda (A,B)$ such that
$\gamma = g_* \beta = g \circ \beta$ because $g_*$ is onto. Thus A
has the universal lifting property, hence A is projective.




\subsection{Examples}\label{1.6.3}
I now give 3 explicit calculations working over the ring $\Z$, and
viewing abelian groups as $\Z$-modules.
\begin{enumerate}
    \item Let $A=\Z /p$ then we have the resolution $$0\rightarrow
    \Z\rightarrow^{p} \Z\rightarrow \Z /p\rightarrow 0$$ Since
    $Hom(\Z,B)\cong B$ we see that to calculate $Ext^*(\Z/p,B)$ we
    need to take the cohomology of $0\rightarrow B\rightarrow^p
    B\rightarrow  0$. Hence,
    $$Ext_\Z^n =  \begin{cases}
      pB   &   \text{ if } n=0 \\
      B/{pB} &   \text{ if } n=1 \\
      0    &   \text{ if } n\geq 2 
    \end{cases}$$

    \item We know, embedding B in an injective abelian group $I^0$ and
taking its quotient $I^1$ (which is divisible, and hence again exact)
we get that \\ \textbf{$\mathbf{Ext_\Z^n (A,B) = o}$ for $n\geq 2$
and all abelian groups A,B}.

We can calculate the small terms exactly since every finite
abelian group may be expressed in the form $B\cong \Z^m\oplus
\Z/p_1\oplus \dots \oplus \Z/p_n$. And hence we may calculate the
Ext groups of any pair of abelian groups.

    \item Now take $B=\Z$. We have the injective resolution
    $0\rightarrow \Z\rightarrow \Q\rightarrow \Q /\Z \rightarrow
    0$. This gives $Ext^0_\Z (A,\Z) = 0$ and $Ext_\Z^1 (A,\Z) =
    A^*= Hom(A,\Q / \Z)$, the Pontrajagin dual.
\end{enumerate}

\subsection{$Ext_\Lambda^n\text{ where }n\geq 1$ viewed as extensions}\label{1.6.4}

Recall, from \ref{df1.6.1} that $Ext_\Lambda (A,B) = Ext_\Lambda^1
(A,B)$ can be interpreted as the group of equivalence classes of
extensions. I would like to quote Yoneda's ideas on how to expand
this idea using "n-extensions".

An n-extension of A by B is an exact sequence $\textbf E :
0\rightarrow B\rightarrow E_n \rightarrow \dots \rightarrow E_1
\rightarrow A \rightarrow 0 $ of $\Lambda$-modules.

So above, an extension is a 1-extension. We now need a notion of
equivalence of n-extensions. We first define a non-symmetric
relation for $n\geq 2$. We shall say that the n-extensions
$\textbf {E,E'}$ satisfy the relation $\textbf E \rightsquigarrow
E'$ if there is a commutative diagram:
$$\begin{array}{cccccccccccc}
  \textbf E :  & 0 & \rightarrow & B         & \rightarrow & E_n        & \rightarrow \dots \rightarrow & E_1        & \rightarrow & A & \rightarrow & 0 \\
               &   &             & \parallel &             & \downarrow &                               & \downarrow &             & \parallel &  &  \\
  \textbf E' : & 0 & \rightarrow & B         & \rightarrow & E'_n       & \rightarrow \dots \rightarrow & E'_1       & \rightarrow & A & \rightarrow & 0 \\
\end{array}$$

We now induce an equivalence relation. We say \textbf{E is
equivalent to $E'$}:

$$E \sim E' \Longleftrightarrow \exists \text{ a chain } E=E_0,
E_1, \dots , E_k = E' \text{ with } E_0\rightsquigarrow E_1
\leftrightsquigarrow E_2 \rightsquigarrow \dots
\leftrightsquigarrow E_k$$

Denoting [E] for the equivalence class of the n-extension E, and
$Yext^n_\Lambda (A,B)$ for the set of such classes of n-extensions
it can be shown that $Yext^n_\Lambda (-,-)$ is a bifunctor and
that there is a natural equivalence of set valued bifunctors:
$$Yext^n_\Lambda (-,-)\cong Ext^n_\Lambda (-,-)$$
Giving a more tangible interpretation of the higher $Ext$ groups.





